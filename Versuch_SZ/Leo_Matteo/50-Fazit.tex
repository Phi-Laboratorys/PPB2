%Matteo Kumar - Leonard Schatt
% Fortgeschrittenes Physikalisches Praktikum

% 5. Kapitel Einleitung

\chapter{Fazit}
\label{chap:fazit}


% Platz für Text
In diesem Versuch wurde sich ausgiebig mit verschiedenen Solarzellen beschäftigt. An den U/I-Kennlinien wurde sichtbar, dass Solarzellen nach den 
selben Prinzipien wie eine Diode funktioniert, was das Ersatzschaltbild aus den Grundlagen bestätigt. Zudem wurden viele Faktoren betrachtet, 
an denen die Güte einer Solarzelle gemessen werden kann. Dabei stellte sich heraus, dass die reinen Siliziumzellen wesentlich effektiver
sind als die CIS-Zellen. Wichtig war auch die Beschäftigung mit zwei wichtigen Geräten, dem Lock-in-Verstärker und dem 
Vier-Quanten-Netzgerät. Ersterer war, vor allem wenn man die geringen berechneten Ströme betrachtet, sehr sinnvoll, um ein geeignetes 
Messergebnis zu bekommen, während das Netzgerät die Sicherheit der Messung gewährleistete, indem es durch die freie Wahl des zu regelnden 
Parameters einer Überlasung und damit einer Beschädigung der Solarmodule entgegenwirkte.