% Teilaufgabe 11
\newpage
\section{Möglichkeiten der Signal-Rausch Verbesserung}
\label{sec:verbesserung}
Dieser Abschnitt behandelt die Methoden zur Signal/Rausch-Verbesserung und dessen Umsetzung.
\subsection{Filter}
\label{sub:filter}
Eine Möglichkeit zur Signal/Rausch-Verbesserung ist der Einsatz eines Filters. Die Aufgabe des Filters ist hierbei die Unterdrückung bestimmter Frequenzen. Dabei ist zu beachten, dass das Signal bei einer anderen Frequenz auftritt als das Rauschen, sonst würden man nämlich das Signal mit filtern. Bei der Filterung wird die Bandbreite das Rauschen stark reduziert (Zu sehen an Gleichung (2.1) und (2.3)), aber auch Störstrahlen aus der Umgebung werden minimiert \citep{VA}.
\subsection*{Filtertypen}
\begin{itemize}
    \item[1)]\textbf{Tiefpass}\\
    Ein Tiefpassfilter filtert Frequenzen \textbf{oberhalb} der Grenzfrequenz heraus und lassen Frequenzen unterhalb nahezu ungedämpft durch. Die Grenzfrequenz ist dadurch charakterisiert, dass das Ausgangssignal zu dieser Frequenz um 3dB kleiner ist (ab dort beginnt der Durchlassbereich) \citep{electronik}.
    \item[2)]\textbf{Hochpass}\\
    Ein Hochpassfilter filtert Frequenzen \textbf{unterhalb} der Grenzfrequenz heraus und lassen Frequenzen oberhalb nahezu ungedämpft durch. Ein Hochpassfilter ist das Gegenstück zum Tiefpassfilter \citep{electronik}. Ein Hochpassfilter kann dazu verwendet werden, die 50Hz Brummspannung aus dem Messsignal zu filtern \citep{VA}.
    \item[3)]\textbf{Bandpass}\\
    Ein Bandpass sperrt Frequenzen \textbf{unter- und oberhalb} eines definierten Frequenzbandes. Das Frequenzband ist durch die 3dB-Bandbreite um die Mittenfrequenz charakterisiert. Diese Art des Filters ist eine Reihenschaltung aus Tiefpass- und Hochpassfilter und die Bandbreite wird durch die Grenzfrequenzen der jeweiligen Filter festgelegt \citep{electronik}. Der Bandpass findet heirbei Einsatz bei der Filterung von breitbandigen Rauschen \citep{VA}.
    \item[4)]\textbf{Bandsperre/Notch-Filter}\\
    Ein Bandsperre sperrt einen schmalen Frequenzbereich innerhalb eines breiten Frequenzbandes und kann wie das Gegenstück zu einem Bandpass angesehen werden \citep{electronik}.
\end{itemize}
In Abbildung \ref{image:verlauf} ist der generelle Verlauf der jeweiligen Filtertypen im Frequenzbereich dargestellt.
\newpage
\begin{center}
    \includegraphics[scale=0.3]{VerlaufFilter.png}
    \captionof{figure}{Verlauf der jeweiligen Filtertypen im Frequenzbereich \citep{grafikQuelle}}
    \label{image:verlauf}
\end{center}
\subsection*{Ordnung eines Filters}
Die Ordnung eines Filters gibt an wie oft ein Filter hintereinander in Reihe geschaltet wurde. Somit wären zwei Tiefpassfilter in Reihe geschaltet ein Tiefpassfilter 2.Ordnung. Je höher die Ordnung eines Filters ist, desto steiler ist die sogenannte Flankensteilheit (Steigung der Flanken in Abbildung \ref{image:verlauf}). Es ist aber zu beachten, dass sich durch die Ordnung die Phase in der Nähe der Grenzfrequenz verändern kann \citep{electronik}.
\subsection*{Arten der Filterimplementierung}
\begin{itemize}
    \item[1)]\textbf{Butterworth-Tiefpassfilter}\\
    Der Butterworth-Tiefpassfilter besitzt einen lagen horizontalen Frequenzgang, welcher erst kurz vor der Grenzfrequenz scharf abknickt. In der Sprungantwort lässt sich ein kräftiges Überschwingen (abhängig von der Ordnung) registrieren \citep{VA}.
    \item[2)]\textbf{Tschebyscheff-Tiefpassfilter}\\
    Ein Tschebyscheff-Tiefpassfilter besitzt oberhalb der Grenzfrequenz einen noch steileren Abfall als der Butterworthfilter, womit das Überschwingen der Sprungantwort im Vergleich noch stärker ist. Im Durchlassbereich verläuft die Verstärkung aber nicht monoton, sondern wellig mit konstanter Amplitude \citep{VA}.
    \item[3)]\textbf{Bessel-Tiefpassfilter}\\
    Der Bessel-Tiefpassfilter besitzt unter der Voraussetzung, dass die Phasenverschiebung in einem bestimmten Frequenzbereich proportional zur Frequenz ist, ein optimales Rechteck-Übertragungsverhalten. Der Frequenzgang knickt aber nicht so stark ein, wie bei den zwei vorher genannten Filter \citep{VA}.
    \item[4)]\textbf{Tschebyscheff-Tiefpass}\\
    Der Tschebyscheff-Tiefpass besitzt nach der Grenzfrequenz einen steilen Knick im Frequenzgang, wodurch auch dieser Filter eine Überschwingung der Sprungantwort aufweist. Dies hat zur Folge, dass der Frequenzgang im Durchlassbereich eine Welligkeit besitzt. Durch Verminderung dieser Welligkeit geht der Tschebyscheff kontinuierlich in den Butterworth über \citep{VA}. 
\end{itemize}

\subsection{Signalmittelung}
\label{sub:mittelung}
Bei stark verrauschten Signalen wird die Signalmittelung verwendet. Zu beachten ist aber, dass die Bandbreiten des Signals und des Rauschens in derselben Größenordnung liegen, was die Anwendung eines Filters ausschließen würde. Eine weitere Voraussetzung ist, dass das Signal wiederholbar und dessen Phase bekannt ist. Bei der Signalmittelung wird das Signal mit Rauschen in $n$ Segmente unterteilt und in $n$ Kanälen gespeichert. Dieser Vorgang wird mehrmals wiederholt und jeder neue Durchgang zum vorhandenen Speicherinhalt addiert, was ein Anwachsen des Signals proportional zu den Wiederholungen verursacht. Beim Rauschen aufgrund seiner statischen Natur wird nur der quadratischen Mittelwert auf das vorhandene Messsignal addiert. Allgemein erhält man für $N$ Wiederholungen eine Signal/Rausch-Verbesserung von $\sqrt{N}$ \citep{VA}. In Abbildung \ref{image:mittelung} wird die erzielte Signal/Rausch-Verbesserung durch ein Vorher-Nachher-Bild gezeigt.
\begin{center}
    \begin{tabular}{c c}
        \includegraphics[scale = 0.3]{Mittelung1.png} &
        \includegraphics[scale = 0.3]{Mittelung2.png}  
    \end{tabular}
    \captionof{figure}{Anwendung Signalmittelung: Links Signalmessung und Rechts Mehrfachmessung \citep{VA}}
    \label{image:mittelung}
\end{center}
\newpage
\subsection{Lock-In Verstärker}
\label{sub:lockin}
Ein Lock-In Verstärker ist ein Detektor zum Auflösen von kleinen Wechselspannungssignalen bis zu ein paar nV. Dabei ist eine akkurate Messung des Signals, welches 1000\,fach stärkeres Rauschen besitzt, möglich. Es kommt dafür eine sogenannte phasensensitive Detektion zum Einsatz, um eine bestimmte Komponente des Signals bei einer bestimmten Referenzfrequenz $f_{r}$ und Phase $\Theta_{r}$ zu bestimmen. Dabei wird Rauschen, welches nicht auf der Referenzfrequenz liegt, herausgefiltert und trägt nicht mehr zum Signal bei.
\subsection*{Phasensensitive Detektion}
Beim Lock-In wird wie oben erwähnt eine Referenzfrequenz benötigt. Diese Frequenz wird meistens durch das Experiment vorgegeben (z.B. Funktionsgenerator) und der Lock-In detektiert die Antwort des Experiments bei dieser Referenzfrequenz. In Abbildung \ref{image:signalRef} sind schematisch die jeweiligen Signale gezeigt. Als Referenzsignal wird eine Rechteckschwingung mit $f_{r}$ verwendet und das Experiment selbst wird von einer Sinusschwingung (Input-Signal) mit Funktion $s(t)=U_{s}\sin(\omega_{r}t + \Theta_{s})$ mit $\omega_{r}=2\pi f_{r}$, Amplitude $U_s$ und Phasenverschiebung $\Theta_{s}$ angeregt. Der Lock-In Verstärker erzeugt dann selbst eine Sinusschwingung mit Frequenz $f_l$ und der Funktion $l(t) = \sin(\omega_{l}t + \Theta_{r})$ mit $\omega_{l}=2\pi f_{l}$ als Signal, das Lock-In-Signal.
\begin{center}
    \includegraphics[scale = 0.25]{SignalLockIn.png}
    \captionof{figure}{Signale bei einem Lock-In Verstärker \citep{lockin}}
    \label{image:signalRef}
\end{center}
Nun multipliziert der Lock-In Verstärker das Input-Signal und das Lock-In-Signal mit einem phasensensitiven Detektor (PSD) oder einem Multiplikator. Daraus folgt mit Anwendung des Additionstheorems des Sinus:
\begin{gather}
    \begin{aligned}
        s_{x} &= s(t)\cdot l(t) = U_{s}\sin(\omega_{r}t + \Theta_{s}) \cdot \sin(\omega_{l}t + \Theta_{r})\\
                &= \frac{U_{s}}{2}\left[\cos((\omega_{r}-\omega_{l}) + (\Theta_s - \Theta_r)) + \cos((\omega_{r}+\omega_{l}) + (\Theta_s + \Theta_r))\right]
    \end{aligned}
\end{gather}
Damit ist das Output-Signal $s_x$ zwei Wechselspannungssignale mit unterschiedlichen Frequenzen. Dieses Signal wird durch einen Tiefpassfilter geschickt, welcher alle Wechselspannungssignale ab der Grenzfrequenz $f_g$ eliminiert. Die Grenzfrequenz $f_g$ wird dabei am Lock-In Verstärker über die Zeitkonstante $\tau$ eingestellt mit der Beziehung: $f_g\sim\frac{1}{\tau}$\\ Wenn $\omega_r = \omega_l$ ist, ist das Signal mit der Differenz der Frequenz ein Gleichspannungssignal und wird somit nicht gefiltert. Der neue PSD Output ist dann:
\begin{gather}
    s_{x} = \frac{U_s}{2} \cos(\Theta_s-\Theta_r) = \frac{U_s}{2} \cos(\Delta\Theta)  
\end{gather}
Bei einem Phasenunterschied von $\Delta\Theta = 0$ lässt sich hierbei das Maximum der Amplitude messen ($\frac{U_s}{2}$) und bei $\Delta\Theta = \frac{\pi}{2}$ misst man überhaupt kein Output-Signal mehr. Solche Lock-In Verstärker mit nur einem PSD werden auch Einphasen-Lock-In genannt. Man kann aber auch ein zweites Lock-In-Signal mit Cosinusschwinung verwenden und dieses mit einem zweiten PSD mit dem Input-Signal multiplizieren. Mit demselben Prozess wie zuvor erhält man die Beziehung:
\begin{gather}
    s_{y} = \frac{U_s}{2} \sin(\Delta\Theta) 
\end{gather} 
Das Output $s_x$ heißt \enquote{in Phase}-Komponente und $s_y$ die \enquote{Quadratur}-Komponente, weil bei $\Delta\Theta = 0$ ist $s_y = 0$ und $s_x$ misst das Input-Signal.\\

Weiterhin kann man mit $s_x$ und $s_y$ wie folgt die Amplitude des Output-Signals bestimmen:
\begin{gather}
    A = (s_x^2 + s_y^2)^{\frac{1}{2}} = \frac{U_s}{2}
\end{gather}
\subsection*{Was wird mit einem Lock-In gemessen?}
Ein Lock-In Verstärker misst aufgrund der Multiplikation mit einer Sinusschwingung (Cosinusschwingung) den ersten Term der Fourier-Reihe des Input-Signals bei der Referenzfrequenz $f_r$. Dabei ist aber zu beachten, dass auch Rauschen, welches bei der Referenzfrequenz auftritt, mit gemessen wird und gegebenenfalls abgezogen werden muss. Dies erkennt man daran, dass das Messsignal am Lock-In Verstärker Schwankungen aufweist.\\ In Abbildung \ref{image:Block} ist zusätzlich der Aufbau des Lock-In Verstärkers dargestellt \citep{lockin}.
\newpage
\begin{center}
    \includegraphics[scale = 0.6]{BLockdiagrammLockIn.png}
    \captionof{figure}{Blockdiagramm Lock-In Verstärker \citep{lockin}}
    \label{image:Block}
\end{center}

\newpage
\section{Leistungspegel}
\label{sec:pegel}
Der Leistungspegel $L_p$ gibt das 10fache logarithmische Verhältnis zwischen Nutzleistung $P$ und Bezugsleistung $P_0$ in \si{\deci\bel} an und ist definiert als:
\begin{gather}
    L_P = 10 \log_{10}\left(\frac{P}{P_0}\right)
    \label{eq:pegel}
\end{gather}
In diesem Versuch ist vom Interesse den Leistungspegel der Spannungen $L_U$ anzugeben. Dafür wird die Formel $P = U \cdot I$ mit dem ohmischen Gesetz $R = \frac{U}{I}$ umgestellt zu $P = \frac{U^2}{R}$ und in Gleichung (\ref{eq:pegel}) eingesetzt. Somit erhält man:
\begin{gather}
    L_U = 10 \log_{10}\left(\frac{U^2}{U_0^2}\right) = 20 \log_{10}\left(\frac{U}{U_0}\right)
    \label{eq:spannungpegel}
\end{gather}
Damit entspricht der Leistungspegel der Spannungen das doppelte des herkömmlich definierten Leistungspegel \citep{electronik}.\\
Um das Verständnis zu vertiefen werden noch einige bestimmte Verhältnisse zu charakteristischen \si{\deci\bel} angegeben:
\begin{center}
    \begin{tabular}{c | c c}
        $L$/dB & $\frac{P}{P_0}$ & $\frac{U}{U_0}$\\
        \hline
        $-n\cdot10$ & $1\cdot10^{-n}$ &  $1\cdot10^{-n/2}$ \\
        -20 & 1/100 & 1/10 \\
        -10 & 1/10 & 1/$\sqrt{10}$ \\
        -6 & 1/4 & 1/2\\
        -3 & 1/2 & 1/$\sqrt{2}$\\
        0 & 1 & 1\\
        3 & 2 & $\sqrt{2}$\\
        6 & 4 & 2\\
        10 & 10 & $\sqrt{10}$ \\
        20 & 100 & 10 \\
        $n\cdot10$ & $1\cdot10^{n}$ &  $1\cdot10^{n/2}$ \\
    \end{tabular}
    \captionof{table}{Leistungspegel und zugehörige Verhältnisse}
    \label{tab:pegel}
\end{center}
In Tabelle \ref{tab:pegel} lässt sich sehr gut die Verdopplungsregel des Leistungspegels erkennen. Dabei steigt der Pegel um 3\,dB an (bei $L_U$ um 6\,dB), wenn man das Verhältnis verdoppelt. Dies hat damit zu tun, dass der 10er-Logarithmus von 2 ungefähr 0.3 ist und dieser mit der Multiplikation von 10 nach der Gleichung (\ref{eq:pegel}) dann 3\,dB ergibt.
\newpage
\section{Theorem von Nyquist}
Das Theorem von Nyquist oder auch Abtasttheorem besagt, dass aus den Abtastwerten das ursprüngliche Signal (kontinuierlich) fehlerfrei rekonstruiert werden kann, wenn die Abtastfrequenz mindestens doppelt so groß ist wie die höchste Signalfrequenz $f_{max}$. 
\begin{gather}
    f \geq 2\cdot f_{max}
\end{gather} 
Die Frequenz $2\cdot f_{max}$ wird als \textit{Nyquist-Frequenz} bezeichnet. Aus dem Abtasttheorem folgt auch, dass das Spektrum des Signals \textit{bandbegrenzt} ist, d.h. das Signal im Spektrum muss ab der maximalen Frequenz $f_{max}$ gleich 0 sein \citep{praktikum}.
\section{Fouriertransformation und Schnelle Fouriertransformation}
Bei einer Fouriertransformation wird ein gegebenes Signal (ggf. eine Funktion) komplett in den Frequenzraum transformiert, dabei wird ein Integral von $-\infty$ bis $\infty$ ausgewertet. Der Rechenaufwand einer Fouriertransformation ist in der Regel in der Praxis sehr hoch, weswegen meist eine Schnelle Fouriertransformation (engl. Fast Fourier Transformation (FFT)) durchgeführt wird. Unter einer FFT versteht man eine effiziente Realisierung der Diskrete Fouriertransformation (DFT), mit der redundante Rechenschritte vermieden werden. Bei einer DFT wird nur ein abgetastetes Signal in den Frequenzraum überführt. Der Rechenaufwand von $N^2$ bei einer DFT verringert sich dann  bei einer FFT zu etwa $N\log_2\left(N\right)$ \citep{praktikum}.
\section{Fourier-Reihe und Effektivspannung}
\label{sec:fourierseries}
\subsection*{Allgemeines zur Fourier-Reihe und Effektivspannungen}
\label{sub:fourierseriesAllgemein}
Eine Fourier-Reihe zerlegt eine gegeben periodische Funktion in ihre jeweiligen Sinus und Cosinusanteile. Die reelle Fourier-Reihe einer bestimmten $T$-periodischen Funktion lässt sich mit den folgenden Formeln berechnen:
\begin{gather}
    f(t) = \frac{a_0}{2} + \sum^{\infty}_{k=1} \left[a_k \sin(k\frac{2\pi}{T} t) +b_k \cos(k\frac{2\pi}{T} t)\right]\\
    a_k = \frac{2}{T} \int^{\frac{T}{2}}_{-\frac{T}{2}} f(t)\cos(k \frac{2\pi}{T} t)dt \tab
    b_k = \frac{2}{T} \int^{\frac{T}{2}}_{-\frac{T}{2}} f(t)\sin(k \frac{2\pi}{T} t)dt
\end{gather}
Dabei sind die Grenzen der Integrale von $-\frac{T}{2}$ bis $\frac{T}{2}$ nicht fest, sie können verschoben werden. Es ist aber wichtig, dass über eine Periode integriert wird in diesem Fall über eine komplette Periodendauer $T$ \citep{praktikum}.\\

Um die effektiven Spannungswerte der jeweiligen Schwingungsform zu bestimmen, bildet man das sogenannte \enquote{Quadratische Mittel}. Dieses ist wie folgt definiert:
\begin{gather}
    U_{eff} = \sqrt{\frac{1}{T}\int^T_0 f(t)^2 dt}
\end{gather}
Hierbei ist es wieder zu erwähnen, dass die Grenzen der Integration nicht fest sind, aber die Integration über eine Periodenlänge erfolgen muss \citep{messtechnik}.\\

Die detaillierteren Berechnungen zu jeder Schwingungsform lassen sich in Anhang \ref{app:Berechnung} nachlesen.

\subsection*{Sinusschwingung}
\label{sub:sinus}
Der Fall der Sinusschwingung ist besonders einfach, da wie vorangegangen erwähnt, die Fourier-Reihe eine periodische Funktion in ihre Sinus und Cosinusanteile zerlegt. Daraus folgt die Fourier-Reihe der Sinusschwingung ist die Sinusschwingung selbst und kann somit trivial angegeben werden als:
\begin{gather}
    \boxed{f(t) = U_0\sin(\frac{2\pi}{T} t)}
\end{gather}
Die Effektivspannungen der Sinusschwingung:
\begin{gather}
    \boxed{U_{eff}=\frac{U_0}{\sqrt{2}}\approx 0,70711 \cdot U_0}
\end{gather}

\subsection*{Rechteckschwingung}
\label{sub:square}
Als Nächstes wird die Fourier-Reihe der Rechteckschwingung bestimmt. Diese hat die Form:
\begin{gather}
    f(t) = 
    \begin{cases}
        +U_0, & 0 \leq t \leq \frac{T}{2} \\
        -U_0, & \frac{T}{2} \leq t \leq T \\
    \end{cases}
\end{gather}
Da die Funktion der Rechteckschwingung punktsymmetrisch zum Ursprung ist, fallen alle Cosinusanteile weg, da $a_k = 0$. Somit muss nur $b_k$ wie folgt berechnet werden:
\begin{gather}
    b_k =
    \begin{cases}
        0, & k~\text{gerade}\\
        \frac{4U_0}{\pi}\frac{1}{k}, & k~\text{ungerade}\\
    \end{cases}
\end{gather} 
Es werden nur noch Terme mit ungeraden $k$ betrachtet und man erhält:
\begin{gather}
    \boxed{f(t) = \frac{4U_0}{\pi} \sum^{\infty}_{k=1} \frac{1}{2k-1} \sin((2k-1)\frac{2\pi}{T}t)}
\end{gather}
Die Effektivspannungen der Rechteckschwingung:
\begin{gather}
    \boxed{U_{eff} = U_0}
\end{gather}
\subsection*{Dreiecksspannung}
\label{sub:triangle}
Als Letztes wollen wir die Dreieckschwingung betrachtet. Die Form dieser ist definiert wie folgt:
\begin{gather}
    f(t) = 
    \begin{cases}
        at, & -\frac{T}{4} \leq t \leq \frac{T}{4} \\
        a\left(\frac{T}{2}-t\right), & \frac{T}{4} \leq t \leq \frac{3T}{4} \\
    \end{cases}
    ~\text{mit}~U_0 = \frac{aT}{4}
\end{gather} 
Die Funktion ist erneut punktsymmetrisch zum Ursprung, wodurch wieder alle $a_k$-Koeffizienten 0 sind. $b_k$ ergibt sich dann durch wie folgt:
\begin{gather}
    b_k =
    \begin{cases}
        0, & k~\text{gerade}\\
        \frac{8U_0}{\pi^2}\frac{(-1)^{k-1}}{k^2}, & k~\text{ungerade}\\
    \end{cases}
\end{gather}
Es werden wieder nur die Terme mit ungeraden $k$ betrachtet. Somit erhält man:
\begin{gather}
    \boxed{f(t) = \frac{8U_0}{\pi^2} \sum^{\infty}_{k=1} \frac{(-1)^{k-1}}{(2k-1)^2} \sin((2k-1)\frac{2\pi}{T}t)}
\end{gather} 
Die Effektivspannungen der Dreieckschwingung:
\begin{gather}
     \boxed{U_{eff} = \frac{U_0}{\sqrt{3}}\approx 0,57735 \cdot U_0}
\end{gather}
\section*{Gemeinsamkeiten von Rechteck und Dreiecksschwingung}
Vergleicht man die Fourierreihe der Rechteck und Dreiecksschwingung erkennt man, dass beide Reihen nur aus Sinusschwingungen bestehen mit den selben Argumenten. Weiterhin ist zu erwähnen, dass die Reihen nur ungerade $k$ besitzen. Die einzigen Unterschiede treten auf bei den Vorfaktoren und den Summenkoeffizienten. Bei den Summenkoeffizienten besitzt die Rechteckschwingung eine $\frac{1}{k}$-Faktor, während die Dreiecksschwingung einen $\frac{1}{k^2}$-Faktor, welcher alterniert, aufweist. Dieser Vergleich zeigt deutlich, dass zwischen den beiden Schwingungsformen in ihrem Aufbau kein großer Unterschied in der Fourierreihe besteht, obwohl die Signal verschiedene Eigenschaften und Formen besitzen.