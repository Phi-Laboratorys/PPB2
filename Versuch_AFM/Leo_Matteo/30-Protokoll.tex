% Charlotte Geiger - Manuel Lippert - Leonard Schatt
% Physikalisches Praktikum

% 3.Kapitel  Protokoll

% Variables
\def\skalierung{0.65}

\chapter{Methodik}
\label{chap:protokoll}
\section{Aufbau}

Der hier verwendete Aufbau besteht aus einem AFM der Marke Nanosurf, einer dazugehörenden Steuerelektronik und einem Computer mit dem 
Programm  'AFM  Nanosurf  Easy  Scan  2'. An diesem kann der Messbereich, der Setpoint, die Zeit pro Zeile und die Anzahl der Zeilen und 
Punkte pro Zeile festgelegt werden. Bilder zum Versuchsaufbau befinden sich im Anhang \ref{section:AnhangAufbau}. \\

\section{Versuchsdurchführung}

Eine Durchführung läuft folgendermaßen ab:\\
Die Probe wird aus ihrer Schachtel genommen und auf die magnetische Platte gegeben. Dann wird mit Hilfe von Millimeterschrauben die Probe unter den Cantilever gefahren. 
Dort wird der Cantilever manuell abgesenkt bis man den Schatten des Cantilevers auf dem Präparat in der Kamera sieht. Danach lässt man den Computer den Cantilever 
automatisch auf die richtige Distanz fahren. Dies geschieht durch eine PI(D)-Regler Schaltung, welche schon voreingestellt war. \\
Anschließend drückt man, nachdem man die entsprechenden Größen eingegeben hat, 'Start' und die Messung läuft von alleine ab. Die Einstellung zur jeweiligen Messung sind unten 
vermerkt.\\
Danach speichert man die Messung als nid-File. Den Angleich der X-Y-Ebene muss man hier nicht manuell durchführen. Diesen übernimmt das Programm automatisch.


\subsection*{Eichgitter}
\begin{center}
    \centering
    \begin{tabular}{l|r}
        Modus & Contact Mode \\
        Image size & 62,65 $\mu$m \\
        Time / Line & 2s \\
        Points/Line & 512\\
        Setpoint & 15 nN \\
        P-Gain & 10000 \\
        I-Gain & 1000 \\
        D-Gain & 0 \\
        
    \end{tabular}
\end{center}

\subsection*{CD-Presswerkzeug}
\subsubsection*{50$\mu$m}
\begin{center}
    \centering
    \begin{tabular}{l|r}
        Modus & Contact Mode\\
        Image size & 50,00 $\mu$m \\
        Time / Line & 2s \\
        Points/Line & 512\\
        Setpoint & 15 nN \\
        P-Gain & 10000 \\
        I-Gain & 1000 \\
        D-Gain & 0 \\
        
    \end{tabular}
\end{center}

\subsubsection*{20$\mu$m}

\begin{center}
    \centering
    \begin{tabular}{l|r}
        Modus & Contact Mode\\
        Image size & 19,34 $\mu$m \\
        Time / Line & 2s \\
        Points/Line & 512\\
        Setpoint & 15 nN \\
        P-Gain & 10000 \\
        I-Gain & 1000 \\
        D-Gain & 0 \\
        
    \end{tabular}
\end{center}

\subsection*{Nanotubes}
\subsubsection*{15$\mu$m}

\begin{center}
    \centering
    \begin{tabular}{l|r}
        Modus & Contact Mode\\
        Image size & 15,00 $\mu$m \\
        Time / Line & 1,3s \\
        Points/Line & 512\\
        Setpoint & 3,02 nN \\
        P-Gain & 10000 \\
        I-Gain & 1000 \\
        D-Gain & 0 \\
        
    \end{tabular}
\end{center}

\subsubsection*{2$\mu$m}

\begin{center}
    \centering
    \begin{tabular}{l|r}
        Modus & Contact Mode\\
        Image size & 2,168 $\mu$m \\
        Time / Line & 1,3s \\
        Points/Line & 512\\
        Setpoint & 3,02 nN \\
        P-Gain & 10000 \\
        I-Gain & 1000 \\
        D-Gain & 0 \\
        
    \end{tabular}
\end{center}

\subsection*{Goldcluster}

\subsubsection*{2,5 $\mu$m}
\begin{center}
    \centering
    \begin{tabular}{l|r}
        Modus & Non-Contact Mode\\
        Image size & 2,5 $\mu$m \\
        Time / Line & 1s \\
        Points/Line & 512\\
        Setpoint & 60\% \\
        P-Gain & 10000 \\
        I-Gain & 1000 \\
        D-Gain & 0 \\
        
    \end{tabular}
\end{center}

\subsubsection*{1,5 $\mu$m}
\begin{center}
    \centering
    \begin{tabular}{l|r}
        Modus & Non-Contact Mode\\
        Image size & 1,5 $\mu$m \\
        Time / Line & 1s \\
        Points/Line & 512\\
        Setpoint & 70\% \\
        P-Gain & 10000 \\
        I-Gain & 1000 \\
        D-Gain & 0 \\
        
    \end{tabular}
\end{center}

\subsubsection*{0,375 $\mu$m}
\begin{center}
    \centering
    \begin{tabular}{l|r}
        Modus & Non-Contact Mode\\
        Image size & 0,375 $\mu$m \\
        Time / Line & 1s \\
        Points/Line & 512\\
        Setpoint & 70\% \\
        P-Gain & 10000 \\
        I-Gain & 1000 \\
        D-Gain & 0 \\
        
    \end{tabular}
\end{center}

\subsection*{Oberflächengitter}

\begin{center}
    \centering
    \begin{tabular}{l|r}
        Modus & Non-Contact Mode\\
        Image size & 20 $\mu$m \\
        Time / Line & 1,75s \\
        Points/Line & 512\\
        Setpoint & 60\% \\
        P-Gain & 10000 \\
        I-Gain & 1000 \\
        D-Gain & 0 \\
        
    \end{tabular}
\end{center}

\subsection*{PSPMMA}

\begin{center}
    \centering
    \begin{tabular}{l|r}
        Modus & Non-Contact Mode, Phase Contrast\\
        Image size & 2 $\mu$m \\
        Time / Line & 1s \\
        Points/Line & 512\\
        Setpoint & 70\% \\
        P-Gain & 10000 \\
        I-Gain & 1000 \\
        D-Gain & 0 \\
        
    \end{tabular}
\end{center}




\section{Geräte und Fehler}

Rasterkraftmikroskop: Inventarnummer: 88459\\
Steuerelektronik: Inventarnummer: 88080\\
Cantilever (Contact Mode): NANOSENSORS type PPP-CONT nachdeR-C, S/N 78932F10L995, vom 31.7.14, Set 5\\
Cantilever (Non-Contact): NANOSENSORS, Type PPP-NCLR-10, S/N 66017F4L734
Eichgitter: Nummer: BT00250, x-y-Periodizität: 10,0 $\mu$m, z-Höhe: 119nm, Batch: 2003-03-29.2
Diverse Proben: Nanosurf Extended Sample Kapitel\\
Oberflächengitter: Gitter(1.9.14)



% Einbindung des Protokolls als pdf (mit Seitenzahl etc.)
% Erste Seite mit Überschrift
%\includepdf[pages = 1, landscape = false, nup = 1x1, scale = \skalierung , pagecommand={\thispagestyle{empty}\chapter{Protokoll}}]
%            {03-Protokoll/Protokoll.pdf}
% Restliche Seiten richtig skaliert
%\includepdf[pages = -, landscape = false, nup = 1x1, scale = \skalierung , pagecommand={}]
%            {03-Protokoll/Protokoll.pdf}