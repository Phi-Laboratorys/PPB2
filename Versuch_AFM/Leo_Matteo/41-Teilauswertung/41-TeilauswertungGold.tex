\section{Gold}

Bei diesem Versuchsteil soll die Rauheit von Materialien genauer betrachtet werden. Diese gibt an, wie stark die Oberfläche von einer idealen, glatten Oberfläche abweicht. 
Dabei hätte eine ideal glatte Fläche eine Rauheit von 0nm. Je größer diese ist, desto rauer ist die Oberfläche.\\

Hier wird die Rauheit von Gold bestimmt. Dabei wird von einer Goldprobe die Rauheit in drei unterschiedlichen 
Aufnahmegrößen bestimmt. Diese werden dann verglichen und diskutiert.

Die entstehenden Bilder von Gold kann man im Anhang betrachten unter \ref{Gold1}, \ref{Gold2} und \ref{Gold3}.

Dabei erhält man für die Rauheit die folgenden Werte:\\
\begin{center}
    \centering
    \begin{tabular}{lr}
        \toprule
        Seitenlänge des Aufnahmebereichs ($\mu$m) & Rauheit (nm)\\
        \midrule
        2,5 & 1,335\\
        1,5 & 1,468 \\
        0,375 & 1,537\\
    \end{tabular}
\end{center}

Das Ergebnis ist zunächst kontraintuitiv. Man hätte erwartet, dass die Rauheit der Probe sinkt, 
wenn man näher heranzoomt. Aber gerade der gegenteilige Effekt ist zu beobachten. Dies ist damit zu erklären, 
dass bei größeren Ausschnitten das Abtasten nicht präzise genug ist, weil 
die Spitze zu schnell über die Probe lauft. Damit bekommt man kein vollständiges Bild der Probe, sondern nur einen groben Überblick. 
Um diese Hypothese zu überprüfen könnte man noch eine Referenzmessung machen. Bei dieser nimmt man dann den großen Ausschnitt und 
erhöht die Abtastzeit für eine Zeile massiv. Dann sollte man eine höhere Rauheit als die 
drei gemessenen Werte messen.