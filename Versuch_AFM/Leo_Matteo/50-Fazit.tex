%Matteo Kumar - Leonard Schatt
% Fortgeschrittenes Physikalisches Praktikum

% 5. Kapitel Einleitung

\chapter{Fazit}
\label{chap:fazit}


Der Versuch zeigt eindeutig, dass man mit Rasterkraftmikroskopen Objekte im Nanometerbereich auflösen kann. Durch diese Fähigkeit Oberflächen auf so kleinen Skalen zu betrachten eröffnen sich 
gigantische Möglichkeiten zur Charakterisierung von Materialoberflächen, welche einem auf den ersten Blick verborgen geblieben wären. Dabei kann man zum BeispielUnterschiede der Gleitreibungskoeffizienten erklären, 
die auf einer makroskopischen Ebene unerklärlich sind. Dies und vieles mehr lässt sich mit dem Rasterkraftmikroskopen bewerkstelligen. Dabei sind wir auch an praktischen Grenzen beim 
Einsatz von Rasterkraftmikroskopen gestoßen, beispielweise dass es nicht möglich ist anhand des Phasenkontrastmoduses die Dichte von Materialien quantitativ zu bestimmen. Nichts 
desto trotz ist ein AFM ein großartiges Werkzeug um Einblicke auf der Nanometerskala zu erhalten. 
