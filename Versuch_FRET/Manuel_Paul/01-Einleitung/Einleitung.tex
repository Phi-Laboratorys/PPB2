% 1. Einleitung

\chapter{Einleitung}
\label{chap:einleitung}

In biologischen Proben stellt es als Schwierigigkeit heraus, die Abstände von Atomen oder Molekülen zu bestimmen, da die gängigen Mikroskope eine zu geringe Auflösung besitzen. Eine wichtige Messmethode von Abständen in biologischen Proben ist der Förster Resonanzenergietransfer oder auch abgekürzt FRET.\\

In diesem Versuch wird die Theorie von FRET anhand eines cyan-farbenes fluoreszentes Protein (CFP) und eines gelb-farbenes fluoreszentes Protein (YFP) in der Praxis angewandt. Dabei wird in diesem Versuch die sogenannte FRET-Effizienz bestimmt, welche selbst vom Abstand zwischen den Proben abhängt. Dies wird in drei Arten durchgeführt, durch Sensitized Emission, Photobleaching und Lebenszeitmessung, was im Folgendem diskutiert wird.