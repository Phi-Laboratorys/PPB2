\documentclass[a4paper, twoside]{article}
\usepackage{geometry} \geometry{top=30mm, bottom=25mm, inner=20mm, outer=20mm}
\usepackage[linktoc=all]{hyperref}
\hypersetup{
	bookmarksopen=true,
	bookmarksdepth=3,
	colorlinks=true,
	citecolor=blue,
	linkcolor=[rgb]{0.6,0,0}}
\usepackage{graphicx}
\setlength{\abovecaptionskip}{0pt}
\usepackage{amssymb,amsmath,physics}
\begin{document}

\title{FRET: Versuchsdruchführung}
\author{Anna-Maria Pleyer}
\date{\today}
\maketitle
\section{Bildaufnahmen am Konfokalmikroskop}
 \begin{itemize}
     \item Konfokalmikroskop testen
     \item Wovon hängt die minimale Aufnahmedauer einer konfokalen Aufnahme ab?
     \item Wie ist das Detektorsignal von den Aufnahmeparametern abhängig?
     \item Parametereinstellungen (ohne Bleichen!)
 \end{itemize}

\section{Aufnahmen der Sensitized Emission}
FRET Effizenz über fluoreszenzmikroskopische Aufnahmen bestimmen.\\ 
Mithilfe der Argon-Laserlinien und Detektion über die Photomultiplier.\\
FRET Intensität ist die Akzeptorfluoreszenz bei Donor-Anregung.\\
Probe mit doppelten Zellen (CFP und YFP).
\begin{itemize}
    \item FRET Bilder aufnehmen $S_{CY}$
    \item Aufnahmen nur von CFP $D_{CY}$
    \item Aufnahmen nur von YFP $A_{CY}$
    \item $D_{CY}$ mit Anregung und Detektion des Donors
    \item $A_{CY}$ mit Anregung und Detektion des Akzeptors
    \item $S_{CY}$ mit Anregung des Donors und Detektion des Akzeptors
    \item mindest 10 Zellen bei gleicher Anregungsintensität
\end{itemize}  
Bilder für die Korrekturfaktoren\\
CFP: 
\begin{itemize}
    \item NUR Zellen mit reiner CFP Makierung
    \item $D_{CFP}$ mit Anregung und Detektion des Donors
    \item $A_{CFP}$ mit Anregung und Detektion des Akzeptors
    \item $S_{CFP}$ mit Anregung des Donors und Detektion des Akzeptors
    \item mindest 10 Zellen, selbe aufnahme Parameter (wie FRET)
\end{itemize}
Wiederholung mit YFP:
\begin{itemize}
    \item NUR Zellen mit reiner YFP Makierung
    \item $D_{YFP}$ mit Anregung und Detektion des Donors
    \item $A_{YFP}$ mit Anregung und Detektion des Akzeptors
    \item $S_{YFP}$ mit Anregung des Donors und Detektion des Akzeptors
    \item mindest 10 Zellen, selbe aufnahme Parameter (wie FRET)
\end{itemize}
Kann man aus E den Abstand der Fluorophore berechnen? Zusätzliche Messungen?


\section{Donoremission nach Akzeptorbleichen}
\begin{itemize}
    \item Parameter von bleichen bestimmen (!NUR Akzeptor bleichen; Nicht Donor!)
    \item FRET Effizenz bestimmen VOR dem Bleichen
    \item Akzeptormolekül Bleichen  (min. 50\% der Ursprünglichen Intensität)
    \item FRET Effizenz bestimmen NACH Bleichen
    \item mindest. 10 Zellen ausreichend Bilder pro Zelle
    \item Vergleichen mit Sensitized Emission
    \item Mögliche Nachteile/Störgrößen dieser Methode?
\end{itemize}
\section{Lebenszeitmessung}
\begin{itemize}
    \item Lebenszeiten von CFP an Zellen, die nur eine CFP Makierung aufweisen 
    \item Lebenszeiten von YFP an Zellen, die nur eine YFP Makierung aufweisen 
    \item Passender Filter Würfel verwenden
    \item Lebenzeit der Donor Moleküle (YFP \textbf{und}CFP Zellen) VOR Bleichung des Akzeptormolekül
    \item Bleichen des Akzeptormolekül
    \item Lebenzeit der Donor Moleküle (YFP \textbf{und}CFP Zellen) NACH Bleichung des Akzeptormolekül
    \item Vergleich der (vor und nach Bleichung) Lebenszeiten (Nur CFP, YFP und CFP)
    \item Abschätzung der FRET Effizenz
    \item Vergleich FRET Effizenz mit Sensitized Emission und Donoremission unter AKzeptorbelichen
    \item Physikalisch sinnvolle Ergebnisse der Lebenszeiten
    \item Eventuelle Abweichungen?
    \item Vor- und Nachteile dieser Methode?
\end{itemize}



\end{document}