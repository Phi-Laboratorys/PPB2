%Matteo Kumar - Leonard Schatt
% Fortgeschrittenes Physikalisches Praktikum

% 1. Kapitel Einleitung

\chapter{Einleitung}
\label{chap:einleitung}

In diesem Versuch behandelt eine Art des Energietransfers zwischen einem Akzeptor und Donor, welcher strahlungslos geschieht, den Förster-Resonanzenergietransfer, kurz FRET. 
Um diesen Effekt an Zellen zu untersuchen verwenden wir ein Konfokalmikroskop. Dieses Mikroskop unterschiedet sich von klassischen 
Mikroskopen dadurch, dass es nicht die gesamte Probe auf ein Mal beleuchtet. Mit Hilfe dieses Mikroskopes kann man die Membran der Zellen sichtbar machen. 
Der Versuch ist insofern interessant, weil er Einblick in eine vielseitig einsetzbare Methode gibt. In großen Bereichen der Biophysik sowie beim Untersuchen 
organischer Stoffe kann dies sehr hilfreich sein. Des Weiteren ist FRET ein Prozess, der in vielen anderen Themenfeldern hochinteressant ist. In der Halbleitertechnik 
spielt er im Bereich der organischen Leuchtdioden eine entscheidende Rolle. Außerdem ist der bei der Analyse von Protein-Protein-Interaktionen, der Analyse von Protein-Konformationsänderungen 
und Polymerasekettenreaktion sehr wichtig. Daher ist es hilfreich diesen Effekt einmal näher zu betrachten.
