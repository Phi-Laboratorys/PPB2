\chapter{Einleitung}
Laser ist eine Abkürzung und steht für \textit{light amplification by stimulated emission of radiation}.
Zu Deutsch bedeutet dies in etwa: 'Licht Verstärkung durch stimulierte Emission von Strahlung'.
Das besondere an Laserstrahlen, im Vergleich zu z.B. Glühlampen ist, dass sie häufig eine hohe Intensität und 
meist monochromatisches Licht aussenden.\\
Aus diesen und noch weiteren Gründen sind sie aus der modernen Technik nicht mehr wegzudenken.
Dort haben sie sehr viele Anwendungsbereiche: In der Medizin, Industrie aber auch im Alltag sind sie 
vielseitig einsetzbar. Einige Beispiele hierfür wären Laserdrucker oder freiverkäufliche Laserpointer.\\
Laser können allerdings auch als Messinstrument genutzt werden, sie finden beispielsweise Anwendung 
in der Spektroskopie. Dies ist auch der Anwendungsbereich, welcher während dem Versuch genauer betrachtet wird. 
Im Vordergrund des Versuches steht der Aufbau und die Justierung des Lasers, in unserem Fall ein Helium-Neon Laser. 
Zunächst soll mit einfachen Messmethoden dessen Verstärkung, sowie die Modenstruktur untersucht werden. 
Des Weiteren wird noch das Fabry-Perot-Interferometer verwendet, dieses wird ebenfalls eingehend betrachtet werden
und dient als Laserstrahlanalyse. Zuerst sollen im Folgenden die theoretischen Grundlagen etwas genauer beleuchtet werden.

