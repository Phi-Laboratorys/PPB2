\section{Messung von Mischfrequenzen mittels einer Photodiode}
Im Resonator kann es vorkommen, dass die verstärkten Frequenzen sich leicht unterscheiden und es zu einer Überlagerung kommt.
Somit kommt es zu einer Mischfrequenz, einer Schwebung. Die axialen Moden überlagern sich 
bei gleicher Frequenz aber unterschiedlicher Wellenlänge. 
Die Summe der Einzelfrequenzen ist nicht mehr detektierbar. Es kann nur die Schwebungsfrequenz bestimmt werden, da diese eine geringer Frequenz aufweist. Die Schwebungsfrequenz entspricht der Differenzfrequenz, also dem Modenabstand (vgl. \eqref{a}). Das heißt, man kann mithilfe der Schwebungsfrequenz Rückschlüsse auf den axialen Modenabstand ziehen.
Um diese Frequenzen zu messen benötigt man allerdings eine sehr schnelle Photodiode.
