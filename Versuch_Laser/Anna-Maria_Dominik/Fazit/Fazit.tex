
\chapter{Fazit}
In diesem Versuch haben wir uns eingehend mit dem physikalischen System Laser befasst.
Das besondere an diesen Versuch war, dass wir hier zum ersten Mal einen Laser justieren mussten, bevor wir mit ihm Messung durchführen konnten.
Des Weiteren haben wir den Umgang mit optischen Bauelementen, dem Fabry-Perot-Interferometer und einer CCD-Kamera erlernt und trainiert. Vor allem das Justieren der optischen Bauelemente sowie das Einstellen der Spiegel war von großem Nutzen, diese Fähigkeit werden wir noch häufiger im Laufe des Praktikums benötigen.
Das außergewöhnlichste und spannendste an diesem Versuch war der Teil mit der Holografie, diese wurde hier erstmalig besprochen und untersucht.\\
Betreffend der Auswertung gab es jedoch wenig überraschendes. Die meisten Messwerte lagen im Bereich des erwarteten, vor allem wenn man die Fehler mitberücksichtigt. Die Abweichungen kamen hauptsächlich durch die entsprechende Messmethode zustande.
