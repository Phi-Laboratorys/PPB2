% Leonhard Schatt

\chapter{Methodik}


\section{Justierung}
Zunächst wurde der Laser aufgebaut und justiert. Dazu wurde als Erstes ein Hilfslaser parallel zur Tischoberfläche in ca. 21,5\,cm Höhe mithilfe einer Lochblende ausgerichtet. Dieser soll bei der Ausrichtung der Spiegel helfen. 
Danach wurde die Plasmaröhre mit dem laseraktiven Material eingesetzt und so ausgerichtet, dass das passierende Licht auf einem Blatt Papier in einigem Abstand ein möglichst rundes Abbild darauf warf und möglichst hell war. 
Im Anschluss wurde der totalreflektierende Spiegel so eingesetzt, dass der Laserstrahl diesen mittig traf und das Licht in die Richtung reflektierte, aus der es kam. Dazu wurde eine Irisblende verwendet. 
Auf der anderen Seite der Röhre wurde nun der Auskoppelspiegel so eingesetzt, dass dessen Rückreflex sich mit dem Hilfslaserstrahl deckte und dass der Gesamtresonator eine Länge von ca. 50\,cm hatte (aus $g_1g_2\leq 1$). 
Der Laser sprang danach sofort an, sobald die Pumpspannug eingeschaltet wurde. 
Der Laserstrahl wurde auf ein Powermeter gelenkt, sodass die Feinjustierung durch abwechselndes Verstellen der Spiegel so durchgeführt werden konnte, dass die Leistung des Lasers maximal wurde. Unsere erreichte Leistung lag bei 2,33\,mW. 

\section{Bestimmung des Verstärkungsfaktors}
Zur Bestimmung des Verstärkungsfaktors werden gezeilt Verluste im Resonator erzeugt, in diesem Fall durch ein Glasplättchen, das in den Strahlengang im Resonator eingebracht 
wurde. Abhängig von dessen Winkel werden unterschiedliche Anteile des Strahls im Resonator reflektiert bzw. transmittiert, wodurch die emittierte Leistung des Lasers aus dem 
Resonator variiert. \\
Zunächst wurde der Drehtisch, auf dem das Plättchen befestigt war, so justiert, dass der Stahl senkrecht auf das Plättchen trifft. Dies war bei einem Winkel von $21,19^{\circ}$ der 
Fall. Anschließend wurde der Winkel von $60^{\circ}$-$90^{\circ}$ in $0,01^{\circ}$-Schritten variiert und die Leistung des Lasers mittels einer Photodiode gemessen und aufgenommen.

\section{Axiale Moden}
Zur Messung der axialen Moden wurde der Laser möglichst parallel in ein durchstimmbares konfokales Fabry-Pérot-Interferometer geführt. Die vom Interferometer durchgelassene Intensität wurde dann von einer 
Photodiode detektiert, verstärkt um einen Faktor 100 und dann graphisch dargestellt. Die Rampenspannung des Interferometers wurde dann so eingestellt, 
dass man zweimal den freien Spektralbereich sehen konnte. Dies war daran zu erkennen, dass man genau zweimal das gleiche Bild nebeneinander auf dem Oszilloskop sah. \\
Um das Verstärkungsprofil zu bestimmen, haben wir das Oszilloskop auf einen Modus gestellt, bei dem es alle Spuren 
überlagert. Dies haben wir getan und haben den Tisch mit leichten Erschütterungen zum Vibrieren gebracht. Das entstandene Bild wurde aufgenommen. \\
Daraufhin wurde ein Etalon ($FSR$ 10\,GHz) in den Strahlengang eingebracht und so eingestellt, dass annähernd ein single-mode Betrieb gewährleistet war. 
Mithilfe einer schnellen Photodiode und eines Spektrumanalysators wurde ein Screenshot der Lage des Peaks gemacht. Anschließend wurde ein Glasplättchen in den Strahlengang eingebracht und der (leicht) verschobene 
Peak erneut aufgenommen.

\section{Gaußstrahl}
Um den Strahlausbreitungsparameter eines Gaußstrahls zu berechnen, wurden nacheinander der Experimentierlaser und der Hilfslaser auf eine Linse (feste Position) gelenkt. Nach 
Durchquerung dieser wurde mittels einer CCD-Kamera ein Profil der einfallenden Intensität aufgenommen. Dies geschah anhand eines vertikalen und eines horizontalen Schnitts 
durch das Strahlprofil mithilfe der Software 'Laserscan', wodurch sich in Näherung zwei Gaußkurven ergaben. Zudem wurden auch 2D-Plots der Intensität aufgenommen. 
Um den Chip in der Kamera nicht zu übersteuern wurden Graufilter in verschiedenen Stärken verwendet.

\section{Transversalmoden}
Zur Darstellung der Transversalmoden wurde der Strahl durch eine Linse aufgeweitet und auf einen in ca. 50\,cm entfernten weißen Schirm umgelenkt. Durch Einbringung einer Drahtblende in verschiedenen Positionen relativ zum Strahl und Veränderung der 
Spiegelstellungen konnten verschiedene Transversalmoden auf dem Schirm sichtbar gemacht und photographiert werden. Am Ende dieses Versuchsteils erlosch der Laser, da ein Spiegel, 
den wir verstellten, nicht korrekt festgeschraubt war, und dessen Position nicht wiederhergestellt werden konnte.

\section{Hologramm}
Da der Experimentierlaser erloschen war, strahlten wir mit dem Hilfslaser auf das Hologramm, welches dann betrachtet werden konnte. Dabei war die Qualität des entstandenen Bildes eher mäßig.
