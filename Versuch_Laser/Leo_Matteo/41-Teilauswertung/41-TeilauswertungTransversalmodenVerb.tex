\section{Transversalmoden eines Lasers}
\label{section:transvM}

Ein Laser hat mehrere Moden. Normalerweise ist die $TEM_{00}$-Mode dominant, das heißt man sieht einen zusammenhängenden 
Punkt mit näherungsweise gaußförmigem Profil. Es gibt jedoch auch noch andere Moden, wie in dem Grundlagenkapitel \ref{subs:moden}
beschrieben. Diese haben wir versucht durch die eingebrachte Drahtblende zu erzeugen. Dabei haben wir die in Abbildung \ref{bild:Moden} gezeigten Moden beobachtet.
\begin{figure}[ht]
    \centering
    \subfloat[$TEM_{00}$]{\label{TEM00}%
      \includegraphics[width=0.235\textwidth]
      {Bilder/Auswertung/TEM00.png}}\quad
    \subfloat[$TEM_{10}$]{\label{TEM10}
      \includegraphics[width=0.25\textwidth]
      {Bilder/Auswertung/TEM10.png}}\quad
    \subfloat[$TEM_{01}$]{\label{TEM01}%
      \includegraphics[width=0.25\textwidth]
      {Bilder/Auswertung/TEM01.png}}\quad
    \subfloat[$TEM_{20}$]{\label{TEM20}%
      \includegraphics[width=0.25\textwidth]
      {Bilder/Auswertung/TEM20.png}}\quad
      \subfloat[$TEM_{30}$]{\label{TEM30}%
      \includegraphics[width=0.25\textwidth]
      {Bilder/Auswertung/TEM30.png}}\quad
      \subfloat[$TEM_{un.}$]{\label{TEMunz}%
      \includegraphics[width=0.25\textwidth]
      {Bilder/Auswertung/TEMunsugeordnet.png}}
      \subfloat[$TEM_{01*}$]{\label{TEM01*}\quad
      \includegraphics[width=0.245\textwidth]
      {Bilder/Auswertung/TEM11.png}}
      \caption{Transversalmoden der HeNe-Laser}
      \label{bild:Moden}
  \end{figure}
  Normalerweise sollten die Moden radialsymmetrisch sein. Diese kann man hier nur vereinzelt beobachten, da die Brewsterfenster die Radialsymmetrie aufheben.
  Spannenderweise kann in Abbildung \ref{TEM01*} auch eine radialsymmetrischen Mode beobachten werden. Außerdem
  gibt es eine Mode in \ref{TEMunz}, welche wir nicht identifizieren konnten. Diese könnte eine Mischmode sein. Abgesehen von den fotografierten Moden konnten wir auch eine 
  $TEM_{11}$-Mode beobachten, diese aber leider nicht aufnehmen.
  \clearpage