%Matteo Kumar - Leonard Schatt
% Fortgeschrittenes Physikalisches Praktikum

% Teilauswertung Alpha


\section{Energiekalibrierung}
\label{subs:kali}

Der Alphastrahlendetektor ist im Wesentlichen eine Photodiode. Deren Schutzschicht wurde jedoch entfernt, da diese zu einer signifikanten 
Abschwächung der Alphastrahlen führen würde.\\
Zunächst wurde das Spektrum von Radium-226 mit einer 1\,mm-Blende in einem Abstand von 5\,mm gemessen, um eine Relation zwischen Kanalnummer 
und Energieinhalt in diesem herzustellen. Dafür wurde das sich ergebende Spektrum aus Abb. \ref{bild:kali} mit den Werten aus der 
Tabelle in Abb. \ref{bild:TabelleRa} verglichen.
\\
\\
\\

\begin{figure}[h]
    \centering
    \includegraphics[scale=0.5]{Bilder/TabelleRadium.png}
    \caption{Zerfallsdaten der Ra-226-Reihe (\cite{Kador2021}, S.14)} %\protect \footnotemark}
    \label{bild:TabelleRa}
\end{figure}
%\footnotetext{\cite{Kador2021}}

\clearpage

\begin{figure}[h]
    \centering
    \includegraphics[scale=0.75]{Bilder/kali.png}
    \caption{Plot der Kalibrationsmessung} 
    \label{bild:kali}
\end{figure}


Dabei sind in der Tabelle zwei sehr kurzlebige Alphastrahler zu finden: Po-214 und Po-218. Da Po-214 die höchste Strahlenenergie hat, 
sollte der rechte einzelne Peak diesem Strahler zuzuordnen sein. Po-218 ist der rechte Peak des Dreifachpeaks zuzuordnen. Dessen 
linker Peak (Ra-226) ist für eine Kalibrierung zu schlecht aufgelöst und der mittlere aufgrund der Überlagerung zweier Strahler 
(Rn-222, Po-210) ebenfalls nicht geeignet. \footnotemark
\footnotetext{\url{https://www.ld-didactic.de/software/524221en/Content/Appendix/Ra226.htm}, Stand: 14.09.21}

In den Daten wird nun der Kanal mit den maximalen Einträgen für jeden Peak gesucht. Dafür wurden an die beiden Peaks eine Gaußfunktion der Form 
\begin{equation*}
    g(x) = A \cdot \exp(\frac{-(x - x_0)^2}{2\sigma^2})
\end{equation*}
gefittet. Dies wurde auch für den schlechter aufgelösten Peak versucht, doch der Fit lieferte kein sinnvolles Ergebnis. Für die beiden anderen Peaks ergaben sich die 
Fitparameter und Fehler, die in Tab.\ref{tab:parameterKali} zu finden sind.

\begin{center}
    \centering
    \begin{tabular}{l|rrrrrr}
        Peak & $A$ &  $x_0$ & $\sigma$ & $s_A$ & $s_{x_0}$ & $s_{\sigma}$\\
        \hline
        rechts & 156,77 & 192,31 & 5,82 & 2,85 & 0,122 & 0,127\\
        rechts(Dreifachpeak) & 141,57 & 126,74 & 6,82 & 3,06 & 0,245 & 0,280\\
    \end{tabular}
    \captionof{table}{Fitparameter für die beiden auszuwertenden Peaks im Alphaspektrum von ra-226 zur Energiekalibrierung und dazugehörige Fehler.}
    \label{tab:parameterKali}
\end{center}

Daraus ergeben sich die Lagen der Peaks. Für Po-214 lag dieser bei Kanal 192,31 $\pm$ 0,12  und 
für Po-218 bei Kanal 126,47 $\pm$ 0,25. Mit den Energie aus der Tabelle für Po-214 ($7,69$ MeV) und Po-218 ($6,00$ MeV) erhält man folgendes 
Gleichungssystem:

\begin{align*}
    7,69 \, MeV &= 192,31 \cdot m + E_0 \\
    6,00 \, MeV &= 126,47 \cdot m + E_0
\end{align*}

Dieses hat die Lösung

\begin{equation*}
    m = (0,02567 \pm 0,14) \, MeV, \qquad E_0 = (2,753 \pm 0,25) \, MeV,
\end{equation*}

wobei die Fehler über die Werte für $m$ und $E_0$ für die flachst- und steilstmögliche Gerade innerhalb des Fehlers der Maxima der Peaks ermittelt wurden. 
Damit ergibt sich die Energiekalibrierung zu

\begin{equation}
    E = Kanalnummer \cdot 0,02567 \, MeV + 2,753 \, MeV
\end{equation}

Eine Problematik der Energiekalibrierung ist die starke Anhängigkeit der gemessenen Energien von Art und Dicke des Absorbers, wie in 
\ref{subs:abs} näher beleuchtet. Ideal wäre demnach eine Messung unter Auschluss jedweden Absorbers, also im Vakuum.\\





\section{Geiger-Nuttall-Regel}

Die Geiger-Nuttall-Regel ist eine empirische Abschätzung zur Halbwertszeit. Sie folgt dem Gesetz
\begin{equation}
    \lg \biggl (\frac{T_{1/2}}{1s} \biggl ) = a \cdot \frac{Z}{\sqrt{E_{\alpha}}} + b,
    \label{eq:gnr}
\end{equation}


mit Konstanten a und b. Abb.\ref{bild:Nuttall} veranschaulicht den Zusammenhang aus Gl.\ref{eq:gnr} graphisch.\\

\begin{figure}[h]
    \centering
    \includegraphics[scale=0.75]{Bilder/Nuttall.jpg}
    \caption{Graphische Darstellung der Geiger-Nuttall-Regel für verschiedene Atomsorten. Dabei ist in der logarithmischen Auftragung sehr gut eine Gerade für jede Atomsorte zu sehen.}
    \label{bild:Nuttall}
\end{figure}


Die Wertepaare für Po-214 und Po-218 aus Kapitel \ref{subs:kali} in Gl.\ref{eq:gnr} eingesetzt liefert a und b:

\begin{equation}
    a = 1747,12 \sqrt{1eV} \qquad b = -56,71
\end{equation}

Nun kann zur Überprüfung ein Element der Zerfallsreihe von U-238 eingesetzt werden. Dies sei Rn-222; bei einer Halbwertszeit von 
3,82\,d sollte sich nach \ref{bild:TabelleRa} eine Energie von 5,49\,MeV ergeben. Es ist: \\

\begin{equation*}
    \ref{eq:gnr}  \leftrightarrow E_{\alpha} = \biggl(\,\frac{1747,12602759 \, Z}{\lg(T_{1/2})+56,70765593}\, \biggr)^2 \\
    \to E_{\alpha, Rn-222} \approx 5,56 \, MeV
\end{equation*}

Dies entspricht zwar nicht ganz dem gefordertem Wert, allerdings liegt die Abweichung nur bei ca. 1\%. Deshalb kann die 
Geiger-Nuttall-Regel durchaus als bestätigt angesehen werden.





\section{Blendenverkleinerung}

Nun wurde eine Messung identisch zu der Kalibrationsmessung durchgeführt, mit dem einzigen Unterschied, dass anstatt der 1\,mm-Blende 
jetzt die 3\,mm-Blende verwendet wurde. Trägt man die detektierten Teilchen in den Kanäle gegen die selbigen auf, so ergibt sich der 
Graph in Abb. \ref{bild:blende}. \\

\begin{figure}[h]
    \centering
    \includegraphics[scale=0.75]{Bilder/blende.png}
    \caption{Detektierte Teilchen aufgetragen gegen die Kanalnummer, 3mm-Blende}
    \label{bild:blende}
\end{figure}

Dabei fällt zunächst einmal auf, dass die Zahl der detektierten Teilchen trotz identischer Messzeit deutlich angestiegen ist. Dies ist 
allerdings zu erwarten, da eine größere Blende offensichtlich eine größere Detektorfläche freigibt. Skaliert man nun die Messung 
mit der kleineren Blende entsprechend hoch und legt diese über die Messung mit der größeren Blende, so ergibt sich Abb. 
\ref{bild:blendebeide}

\begin{figure}[h]
    \captionsetup{justification=centering,margin=2cm}
    \centering
    \includegraphics[scale=0.75]{Bilder/blendebeide.png}
    \caption{Vergleich der Plots für beide Blendenweiten; Werte für 1mm-Blende mit Faktor 3,35 zur Vergleichbarkeit skaliert}
    \label{bild:blendebeide}
\end{figure}

Dabei ist erkennbar, dass die Spektren beinahe identisch sind. Der Graph der größeren Blende wirkt jedoch glatter. Dies könnte 
daran liegen, dass bei kleineren Zählraten sich deren Fehler von $\sqrt{n}$ (poissonverteilt) deutlich stärker auswirkt und die Schwankungen der 
Messwerte dadurch größer ist. Zudem ist eine kleine Verbreiterung in Richtung niedrigerer Energien erkennbar. Dies könnte darin begründet 
liegen, dass nun auch die in einem größeren Raumwinkel emittierten Alphateilchen duch die größere Blende auf den Detektor treffen können. 
Diese haben allerdings eine (etwas) längere Wegstrecke in Luft zurückgelegt und so durch Stöße mehr Energie abgegeben.





\section{Absorption von $\alpha$-Strahlen}
\label{subs:abs}

Zur Bestimmung der Absorption von Alphastrahlen wurde das Spektrum der Radiumquelle zunächst in verschiedenen Abständen zum Detektor 
aufgenommen (1\,cm, 1,5\,cm, 2\,cm, 2,5\,cm, 3\,cm; Ablesefehler jeweils auf 0,2\,cm abgeschätzt), danach in konstantem Abstand ($1 \pm 0,2$)\,cm, aber mit Durchgang durch 
unterschiedliche Anzahl an Schichten Mylarfolie (eine bis neun Lagen). 
Dabei betrug der Durchmesser der Blende stets 3\,mm. Die Messzeit bei der Abstandsmessung wurde i.d.R. immer größer, die der 
Folienmessung betrug stets ca. 300\,s.\\

Die Graphen mit allen Abständen bzw allen und ausgewählten Lagen finden sich in den Abb. \ref{bild:abstandAlle} bis \ref{bild:lagenAusgewaehlt}. \\

\begin{figure}[h]
    \centering
    \includegraphics[scale=0.75]{Bilder/abstandAlle.png}
    \caption{Alphaspektren für verschiedene Abstände d}
    \label{bild:abstandAlle}
\end{figure}

\begin{figure}[h]
    \centering
    \includegraphics[scale=0.75]{Bilder/lagenAlle.png}
    \caption{Alphaspektren für alle Lagenzahlen an Mylarfolie l}
    \label{bild:lagenAlle}
\end{figure}

\begin{figure}[h]
    \centering
    \includegraphics[scale=0.75]{Bilder/lagenAusgewaehlt.png}
    \caption{Alphaspektren für ausgewählte Lagenzahlen an Mylarfolie l}
    \label{bild:lagenAusgewaehlt}
\end{figure}

\clearpage

Betrachtet man die entstandenen Spektren beider Messreihen, so fallen einige Gemeinsamkeiten auf:\\
Die Anzahl der Einträge in den Kanälen, respektive die Höhe der Peaks, fallen mit zunehmenden Abstand bzw Schichtdicke, und das trotz 
teilweise längerer Messdauer. Dies ist allerdings nicht verwunderlich: Da Alphastrahen eine sehr geringe Reichweite haben, genügen schon 
kleinere Variationen in Abstand oder Dicke des Absorbermaterials, um die Transmission signifikant herabzusetzen.\\
Zudem verschiebt sich die Lage der Peaks zunehmend zu kleineren Energien hin. Das scheint auf den ersten Blick verwunderlich, sollten 
die Peaks doch den Energien der Elemente der Zerfallsreihe des Strahlers entsprechen und damit eine feste Energie haben. Jedoch 
wechselwirken die Teilchen der Alphastrahlung durch Stöße mit dem Absorbermaterial (Luft bzw. Mylar) und geben so Energie ab. Dieser 
Verlust ist natürlich abhängig von der Wegstrecke durch den Absorber, was zu geringeren detektierten Energien für größere Abstände 
bzw Schichtdicken führt. Diese Erklärung wird durch die Messung der Gammaspektren gestützt: Gammastrahlung ist keine Teilchenstrahlung 
sondern eine elektromagnetische Welle. Deshalb ist hier keine Verschiebung der Peaks bei Abstands- oder Absorberdickenvariation zu 
erwarten; genau dies wurde im ersten Versuchsteil bestätigt.\\
Aus der Verschiebung der Peaks ergibt sich allerdings eine praktische Konsequenz: Die Messung der Energien von Alphastrahen ist stark 
abhängig von Art und Dicke des Absorbers. Dies muss unbedigt bei der Kalibration des Messgeräts berücksichtigt werden. Da allerdings 
in der verwendeten Tabelle keine näheren Angaben gemacht wurden, gehen wir an dieser Stelle von Werten ohne Absorber aus. Die Messung 
zur Kalibration wurde mit minimalem Abstand und kleinster Blende durchgeführt, weshalb auch ein Minimum an Absorber zwischen Probe 
und Messgerät war. Deshalb ist davon auszugehen, dass die Kalibration im Wesentlichen richtig ist.\\

Die Bethe-Bloch-Formel beschreibt den Zusammenhang zwischen Energieverlust der Strahlung und durchquerter Weglänge durch einen Absorber. 
Im vereinfachten Fall für schwere Teilchen in der nicht-relativistischen Näherung lautet diese: \\

\begin{equation}
    - \frac{dE}{dx} = \frac{k}{E}, 
\end{equation}

wobei k ein materialabhängiger Parameter ist, der im folgenden für Luft und Mylar bestimmt werden soll (\cite{Jaekel1997}, S.64).\\

Trägt man nun die Quadrate der Energien (des rechten Peaks im Spektrum) gegen den Abstand (ergo Absorberdicke von Luft) bzw 
gegen die Schichtdicke der Mylarfolien (Foliendicke: 4,3\,$\mu$m) auf, so ergibt sich jeweils näherugsweise eine Gerade (Abb. \ref{bild:luftger} 
und \ref{bild:mylarger}). Dabei war vor allem der Abstand aufgrund des Aufbaus nicht leicht zu messen und soll mit einer Unsicherheit 
von $\pm$2\,mm berücksichtigt werden.\\

\begin{figure}[h]
    \captionsetup{justification=centering,margin=2cm}
    \centering
    \includegraphics[scale=0.75]{Bilder/luftger.png}
    \caption{Quadrate der Energien gegen Abstand des Detektors zur Probe aufgetragen; dazu berechnete Regressionsgerade}
    \label{bild:luftger}
\end{figure}

\begin{figure}[h]
    \captionsetup{justification=centering,margin=2cm}
    \centering
    \includegraphics[scale=0.75]{Bilder/mylarger.png}
    \caption{Quadrate der Energien gegen Sichichtdicke des Mylars aufgetragen; dazu berechnete Regressionsgerade}
    \label{bild:mylarger}
\end{figure}

\clearpage

Dabei wurden bei der Mylarfolie die Messung mit den neun Lagen nicht berücksichtigt, da die Datenlage mit einer maximalen Kanalbelegung 
von vier zu gering ist. Ohnehin ist erkennbar, dass für größere Schichtzahlen die Abweichungen von der Gerade zunehmend größer werden. 
Mittels linearer Regression werden die Steigungen $m_{Luft}$ und $m_{Mylar}$ berechnet. Die Abschwächungskoeffizienten aus der 
Bethe-Bloch-Formel ergeben sich nach: \\

\begin{align}
    k &= 0,5 \cdot m \nonumber \\
    \to k_{Luft} &= 1,31362546 \cdot 10^{-23} \, \frac{J^2}{m}, \qquad k_{Mylar} = 1,48780223 \cdot 10^{-20} \, \frac{J^2}{m} \\
\end{align}
(\cite{Jaekel1997}, S.71f) \\

Außerdem gilt: \\

\begin{equation}
    k = K \, \frac{Z \rho}{A}
\end{equation}
(\cite{Kador2021}, S. 15) \\

Damit ist: \\

\begin{equation}
    \frac{k_{Luft}}{k_{Mylar}} = \frac{Z_{Luft} \rho_{Luft} A_{Mylar}}{Z_{Mylar} \rho_{Mylar} A_{Luft}} = 
    \frac{7,22 \cdot 1,29 \cdot 10^{-3} \cdot 4,67}{2,67 \cdot 0,92 \cdot 14,44} = 0,001227
\end{equation}
 
Das Verhältnis der gemessenen Abschwächungskoeffizienten ergibt sich zu:\\

\begin{equation}
    \frac{k_{Luft}}{k_{Mylar}} = \frac{1,31362546 \cdot 10^{-23}}{1,48780223 \cdot 10^{-20}} = 0,0008829
\end{equation}

Messung und Theorie stimmen zwar in der Größenordnung überein, haben allerdings einen Fehler von ca. 30\%. Dabei gehen wir davon aus, 
dass der Hauptbeitrag zu diesem von der enormen Unsicherheit in der Abstandsmessung ausgeht. Durch den geringen Abstand der Messpunkte 
überlappen sich die Fehlerbalken beinahe. Dadurch ergibt sich ein großer Spielraum für die Lage und damit auch die Steigung der 
Regressionsgerade. In dieser Hinsicht sind unseres Erachtens der Fehler von 30\% nicht wünschenswert, aber unter diesen Voraussetzungen 
akzeptabel. Der Fehler der Mylarfolie und die Nicht-Berücksichtigung des Absorbers Luft bei der Mylarmessung 
sollten um Größenordnungen kleiner liegen und deshalb kaum eine Rolle spielen.






\section{Vermessung eines radioaktiven Zeigers}

Zuletzt wurde noch das Alphaspektrum eines alten Uhrzeigers mit radioaktiver Farbe aufgenommen. Dabei wurde aufgrund der sehr schwachen 
Strahlung die größtmögliche Blende (5mm) und ein kleinstmöglicher Abstand gewählt. Zudem betrug die Messdauer ca. einen Tag. Das 
Spektrum findet sich in Abb. \ref{bild:uhr} wieder. \\

\begin{figure}[h]
    \centering
    \includegraphics[scale=0.75]{Bilder/uhr.png}
    \caption{Alphaspektrum eines alten Uhrzeigers. Die Peaks sind mit den jeweils zugeordneten Strahlern beschriftet.}
    \label{bild:uhr}
\end{figure}

In dem Spektrum sind die Peaks sehr scharf zu sehen. Auch in den Messtabellen ist durch die große Differenz zu den nächsten Kanälen die Lage der Peaks so eindeutig, 
dass die zugehörigen Energien einfach ausgelesen werden können: \\

\begin{align*}
    E_0 &= 6,06 \, MeV \\
    E_1 &= 6,55 \, MeV \\
    E_2 &= 6,67 \, MeV \\
    E_3 &= 7,17 \, MeV \\
    E_4 &= 8,66 \, MeV \\
\end{align*}

%(\cite{Mende2016}, S. 376f)
In dem Wissen, dass der Zeiger Radium enthält, können die Energien mit denen der Alphastrahler der entsprechenden Zerfallsreihe verglichen werden. In Abb.\ref{bild:TabelleRa}
sind genau fünf solche Strahler gelistet, deren Energien jedoch nicht mit denen aus Abb.\ref{bild:uhr} übereinstimmen. Ordnet man diese allerdings nach ihren relativen Positionen 
zueinander den Peaks im Spektrum zu, entspricht der Doppelpeak bei ca. 6,5\,MeV den Isotopen Po-210 und Rn-222, die auch im Vergleich zu den übrigen Strahlern energetisch nah 
beieinander liegen. Das lässt die Vermutung zu, dass eine Zuordnung anhand der Zerfallsreihe möglich ist, die Energiekalibrierung in Abschnitt \ref{subs:kali} aber dann 
wohl doch nicht genau genug war. Die Peaks in Abb.\ref{bild:uhr} wurden mit den auf diese Weise zugeordneten Nukliden beschriftet.\\

Im Graphen ist außerdem auffällig, dass die Linien langsam ansteigen, aber nach dem Peak prompt abfallen. Dies könnte damit erklärt werden, dass 
die Peaks die maximale Energie der Alphateilchen abbilden. Diese können aber auch mit Luftteilchen zwischen Probe und Detektor wechselwirken, wobei 
sie einen Teil ihrer Energie abgeben. Im Spektrum ist das als Ausläufern zu niedrigeren Energien bzw. asymmetrischen Linien zu sehen. \\
