% Matteo Kumar
% PPB2 Alpha-Gamma
%Grundlagen

\chapter{Grundlagen}

Eine Vielzahl der bekannten Elemente sind instabil, sie zerfallen spontan in kleinere Tochterkerne. Dies wird als natürliche 
Radioaktivität bezeichnet. Dabei werden im Allgemeinen drei Arten von Strahlung unterschieden: Alpha-, Beta- und Gammastrahlung. 
Im Gegensatz zur Betastrahlung, die ein kontinuierliches Spektrum aufweist, haben Gamma- und Alphastrahlung ein diskretes Linienspektrum, 
das charakteristisch für das jeweilige Element ist. Deshalb eignen sich diese Strahlungsprozesse gut für Analyseprozesse. \\

Gammastrahlen sind elektromagnetische Strahlen, die emittiert werden, wenn ein Atom aus einem angeregten Zustand in den Grundzustand 
relaxiert. Sie zerfallen also nach\
\begin{equation*}
    ^Y_ZX^* \to \, ^Y_ZX + \gamma.
\end{equation*}

Sie haben eine theoretisch unendliche Reichweite und können nur schlecht abgeschirmt werden (dazu später in der Auswertung). Dafür ist 
ihr Ionisationsvermögen eher niedrig. \\

Alphastrahlen sind emittierte Heliumkerne. Die zugehörige Zerfallsgleichung lautet

\begin{equation*}
    ^Y_ZX \to \, ^{Y-4}_{Z-2}V + ^4_2\alpha.
\end{equation*}

Dadurch haben sie zwar eine sehr geringe Reichweite und können leicht abgeschirmt werden; durch ihre zweifach positive Ladung haben sie 
jedoch ein sehr hohes Ionisationsvermögen. Deshalb können Alphastrahler, sobald sie in einen Organismus gelangen, enorme Schäden anrichten. \\

Zur Detektion werden in diesem Versuch Halbleiterdetektoren verwenden. Diese sind im Wesentlichen pin-Übergänge, die in Sperrrichtung 
betrieben werden. Deshalb ist während der Dauer des Versuches im Falle des Gammastrahlendetektors eine Kühlung der Detektoren notwendig, damit dieser nicht überhitzt. 
Dies ist beim Aplhastrahlendetektor nicht nötig, da dieser eine einfache Photodiode ist. 
Gelangt nun Strahlung in die Sperrzone, werden Elektron-Loch-Paare mit Energien proportional zur einfallenden Strahlung erzeugt. Dieser 
Stromfluss kann abgegriffen und ausgewertet werden. Dazu werden die Signale aus dem Detektor verstärkt und in einen Vierkanalanalysator 
geleitet. Dieser analysiert die Pulshöhen und sortiert diese in eine vorgegebene Anzahl an Kanäle, wodurch ein Histogramm entsteht.\\

