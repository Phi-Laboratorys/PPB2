\section{Herleitung Interferogramm einer quasimonochromatischen Welle}
\textbf{Machen Sie sich die Herleitung der Beziehung für das Interferogramm einer quasimonochromatischen Lichtquelle klar! Mit Hilfe des Ergebnisses können Sie die Linienform der Spektrallinien bestimmen.}\\

Quasimonochromatisches Licht hat im Gegensatz zu chromatischen Licht keine singuläre Wellenlänge.
Es handelt sich Licht mit einer spektralen Intensitätsverteilung $G(k)$.
Die vorkommenden Frequenzen in quasimonochromatischen Licht, sind eng verteilt, z.B. Gauß-förmig.\\%sind an ihrer Intensität gemessen Glockenförmig (meist Gaußförmig) verteilt.\\

Um nun das Interferogramm dieser quasimonochromatischen Welle zu berechnen, muss man erst einmal die Gesamtintensität der Interferenzfunktionen berechnen:
\begin{equation}
    I(l)=\sum_{i=1}^n\frac{I_{0_i}}{2}\left(1+\cos(\frac{2\pi\cdot l}{\lambda_i})\right)
\end{equation}
Wenn wir nun den Fall $n\to\infty$ betrachten und die Intensität als Intensitätsverteilung (allgemeiner Fall) vorliegt, folgt:
\begin{equation}
    I(l)=\frac{I_0}{2}\int_0^\infty G(k)\left(1+\cos(k\cdot l)\right)dk
\end{equation}
Hierbei ist die Funktion $G(k)$ normiert: $\int_0^\infty G(k)=1$\\
Nun kann das Interferogramm wie in der Versuchsanleitung beschrieben hergeleitete werden:
\begin{equation}
    I(l)-\frac{I_0}{2}=\int_{-\infty}^\infty G(k)\cos(k\cdot l)dk
\end{equation}
Somit entspricht das Interferogramm der Cosinus-fouriertransormation der Intensitätsverteilung.
Diese kann man durch eine Rücktransformation bekommen und somit ist dann die Linienform der Spektrallinien bekannt:
\begin{equation}
    G(k)=\int_{-\infty}^\infty\left(I(l)-\frac{I_0}{2}\right)\cos(k\cdot l)dl
\end{equation}\newpage
\section{Natürliche Linienbreite, Doppler-/ Druckverbreiterung und ihre Unterscheidung in Fourierspektrum}
\textbf{Was versteht man unter natürlicher Linienbreite, Doppler sowie Druckverbreiterung? Wie lassen sich im Fourierspektrum die beiden Verbreiterungsmechanismen unterscheiden?}\\
\subsection{Natürliche Linienbreite}
Die \textbf{natürliche Linienbreite} basiert auf der Energie-Zeit-Unschärferelation.
Die Bestimmung der Lebensdauer und der exakten Energie eines 
angeregten Zustandes ist nicht möglich. 
Dies führt zu einer statischen Verbreiterung der Linienbreite.\\
Unter Verbreiterungsmechanismen versteht man die Vergrößerung der Linienbreite
über die natürliche Linienbreite hinaus. Hierbei unterscheidet man zwischen zwei Arten:\\
Die \textbf{homogene Verbreiterung} tritt auf, wenn die Emissionswahrscheinlichkeit für eine bestimmte Frequenz 
für alle Teilchen gleich groß ist. Hierzu zählen z.B. Druckverbreiterung.\\
Die \textbf{inhomogen Verbreitung} tritt auf, wenn die Emissionswahrscheinlichkeit für eine bestimmte Frequenz 
nicht für alle Teilchen gleich groß ist. Hierzu zählen z.B. Doppelverbreiterung.
\subsection{Dopplerverbreiterung}
Die Doppelverbreiterung kommt daher zustande, wenn die Atome oder Moleküle (welche das Licht abstrahlen) eine Geschwindigkeit ungleich null besitzen.
Durch diese Bewegung ändert sich die Frequenz und somit auch die Wellenlänge des abgestrahlten Lichts, der sogenannte Dopplereffekt.\\
Für die Verschiebung der Wellenzahl folgt:
\begin{equation}
    k=k_0\left(1+\frac{v}{c}\right) \Rightarrow v=\frac{k-k_0}{k_0}c
\end{equation}
Die Bewegung der Moleküle sind temperaturabhängig und die Geschwindigkeiten sind Bolzmann-verteilt:
\begin{equation}
    P(v)dv=\sqrt{\frac{\beta}{\pi}}\exp\left(-\beta v^2\right)dv
\end{equation} 
Mit:
\begin{equation}
    \beta=\frac{m}{2k_BT}
\end{equation}
Für die Intensitätsverteilung folgt:
\begin{equation}
    G(k)\sim\sqrt{\frac{\beta}{\pi}}\frac{k_0}{c}\exp\left(-\beta\left(c\frac{k-k_0}{k_0}\right)^2\right)
\end{equation}
Hier ist die Intensität Gauß-verteilt.\newpage
\subsection{Druckverbreiterung}
Bei der Druckverbreiterung wird die Linie aufgrund von Wechselwirkungen zwischen den Atomen/Molekülen verbreitert.
Durch elastische Stöße, kommt es zu einer kurzzeitigen Verschiebung der Energieniveaus weswegen die Spektrallinie verbreitert wird.
Hierzu wird die kinetische Gastheorie verwendet:
\begin{equation}
    P(t) dt=\frac{\exp(-t/\bar{t})}{\bar{t}}dt
\end{equation}
Hier bezeichnet $\bar{t}$ die mittlere Zeit zwischen zwei Stößen.
Somit folgt für die Intensitätsverteilung:
\begin{equation}
    G(k)\sim\frac{2(tc)^{-1}}{(tc)^{-2}+\left(k-k_0\right)^2}
\end{equation}
Hier ist die Intensitätsverteilung Lorentz-verteilt.
\section{Kohärenzlänge für Gauß- und Lorentzverteilung}
\textbf{Leiten Sie die Kohärenzlänge für ein Gaußförmiges und für ein Lorentzförmiges Spektralprofil aus den jeweiligen Linienbreiten her.}
\subsection{Gaußform}
Hier wird zuerst eine Koordinatentransformation gemacht, um den Ursprung der Intensitätsverteilung auf $k_0$ zu legen:
\begin{align}
    k'&=k-k_0\\
    H(k')&=G(k-k_0)
\end{align}
Somit folgt für die Doppelverbreiterte Intensitätsverteilung:
\begin{equation}
    H(k')=\sqrt{\frac{\beta}{\pi}}\frac{k_0}{c}\exp\left(-\beta\left(c\frac{k'}{k_0}\right)^2\right)
\end{equation}
Nun wird die Breite $a$ eingeführt. An dieser ist die Intensitätsverteilung auf $1/e$ der Ursprungsverteilung angefallen:
\begin{align}
    H(k')&=\frac{H(0)}{e}\\
    &\Rightarrow a^2=4\frac{k_0^2}{\beta c^2}\\
    H(k')&=\sqrt{\frac{\beta}{\pi}}\frac{k_0}{c}\exp\left(-\frac{4}{a^2}k'^2\right)
\end{align}
Wird nun $H(k')$ normiert, so folgt:
\begin{equation}
    H(k')=\frac{2}{a\sqrt{\pi}}\exp\left(-\frac{4}{a^2}k'^2\right)
\end{equation}
Nun folgt für das Interferogramm:
\begin{align}
    I(l)&=C(l)\cos\left(k_0l\right)\\
    C(l)&=I_0\int_{-\infty}^\infty H(k')\cos(k'l)dk'\\
    &=\Re\left\{I_0\int_{-\infty}^\infty H(k')\exp(ik'l)dk'\right\}\\
    &=\frac{2}{a\sqrt{\pi}}\Re\left\{\int_{-\infty}^\infty \exp\left(-\left(\frac{2k'}{a}-\frac{ila}{4}\right)^2-\left(\frac{al}{4}\right)^2\right)\right\}
\end{align}
Mit der Substitution $z=\left(\frac{2k'}{a}-\frac{ila}{4}\right)$ folgt:
\begin{align}
    C(l)&=\frac{2}{a\sqrt{\pi}}I_0\exp\left(-\left(\frac{al}{4}\right)^2\right)\cdot\frac{a}{2}\underbrace{\int_{-\infty}^\infty \exp\left(-z^2\right)}_{=\sqrt{\pi}}\\
    &=I_0\exp\left(-\frac{a^2}{16}l^2\right)\\
    \Rightarrow I(l)&=I_0\exp\left(-\frac{a^2}{16}l^2\right)\cos\left(k_0l\right)
\end{align}
Die Kohärenzlänge $L$ ist die halbe Länge zwischen den Punkten, an denen $I(l)$ auf $I_0/e$ abfällt.
Da der Cosinus symmetrisch um $k_0$ ist folgt:
\begin{equation}
    L_{dopp.}:=\frac{4}{a}\propto\frac{1}{\sqrt{T}}
\end{equation}
\subsection{Lorentzform}
Bei der Lorentzform wird zuerst die mittlere Zeit zwischen zwei Stößen wie folgt umgeschrieben.
Die mittlere freie Weglänge ist dabei $\bar{s}$:
\begin{equation}
    \bar{t}=\frac{\bar{s}}{\bar{v}}=\frac{1}{\sigma p}\sqrt{\frac{\pi m k_B T}{8}}
\end{equation}
Das Interferogramm erhält man durch eine fouriertransormation der Intensitätsverteilung:
\begin{equation}
    I(l)=I_0\exp\left(\frac{-\left|l\right|}{\bar{t}c}\right)\cos(k_0l)
\end{equation}
Die Kohärenzlänge $L$ ist wiederum die halbe Länge zwischen den Abfall auf $I(l)=I_0/e$:
\begin{equation}
    L_{druck.}=\bar{t}c\propto\frac{\bar{T}}{p}
\end{equation}\newpage
\section{Zusammenhang Kohärenzlänge und FWHM-Breite}
\textbf{Wie hängt die halbe 1/e-Breite mit der vollen Halbwertsbreite FWHM einer Gauß- bzw. Lorentzförmigen Linie zusammen?}\\

Die volle Halbwertsbreite ist definiert als volle Breite einer Kurve, zwischen zwei Punkten mit halber maximal Intensität.
Für die Lorentzverteilung (Druckverbreiterung) folgt:
\begin{equation}
    G(k)=\frac{2(tc)^{-1}}{(tc)^{-2}+\left(k-k_0\right)^2}
\end{equation}
Die FWHM-Breite einer Lorentzverteilung lässt sich wie folgt berechnen \citep[vgl.]{FWHM-Lorentz}:
\begin{align}
    f(x)=\frac{\Gamma/2}{\left(x-\mu\right)^2+\left(\Gamma/2\right)^2}
\end{align}
Wobei $\Gamma$ die FWHM-Breite ist.
Angewendet auf unsere Lorentzverteilung folgt:
\begin{align}
    FWHM_{druck.}&=\frac{2}{tc}=\frac{2}{L}\\
    L&=\frac{2}{FWHM_{druck.}}
\end{align}

Für eine Gaußverteilung (Doppelverbreiterung) folgt:
\begin{equation}
    G(k)=\sqrt{\frac{\beta}{\pi}}\frac{k_0}{c}\exp\left(-\beta\left(c\frac{k-k_0}{k_0}\right)^2\right)
\end{equation}
Die FWHM-Breite einer Gaußverteilung lässt sich wie folgt bestimmen:
\begin{align}
    f(x)=\exp\left(-\frac{\left(x-\mu\right)^2}{2\sigma^2}\right)
\end{align}
Bei dieser ist die FWHM-Breite $\sqrt{8\ln(2)}\sigma$ \citep[vgl.]{FWHM-Gauss}.
Angewendet auf unsere Gaußverteilung folgt:
\begin{align}
    FWHM_{dopp.}&=2\sqrt{2\ln(2)}\frac{k_0}{\sqrt{2\beta}c}=\sqrt{\ln(2)}\underbrace{2\frac{k_0}{\sqrt{\beta}c}}_{=a}=\sqrt{\ln(2)}a=\frac{4\sqrt{\ln(2)}}{L}\\
    L&=\frac{4\sqrt{\ln(2)}}{FWHM_{dopp.}}
\end{align}
\section{Interferogramm zweier benachbarter Spektrallinien}
\textbf{Berechnen Sie das Interferogramm zweier monochromatischer, dicht benachbarter Spektrallinien mit den Intensitäten $I_1$ und $I_2$ ab}\\

Man betrachte zwei ebene, polarisierte Wellen:
\begin{align}
    E_1&=A_1\cdot\exp\left(i\phi_1\left(\vec{r},t\right)\right)\\
    E_2&=A_2\cdot\exp\left(i\phi_2\left(\vec{r},t\right)\right)
\end{align}
Wenn diese Wellen nun miteinander interferieren, dann werden beide superpositioniert und für die Gesamtintensität folgt:
\begin{align}
    I&=\left|E_1+E_2\right|^2=\left(E_1+E_2\right)\left(E_1^*+E_2^*\right)\\
    &=A_1^2+A_2^2+A_1A_2\cos(\delta\phi)=I_1+I_2+2\sqrt{I_1I_2}\cos(\delta\phi)
\end{align}
Wobei $\delta\phi$ für die Phasendifferenz steht.
Bei Interferrenz folgt $\delta\phi=\left|\omega_2-\omega_1\right|t$.
Somit folgt für die zeitliche Intensitätsverteilung an einem Ort:
\begin{equation}
    I(t)=I_1+I_2+2\sqrt{I_1I_2}\cos(\left|\omega_2-\omega_1\right|t)
\end{equation}
\section{Abhängigkeit des Auflösungsvermögens}
\textbf{Zeigen Sie, dass das Auflösungsvermögen des Spektrometers vom Verfahrweg des Spiegels abhängig ist!}\\

Das durch den endlichen Spiegelweg gemessenes Interferogramm $I_{obs}$ ist eine Multiplikation eines unendlich ausgedehnten Interferogramms mit einer Blendenfunktion $B(l)$.
\begin{align}
    I_{obs}(l)=I(l)\cdot B(l)
\end{align}
Diese kann einfachheitshalber als eine Rechteckblende angenommen werden.\\
Für das gemessene Spektrum folgt:
\begin{equation}
    G_{obs}(k)=G(k)*b(k)
\end{equation}
Hierbei ist $b(k)$ die Fouriertransormierte von $B(l)$.
Für die Rechteckblende folgt, wobei $M$ der maximale Gangunterschied ist:
\begin{equation}
    b(k)=2M\frac{\sin\left(kM\right)}{kM}
\end{equation}
Das Auflösungsvermögen ist charakterisiert durch die Halbwertsbreite von $b(k)$:
\begin{equation}
    \Delta k = \frac{2,4\pi}{M}\propto\frac{1}{M}
\end{equation}
Somit ist das Auflösungsvermögen umgekehrt proportional zu $M$ (Gangunterschied).
Dieser hängt von dem Verfahrweg der Spiegel ab, somit ist das Auflösungsvermögens abhängig vom Verfahrweg der Spiegel.