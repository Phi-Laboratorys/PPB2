% Manuel Lippert - Paul Schwanitz
% Physikalisches Praktikum

% Teilaufgabe 1

\section{Allgemeines zum Thema Chaos}
\label{sec:allgemeines}

\subsection{Dynamische Systeme}
\label{sub:dynamSys}
Die Formulierung eines \textit{Dynamischen Systems} in der Physik geschieht anhand von gewöhnlichen Differentialgleichungen mit dem Vektor $\vect{x}(t)=(x_1(t)$,..., $x_n(t))\in\mathbb{R}^n$
\begin{gather}
    \dot{\vect{x}}(t)=\vect{F}(\vect{x}(t)),
    \label{eq:dynamDGL}
\end{gather}
dabei beschreibt $\vect{x}(t)$ den \textit{\textbf{Zustand}} des Systems zum Zeitpunkt $t\in\mathbb{R}$.\\
Das dynamische System ist vollständig determiniert, wenn ein Zustand $\vect{x}(t)$ angegeben ist. Aus diesem Zustand lassen sich alle vorangegangen und folgenden Zustände des Systems bestimmen, womit das System nur von der Wahl des Anfangszustands abhängt. Dynamische Systeme können auch zeitdiskret angegeben werden, worauf aber hier nicht weiter eingegangen wird \citep{Lueck}.\\

\begin{itemize}
    \item[\textbf{1.}]\textbf{Phasenfluss}\\
    In der Mathematik wird ein dynamisches System durch den \textit{\textbf{Fluss}} bzw. \textit{\textbf{Phasenfluss}}  beschrieben. Unter dem \textit{Fluss} versteht man die Abbildung $\vect{\vect{\phi}}:\mathbb{R}^n\times\mathbb{R}\rightarrow\mathbb{R}^n$, welche die \textit{\textbf{Flussaxiome}} erfüllt \citep{Mat1}:
    \begin{gather}
        \begin{aligned}
            (1)~&\vect{\vect{\phi}}(\vect{x}_0,0)=\vect{x}_0\\
            (2)~&\vect{\vect{\phi}}(\vect{\phi}(\vect{x}_0,t),s)=\vect{\phi}(\vect{x}_0,t+s).
            \label{eq:flussaxiome}
        \end{aligned}
    \end{gather}
    Der \textit{Fluss} $\vect{\phi}$ ordnet \textbf{jedem} Anfangszustand $\vect{x}_0$ einen neuen Zustand zum Zeitpunkt $t$ zu \citep{Lueck}.\\

    \item[\textbf{2.}]\textbf{Trajektorie}\\
    Der \textit{Fluss} $\vect{\phi}$ kann mit dem \textit{Zustand} $\vect{x}(t)$ in Verbindung gebracht werden mit der Beziehung: $\vect{x}(t)=\vect{\phi}_{\vect{x}_0}(t)=\vect{\phi}(\vect{x}_0,t)$ mit festem $\vect{x}_0$, wobei nach  (\ref{eq:flussaxiome}) $\vect{x}(0)=\vect{\phi}_{\vect{x}_0}(0)=\vect{\phi}(\vect{x}_0,0)=\vect{x}_0$ gilt.\\
    Hierbei beschreibt $\vect{\phi}_{\vect{x}_0}(t)$ die \textit{Lösungskurve}, welche auch \textit{Bahnkurve}, \textit{Orbit}, \textit{Phasenbahn} oder \textit{\textbf{Trajektorie}} des Flusses $\vect{\phi}$ genannt wird und eine spezielle Lösung von (\ref{eq:dynamDGL}) darstellt, welche wiederum die Bewegung des Punktes $\vect{x}$ unter Wirkung des Flusses $\vect{\phi}$ mit dem Anfangszustand $\vect{x}_0$ beschreibt. \citep{Lueck}\\
    Durch die Abhängigkeit der \textit{Trajektorien} vom Anfangszustand $\vect{x}_0$ kann gefolgert werden, dass sich \textit{Trajektorien} mit unterschiedlichen Anfangszuständen $\vect{x}_0$ nicht schneiden können. Es können aber unterschiedliche Anfangszustände $\vect{x}_0$ auf derselben \textit{Trajektorie} befinden und sich nur um eine Zeittranslation unterscheiden \citep{Mat1}.\\

    \item[\textbf{3.}]\textbf{Phasenraum}\\
    Der \textit{\textbf{Phasenraum}} oder \textit{\textbf{Zustandsraum}} beschreibt eine Menge aller Zustände oder eine Darstellung aller Trajektorien eines dynamischen Systems und bietet einen Überblick über das Verhalten der gesamten Differentialgleichung ohne diese explizit lösen zu müssen \citep{Mat1}.
    %Der \textit{Phasenraum} wird vom Zustand $\vect{x}(t)$ und dessen Ableitung $\dot{\vect{x}}(t)$ aufgespannt bzw. ist eine $(\vect{x}(t),\dot{\vect{x}}(t))$-Ebene, was eine Parameterdarstellung der Differentialgleichung über die Zeit $t$ darstellt.
    %In diesem \textit{Phasenraum} lässt sich dann ein Vektor $(\vect{x},\dot{\vect{x}})$ definieren, welcher auf die \textit{Trajektorien}, die sich mit dem Anfangszustand $\vect{x}_0$ änderen, zeigt. Die Ableitung dieses Vektors $\frac{\text{d}}{\text{d}t}(\vect{x},\dot{\vect{x}})=(\dot{\vect{x}},\ddot{\vect{x}})$ erzeugt ein \textit{Vektorfeld} bzw. ein \textit{Richtungsfeld} der Differentialgleichung, dessen Vektoren tangential auf den \textit{Trajektorien} steht.

    \item[\textbf{4.}]\textbf{Attraktor}\\
    In (\ref{eq:dynamDGL}) wird ein Vektorfeld $\vect{F}$ im Phasenraum definiert, welches als Geschwindigkeitfeld des Phasenflusses $\vect{\phi}$ angesehen werden kann.\\ Durch Betrachtung der Divergenz des Vektorfelds $\nabla\vect{F}$ kann eine Aussage getroffen werden über die Rate mit dem sich ein Volumenelement $V$ unter der Wirkung des Flusses verändert. Zwei Fälle sind hier besonders hervorzuheben:
    \begin{gather}
        \begin{aligned}
            (1)~&\nabla\cdot\vect{F}=0\Rightarrow \dot{V}=0 \Rightarrow~\text{Konservatives System}\\
            (2)~&\nabla\cdot\vect{F}<0\Rightarrow \dot{V}<0 \Rightarrow~\text{Dissipatives System}
        \end{aligned}
    \end{gather}
    In einem dissipativen System laufen die Trajektorien nach einer Einlaufsphase (transiente Bewegung) in einem begrenzten Bereich im Phasenraum, welchen man als \textit{\textbf{Attraktor}} bezeichnet (Bewegung auf \textit{Attraktor}: permante oder posttransiente Bewegung). Ein Attraktor weist folgende Eigenschaften auf \citep{Lueck}:
    \begin{itemize}
        \item[(1)] Kompakte Menge im Phasenraum
        \item[(2)] Invariant unter der Wirkung des Flusses
        \item[(3)] Volumen des Attraktors ist Null
        \item[(4)] Eine beliebige Obermenge des Attraktors schrumpft unter der Wirkung des Flusses auf den Attraktor selbst zusammen   
    \end{itemize}
    \textbf{Arten von Attraktor}
    \begin{center}
        \begin{tabular}{ccc}
            \includegraphics[width=4cm]{FixpunktAttraktor.png}
            & \includegraphics[width=4cm]{GrenzzyklusAttraktor.png}
            & \includegraphics[width=4cm]{TorusAttraktor.png}
        \end{tabular}
        \captionof{figure}{Fixpunkt, Grenzzyklus, Tours- bzw. Ringattraktor \citep{Lueck}}
        \includegraphics[width=14cm]{SeltsameAttraktoren.png}
        \captionof{figure}{Seltsame Attraktoren: Rössler- und Lorentzattraktor, Eigens erstellt mit dem Python Packages matplotlib mit Hilfe von \citep{py1, py2}}
    \end{center}
\end{itemize}

\subsection{Deterministisches Chaos}
\label{sub:determChaos}
Systeme, die das Verhalten des \textit{deterministischen Chaos} zeigen, weisen ein zufällig erscheinendes Verhalten auf, was jedoch nicht durch äußere Umstände verursacht wird, sondern das Verhalten folgt aus den Eigenschaften des Systems selbst.\\
Das Verhalten von einem System mit deterministischem Chaos lässt sich langfristig nicht vorhersagen, da ähnliche Uhrsachen langfristig nicht zu ähnlichen Wirkungen führen. Das Prinzip der starken Kausalität wird somit verletz \citep{Lueck}.

\subsection{Fouriertransformation und Leistungsspektrum}
\label{sub:fouriertrafo}
Eine Fouriertransformation ist eine Integraltransformation, mit der aperiodische Signale in ein kontinuierliches Spektrum zerlegt werden können.\\
Fouriertransformation einer Messgröße \(x(t)\):
\begin{gather}
    \hat{x}(\omega) = \lim_{T \to \infty} \int_{0}^{T}  x(t) e^{-i\omega t}\,dt 
\end{gather}
Das Leistungsspektrum \( P(\omega)\) kann man nun folgendermaßen aus der Fouriertransformierten des Messwertes berechnen:
\begin{gather}
    P(\omega) = |\hat{x}(\omega)|^2
\end{gather}
Das Leistungsspektrum stellt die Leistungsanteile für unterschiedliche Frequenzen im Zeitsignal dar.\\
Bei chaotischen Systemen beispielsweiße erhält man ein Leistungsspektrum mit kontinuierlichem Verlauf, das für große Frequenzen abnimmt \citep{Lueck}. Da jedoch weißes stochastisches Rauschen ebenfalls ein kontinuierlichen Verlauf liefert kann dieser nicht als hinreichender Beweis für Chaos angesehen werden.

\subsection{Darstellungsweisen eines chaotischen Attraktor}
\label{sub:darstellungAttraktor}
\begin{itemize}
    \item[\textbf{1.}]{\textbf{Phasenraumdarstellung}}\\
    Die \textit{\textbf{Phasenraumdarstellung}} wie in (\ref{sec:allgemeines}) erwähnt, gibt einen Überblick über den Verlauf der Bewegung des dynamischen Systems. Dabei trägt man je an eine Raumachse eine Phasenraumvariabel auf (z.B. $x, \dot{x}$), wobei der Phasenraum dabei $n$-Dimensionen haben kann und nicht alle Phasenraumvariablen bekannt sein müssen, da ein Attraktor im Phasenraum rekonstruiert werden kann. Dazu werden die Messwerte einer Phasenraumvariablen bei einer festen Zeitspanne $\tau$, also $\varphi(t), \varphi(t+\tau)$,..., $\varphi(t+(n-1)\tau)$, als neu Koordinaten $x_i$ (z.B. $x_1=\varphi(t)~\text{und}~x_2=\varphi(t+\tau))$ eines neuen Koordinatensystems. Bei richtiger Wahl von $\tau$ und $n$ lässt sich dann der tatsächliche Attraktor rekonstruieren. Diese Methode der Rekonstruktion eines Atrraktors wurde vom Mathematik \textit{Floris Takens} bewiesen und ist auch als \textit{Einbettungstheorem} bekannt \citep{Lueck}.
    \item[\textbf{2.}]{\textbf{Poincar\'e-Abbildung}}\\
    Die \textit{\textbf{Poincar\'e-Abbildung}} ist eine Projektion des \textit{\textbf{Poincar\'e-Schnitts}} an einer Ebene. Diese Abbildung ist immer die Dimension $n-1$ und ist somit eine Dimension niedriger als der Phasenraum mit der Dimension $n$ und es gehen keinen Informationen bzgl. dem Langzeitverhalten des Systems verloren. Aufgrund dieser Tatsache lässt sich die \textit{Poincar\'e-Abbildung} zur Analyse von höherdimensionalen Phasenräumen verwenden.\\
    Der \textit{Poincar\'e-Schnitt} ist dabei eine Menge aller Durchstoßpunkte der Trajektorien im Phasenraum auf einer Hyperfläche. Hierbei müssen die Trajektorien die Hyperfläche \textit{transversal} (senkrecht) und in einer vorgegebenen Richtung schneiden. Praktisch werden meistens die Lage von Extremwerten einer Messgröße als Bedingung für die Schnittebene (Hyperfläche) und trägt diese gegen die anderen Phasenraumvariablen zu diesem Zeitpunkt gegeneinander auf oder man wählt eine Ebene, die den Attraktor geeignet schneide \citep{Lueck}.
    \item[\textbf{3.}]{\textbf{Wiederkehr-Abbildung}}\\
    Bei der \textit{\textbf{Wiederkehr-Abbildung}} wird eine diskrete Abbildung aktueller Messwerte über die vorangegangenen Messwerte aufgetragen. Dabei werden Punkte in einer $x(n)$-$x(n+1)$-Ebene aufgetragen mit den Messwerten $x_n$ als Koordinaten, also ($x_0,x_1$), ($x_1,x_2$),...,($x_n,x_{n+1}$). Diese Abbildung ähnelt dann einem \textit{Poincar\'e-Schnitt}, weswegen man bei einem kontinuierlichen System (\ref{eq:dynamDGL}) dessen \textit{Poincar\'e-Abbildung} für dieses Verfahren verwendet \citep{Lueck}.
    % TODO: #37 Recherche + Quelle zu Wiederkehrabbildung @ManeLippert
    \item[\textbf{4.}]{\textbf{Bifurkationsdiagramm}}\\
    Bei einem \textit{\textbf{Bifurkationsdiagramm}} betrachtet man die Projektion der \textit{Poincar\'e-Abbildung} auf \textbf{eine} Achse unter Veränderung eines Kontrollparameters, Parameter welche man aktiv im Experiment verändern kann, wobei man die durch die Projektion gewonnenen Werte gegen den jeweiligen Parameterwert aufträgt \citep{Lueck}.
    \newpage
    \item[\textbf{5.}]{\textbf{Phasendiagramm}}\\
    Wenn bei dem \textit{Bifurkationsdiagramm} mehrere unterschiedliche voneinander unabhängige Parameter existieren, verwendet man das \textit{\textbf{Phasendiagramm}}. Dazu trägt man die als Koordinatenachsen die jeweiligen Parameter, die das Systemverhalten beeinflussen, gegeneinander auf und erhält Landkarte des globalen Systemverhaltens in Abhängigkeit der gewählten Parameter \citep{Lueck}.  
\end{itemize}