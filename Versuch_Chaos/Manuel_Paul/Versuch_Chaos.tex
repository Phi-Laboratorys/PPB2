% Manuel Lippert - Paul Schwanitz
% Physikalisches Praktikum

% Main-Datei für die Auswertung in TeX

% Struktur:
% Für jeden Abschnitt gibt es einen Ordner, damit jeder individuell an seinen Aufgaben arbeiten
% kann, ohne beim merge in GitHub Konflikte zu erhalten. Deshalb werden alle Unteraufgaben auch 
% extra in Ordner angelegt. Die einzelnen Dateien über den input Befehl einfügbar.
% Bilder und andere Grafik werden im Ordner Grafik abgelegt 


% Packages
\documentclass[paper=a4,bibliography=totoc,BCOR=10mm,twoside,numbers=noenddot,fontsize=11pt]{scrreprt}
\usepackage[ngerman]{babel}
\usepackage[T1]{fontenc}
\usepackage[latin1, utf8]{inputenc}
\usepackage[babel,german=quotes]{csquotes} %For Quotes
\usepackage{lmodern}
\usepackage{graphicx}
\usepackage{nicefrac}
\usepackage{fancyvrb}
\usepackage{amsmath,amssymb,amstext}
\usepackage{siunitx}
\usepackage{url}
\usepackage{natbib}
\usepackage{microtype}
\usepackage[format=plain]{caption}
\usepackage{physics}
\usepackage{titleref}

% Zusätzliche Packages
\usepackage{geometry}
\usepackage{anyfontsize}
\usepackage[table]{xcolor}
\usepackage{ifthen}
\usepackage[absolute,overlay]{textpos}
\usepackage{amsfonts}
\usepackage{xstring}
\usepackage{tikz}
\usepackage{pdfpages}
\usepackage{hyperref}
\usepackage{circuitikz}
\usepackage{subcaption}

% Abschnittseinrückung und -abstand
% Die folgenden Zeilen sollen möglichst nicht verändert werden
\parindent 0.0cm
\parskip 0.8ex plus 0.5ex minus 0.5ex

% Anzahl und Größe von Gleitobjekten
% maximal 2 Objekte oben und unten
% erlaubt auch größere Bilder, welche die ganze Seite benötigen
% Die folgenden Zeilen sollen möglichst nicht verändert werden
\setcounter{bottomnumber}{2}
\setcounter{topnumber}{2}
\renewcommand{\bottomfraction}{1.}
\renewcommand{\topfraction}{1.}
\renewcommand{\textfraction}{0.}

%\sc und \bc veraltet. Daher: (20.09.2018)
\DeclareOldFontCommand{\sc}{\normalfont\scshape}{\@nomath\sc}
\DeclareOldFontCommand{\bf}{\normalfont\scshape}{\textbf}

% Verschiedenes
\pagestyle{headings}          % Der Seitenstil sollte möglichst nicht verändert werden
\graphicspath{{./Bilder/}}    % Der Pfad für die Abbildungen Abbildungen wird gesetzt
\VerbatimFootnotes            % \verb etc. auch in \footnotes mφglich

% Funktionen
\newcommand\tab[1][1cm]{\hspace*{#1}}
\newcommand{\vect}[1]{\boldsymbol{\mathbf{#1}}}
\newcolumntype{g}{>{\columncolor[rgb]{ .741,  .843,  .933}}l}
% Tiefgestellte Zahlen nicht kursiv
\catcode`_=\active
\newcommand_[1]{\ensuremath{\sb{\mathrm{#1}}}}

\begin{document}

    \nonfrenchspacing

    % 0. Kapitel Cover
    %Matteo Kumar - Leonard Schatt
% Fortgeschrittenes Physikalisches Praktikum
% 0. Cover
% Noch abänderbar nur ein Vorschlag
\newgeometry{top=30mm, bottom=20mm, inner=20mm, outer=20mm}
\thispagestyle{empty}

% Colors
\definecolor{Notablue}{HTML}{3498DB}		%Theoretische Physik
\definecolor{Notared}{HTML}{CF366C}			%Mathematik
\definecolor{Notagreen}{HTML}{19B092}		%Experimentalphysik
\definecolor{Notaorange}{HTML}{FA9D00}		%Chemie/Wahlfach nicht physikalisch
\definecolor{Notagrey}{HTML}{969696}		%Praktikum
\definecolor{Notalavendel}{HTML}{9DBBD8}	%Wahlfächer physikalisch

% Boolean by default false
\newboolean{twoRows}
\newboolean{symbol}

% Funktions
\makeatletter
   \def\vhrulefill#1{\leavevmode\leaders\hrule\@height#1\hfill \kern\z@}
\makeatother
\newcommand*\ruleline[1]{\par\noindent\raisebox{.8ex}{\makebox[\linewidth]{\vhrulefill{\linethickness}\hspace{1ex}\raisebox{-.8ex}{#1}\hspace{1ex}\vhrulefill{\linethickness}}}}

% Variables
\def\schriftgrosse{70}
\def\linethickness{1,5pt}

\def\farbe{Notagrey}
\def\fach{PPB2}
\def\name{Matteo Kumar - Leonhard Schatt}
\def\uberschrift{Solarzelle} % Absatz mit \\[0,5cm]; u = Übung, k = Klausur; s = Skript, e = Ergebnis
\def\bottom{WS2021}
\def\datum{06.09.2021}
\def\platz{B11 | Raum 0.03   }
\def\betreuer{Paul Recknagel}
\def\groupnr{3}

\begin{titlepage}
			
	\centering
	{\LARGE \sffamily {\textbf{\bottom}\par}}
	\vspace{2,5cm}
    {\fontsize{40}{0}\sffamily\ruleline{\textcolor{\farbe}{\textbf{\fach}}}\par}
    \vspace{6cm}
	{\Large\sffamily \ruleline{\name}\par}
	
	
	% Choose Text
	\ifthenelse{\equal{\uberschrift}{s}} {\def\titel{Skript}}	
		{\ifthenelse{\equal{\uberschrift}{k}} {\def\titel{Klausur}}
			{\ifthenelse{\equal{\uberschrift}{u}} {\def\titel{Übung}}
				{\ifthenelse{\equal{\uberschrift}{e}} {\def\titel{Klausur \\[0,5cm] Ergebnis}}
					{\def\titel{\uberschrift}}
				}
			}
		}
	
	\begin{textblock*}{21cm}(0cm,9cm) % {block width} (coords), centered		
		{\fontsize{\schriftgrosse}{0}\sffamily\textcolor{\farbe}{\textbf{\titel}}\par}
	\end{textblock*}
	
	% Choose Logo
	\ifthenelse {\equal{\farbe}{Notared}} {\def\logo{Bilder/Logo/UniBTNotared}}
		{\ifthenelse {\equal{\farbe}{Notagreen}} {\def\logo{Bilder/Logo/UniBTNotagreen}}
			{\ifthenelse {\equal{\farbe}{Notablue}} {\def\logo{Bilder/Logo/UniBTNotablue}}
				{\ifthenelse {\equal{\farbe}{Notaorange}} {\def\logo{Bilder/Logo/UniBTNotaorange}}
					{\ifthenelse {\equal{\farbe}{Notagrey}} {\def\logo{Bilder/Logo/UniBTNotagrey}}
						{\ifthenelse {\equal{\farbe}{Notalavendel}} {\def\logo{Bilder/Logo/UniBTNotalavendel}}	
							{\ifthenelse {\equal{\farbe}{black}} {\def\logo{Bilder/Logo/UniBT}}	
								{\def\logo{noLogo}}
							}
						}
					}
				}
			}
		}	

	\IfSubStr{\logo}{noLogo}{\setboolean{symbol}{false}}{\setboolean{symbol}{true}}
	
	% Gruppe
	\vspace{10cm}
	{\large\sffamily{Gruppe \groupnr}}
	
	%Logo
	\vfill

	\ifthenelse{\boolean{symbol}}
		{
			\begin{figure}[h]
			\begin{center}
				
				\includegraphics[width=2cm]{\logo}
				
			\end{center}
			\end{figure}
		}
	
\end{titlepage}

\restoregeometry

% Information
\chapter*{Informationen}
\label{chap:info}

\begin{tabular}{l l}

	{\textbf{Versuchstag}} \hspace{1cm} & \hspace{1cm} {\datum}\\[0,2cm]
	{\textbf{Versuchsplatz}} \hspace{1cm} & \hspace{1cm} {\platz}\\[0,2cm]
	{\textbf{Betreuer}} \hspace{1cm} & \hspace{1cm} {\betreuer}\\[1,2cm]
	{\textbf{Gruppen Nr.}} \hspace{1cm} & \hspace{1cm} {\groupnr}\\[0.2cm]
%	{\textbf{Auswertperson}} \hspace{1cm} & \hspace{1cm} {\auswertp}\\[0.2cm]
%	{\textbf{Messperson}} \hspace{1cm} & \hspace{1cm} {\messp}\\[0.2cm]
%	{\textbf{Protokollperson}} \hspace{1cm} & \hspace{1cm} {\protop}\\[0.2cm]

\end{tabular}

    \thispagestyle{empty}
    \cleardoublepage
    \tableofcontents
    \cleardoublepage

    % 1. Kapitel Einleitung
    % Manuel Lippert - Paul Schwanitz
% Physikalisches Praktikum

% 1. Kapitel Einleitung

\chapter{Einleitung}
\label{chap:einleitung}
% TODO: #33 Einleitung schreiben @PaulSchwanitz @ManeLippert

Betrachtet man die physikalischen Prozesse (bsp. Räuber-Beutemodell), die in unserer Umwelt ablaufen, so fällt schnell auf, dass ein Großteil dieser chaotisch bzw. nicht linear ablaufen. Daher ist es für die Physik sehr wichtig auch diese Prozesse zu verstehen. Dabei ist anzumerken, dass chaotisches Verhalten deterministisch ablaufen und keineswegs zufällig sind.\\
Der folgende Versuch soll daher Einblicke in das äußerst bedeutende Gebiet des Chaos und der nichtlinearen Dynamik geben und uns die grundlegenden Konzepte dieser Versuche verständlich machen.

    % 2.Kapitel Fragen zur Vorbereitung
    % Manuel Lippert - Paul Schwanitz
% Physikalisches Praktikum

% 2.Kapitel Fragen zur Vorbereitung

\chapter{Hintergrund zum Versuch}
\label{chap:theo}

% Text

% Input der Teilaufgaben je nach Produktion der Nebendateien ohne Ordner
% Manuel Lippert - Paul Schwanitz
% Physikalisches Praktikum

% Teilaufgabe 1

\section{Allgemeines zum Thema Chaos}
\label{sec:allgemeines}

\subsection{Dynamische Systeme}
\label{sub:dynamSys}
Die Formulierung eines \textit{Dynamischen Systems} in der Physik geschieht anhand von gewöhnlichen Differentialgleichungen mit dem Vektor $\vect{x}(t)=(x_1(t)$,..., $x_n(t))\in\mathbb{R}^n$
\begin{gather}
    \dot{\vect{x}}(t)=\vect{F}(\vect{x}(t)),
    \label{eq:dynamDGL}
\end{gather}
dabei beschreibt $\vect{x}(t)$ den \textit{\textbf{Zustand}} des Systems zum Zeitpunkt $t\in\mathbb{R}$.\\
Das dynamische System ist vollständig determiniert, wenn ein Zustand $\vect{x}(t)$ angegeben ist. Aus diesem Zustand lassen sich alle vorangegangen und folgenden Zustände des Systems bestimmen, womit das System nur von der Wahl des Anfangszustands abhängt. Dynamische Systeme können auch zeitdiskret angegeben werden, worauf aber hier nicht weiter eingegangen wird \citep{Lueck}.\\

\begin{itemize}
    \item[\textbf{1.}]\textbf{Phasenfluss}\\
    In der Mathematik wird ein dynamisches System durch den \textit{\textbf{Fluss}} bzw. \textit{\textbf{Phasenfluss}}  beschrieben. Unter dem \textit{Fluss} versteht man die Abbildung $\vect{\vect{\phi}}:\mathbb{R}^n\times\mathbb{R}\rightarrow\mathbb{R}^n$, welche die \textit{\textbf{Flussaxiome}} erfüllt \citep{Mat1}:
    \begin{gather}
        \begin{aligned}
            (1)~&\vect{\vect{\phi}}(\vect{x}_0,0)=\vect{x}_0\\
            (2)~&\vect{\vect{\phi}}(\vect{\phi}(\vect{x}_0,t),s)=\vect{\phi}(\vect{x}_0,t+s).
            \label{eq:flussaxiome}
        \end{aligned}
    \end{gather}
    Der \textit{Fluss} $\vect{\phi}$ ordnet \textbf{jedem} Anfangszustand $\vect{x}_0$ einen neuen Zustand zum Zeitpunkt $t$ zu \citep{Lueck}.\\

    \item[\textbf{2.}]\textbf{Trajektorie}\\
    Der \textit{Fluss} $\vect{\phi}$ kann mit dem \textit{Zustand} $\vect{x}(t)$ in Verbindung gebracht werden mit der Beziehung: $\vect{x}(t)=\vect{\phi}_{\vect{x}_0}(t)=\vect{\phi}(\vect{x}_0,t)$ mit festem $\vect{x}_0$, wobei nach  (\ref{eq:flussaxiome}) $\vect{x}(0)=\vect{\phi}_{\vect{x}_0}(0)=\vect{\phi}(\vect{x}_0,0)=\vect{x}_0$ gilt.\\
    Hierbei beschreibt $\vect{\phi}_{\vect{x}_0}(t)$ die \textit{Lösungskurve}, welche auch \textit{Bahnkurve}, \textit{Orbit}, \textit{Phasenbahn} oder \textit{\textbf{Trajektorie}} des Flusses $\vect{\phi}$ genannt wird und eine spezielle Lösung von (\ref{eq:dynamDGL}) darstellt, welche wiederum die Bewegung des Punktes $\vect{x}$ unter Wirkung des Flusses $\vect{\phi}$ mit dem Anfangszustand $\vect{x}_0$ beschreibt. \citep{Lueck}\\
    Durch die Abhängigkeit der \textit{Trajektorien} vom Anfangszustand $\vect{x}_0$ kann gefolgert werden, dass sich \textit{Trajektorien} mit unterschiedlichen Anfangszuständen $\vect{x}_0$ nicht schneiden können. Es können aber unterschiedliche Anfangszustände $\vect{x}_0$ auf derselben \textit{Trajektorie} befinden und sich nur um eine Zeittranslation unterscheiden \citep{Mat1}.\\

    \item[\textbf{3.}]\textbf{Phasenraum}\\
    Der \textit{\textbf{Phasenraum}} oder \textit{\textbf{Zustandsraum}} beschreibt eine Menge aller Zustände oder eine Darstellung aller Trajektorien eines dynamischen Systems und bietet einen Überblick über das Verhalten der gesamten Differentialgleichung ohne diese explizit lösen zu müssen \citep{Mat1}.
    %Der \textit{Phasenraum} wird vom Zustand $\vect{x}(t)$ und dessen Ableitung $\dot{\vect{x}}(t)$ aufgespannt bzw. ist eine $(\vect{x}(t),\dot{\vect{x}}(t))$-Ebene, was eine Parameterdarstellung der Differentialgleichung über die Zeit $t$ darstellt.
    %In diesem \textit{Phasenraum} lässt sich dann ein Vektor $(\vect{x},\dot{\vect{x}})$ definieren, welcher auf die \textit{Trajektorien}, die sich mit dem Anfangszustand $\vect{x}_0$ änderen, zeigt. Die Ableitung dieses Vektors $\frac{\text{d}}{\text{d}t}(\vect{x},\dot{\vect{x}})=(\dot{\vect{x}},\ddot{\vect{x}})$ erzeugt ein \textit{Vektorfeld} bzw. ein \textit{Richtungsfeld} der Differentialgleichung, dessen Vektoren tangential auf den \textit{Trajektorien} steht.

    \item[\textbf{4.}]\textbf{Attraktor}\\
    In (\ref{eq:dynamDGL}) wird ein Vektorfeld $\vect{F}$ im Phasenraum definiert, welches als Geschwindigkeitfeld des Phasenflusses $\vect{\phi}$ angesehen werden kann.\\ Durch Betrachtung der Divergenz des Vektorfelds $\nabla\vect{F}$ kann eine Aussage getroffen werden über die Rate mit dem sich ein Volumenelement $V$ unter der Wirkung des Flusses verändert. Zwei Fälle sind hier besonders hervorzuheben:
    \begin{gather}
        \begin{aligned}
            (1)~&\nabla\cdot\vect{F}=0\Rightarrow \dot{V}=0 \Rightarrow~\text{Konservatives System}\\
            (2)~&\nabla\cdot\vect{F}<0\Rightarrow \dot{V}<0 \Rightarrow~\text{Dissipatives System}
        \end{aligned}
    \end{gather}
    In einem dissipativen System laufen die Trajektorien nach einer Einlaufsphase (transiente Bewegung) in einem begrenzten Bereich im Phasenraum, welchen man als \textit{\textbf{Attraktor}} bezeichnet (Bewegung auf \textit{Attraktor}: permante oder posttransiente Bewegung). Ein Attraktor weist folgende Eigenschaften auf \citep{Lueck}:
    \begin{itemize}
        \item[(1)] Kompakte Menge im Phasenraum
        \item[(2)] Invariant unter der Wirkung des Flusses
        \item[(3)] Volumen des Attraktors ist Null
        \item[(4)] Eine beliebige Obermenge des Attraktors schrumpft unter der Wirkung des Flusses auf den Attraktor selbst zusammen   
    \end{itemize}
    \textbf{Arten von Attraktor}
    \begin{center}
        \begin{tabular}{ccc}
            \includegraphics[width=4cm]{FixpunktAttraktor.png}
            & \includegraphics[width=4cm]{GrenzzyklusAttraktor.png}
            & \includegraphics[width=4cm]{TorusAttraktor.png}
        \end{tabular}
        \captionof{figure}{Fixpunkt, Grenzzyklus, Tours- bzw. Ringattraktor \citep{Lueck}}
        \includegraphics[width=14cm]{SeltsameAttraktoren.png}
        \captionof{figure}{Seltsame Attraktoren: Rössler- und Lorentzattraktor, Eigens erstellt mit dem Python Packages matplotlib mit Hilfe von \citep{py1, py2}}
    \end{center}
\end{itemize}

\subsection{Deterministisches Chaos}
\label{sub:determChaos}
Systeme, die das Verhalten des \textit{deterministischen Chaos} zeigen, weisen ein zufällig erscheinendes Verhalten auf, was jedoch nicht durch äußere Umstände verursacht wird, sondern das Verhalten folgt aus den Eigenschaften des Systems selbst.\\
Das Verhalten von einem System mit deterministischem Chaos lässt sich langfristig nicht vorhersagen, da ähnliche Uhrsachen langfristig nicht zu ähnlichen Wirkungen führen. Das Prinzip der starken Kausalität wird somit verletz \citep{Lueck}.

\subsection{Fouriertransformation und Leistungsspektrum}
\label{sub:fouriertrafo}
Eine Fouriertransformation ist eine Integraltransformation, mit der aperiodische Signale in ein kontinuierliches Spektrum zerlegt werden können.\\
Fouriertransformation einer Messgröße \(x(t)\):
\begin{gather}
    \hat{x}(\omega) = \lim_{T \to \infty} \int_{0}^{T}  x(t) e^{-i\omega t}\,dt 
\end{gather}
Das Leistungsspektrum \( P(\omega)\) kann man nun folgendermaßen aus der Fouriertransformierten des Messwertes berechnen:
\begin{gather}
    P(\omega) = |\hat{x}(\omega)|^2
\end{gather}
Das Leistungsspektrum stellt die Leistungsanteile für unterschiedliche Frequenzen im Zeitsignal dar.\\
Bei chaotischen Systemen beispielsweiße erhält man ein Leistungsspektrum mit kontinuierlichem Verlauf, das für große Frequenzen abnimmt \citep{Lueck}. Da jedoch weißes stochastisches Rauschen ebenfalls ein kontinuierlichen Verlauf liefert kann dieser nicht als hinreichender Beweis für Chaos angesehen werden.

\subsection{Darstellungsweisen eines chaotischen Attraktor}
\label{sub:darstellungAttraktor}
\begin{itemize}
    \item[\textbf{1.}]{\textbf{Phasenraumdarstellung}}\\
    Die \textit{\textbf{Phasenraumdarstellung}} wie in (\ref{sec:allgemeines}) erwähnt, gibt einen Überblick über den Verlauf der Bewegung des dynamischen Systems. Dabei trägt man je an eine Raumachse eine Phasenraumvariabel auf (z.B. $x, \dot{x}$), wobei der Phasenraum dabei $n$-Dimensionen haben kann und nicht alle Phasenraumvariablen bekannt sein müssen, da ein Attraktor im Phasenraum rekonstruiert werden kann. Dazu werden die Messwerte einer Phasenraumvariablen bei einer festen Zeitspanne $\tau$, also $\varphi(t), \varphi(t+\tau)$,..., $\varphi(t+(n-1)\tau)$, als neu Koordinaten $x_i$ (z.B. $x_1=\varphi(t)~\text{und}~x_2=\varphi(t+\tau))$ eines neuen Koordinatensystems. Bei richtiger Wahl von $\tau$ und $n$ lässt sich dann der tatsächliche Attraktor rekonstruieren. Diese Methode der Rekonstruktion eines Atrraktors wurde vom Mathematik \textit{Floris Takens} bewiesen und ist auch als \textit{Einbettungstheorem} bekannt \citep{Lueck}.
    \item[\textbf{2.}]{\textbf{Poincar\'e-Abbildung}}\\
    Die \textit{\textbf{Poincar\'e-Abbildung}} ist eine Projektion des \textit{\textbf{Poincar\'e-Schnitts}} an einer Ebene. Diese Abbildung ist immer die Dimension $n-1$ und ist somit eine Dimension niedriger als der Phasenraum mit der Dimension $n$ und es gehen keinen Informationen bzgl. dem Langzeitverhalten des Systems verloren. Aufgrund dieser Tatsache lässt sich die \textit{Poincar\'e-Abbildung} zur Analyse von höherdimensionalen Phasenräumen verwenden.\\
    Der \textit{Poincar\'e-Schnitt} ist dabei eine Menge aller Durchstoßpunkte der Trajektorien im Phasenraum auf einer Hyperfläche. Hierbei müssen die Trajektorien die Hyperfläche \textit{transversal} (senkrecht) und in einer vorgegebenen Richtung schneiden. Praktisch werden meistens die Lage von Extremwerten einer Messgröße als Bedingung für die Schnittebene (Hyperfläche) und trägt diese gegen die anderen Phasenraumvariablen zu diesem Zeitpunkt gegeneinander auf oder man wählt eine Ebene, die den Attraktor geeignet schneide \citep{Lueck}.
    \item[\textbf{3.}]{\textbf{Wiederkehr-Abbildung}}\\
    Bei der \textit{\textbf{Wiederkehr-Abbildung}} wird eine diskrete Abbildung aktueller Messwerte über die vorangegangenen Messwerte aufgetragen. Dabei werden Punkte in einer $x(n)$-$x(n+1)$-Ebene aufgetragen mit den Messwerten $x_n$ als Koordinaten, also ($x_0,x_1$), ($x_1,x_2$),...,($x_n,x_{n+1}$). Diese Abbildung ähnelt dann einem \textit{Poincar\'e-Schnitt}, weswegen man bei einem kontinuierlichen System (\ref{eq:dynamDGL}) dessen \textit{Poincar\'e-Abbildung} für dieses Verfahren verwendet \citep{Lueck}.
    % TODO: #37 Recherche + Quelle zu Wiederkehrabbildung @ManeLippert
    \item[\textbf{4.}]{\textbf{Bifurkationsdiagramm}}\\
    Bei einem \textit{\textbf{Bifurkationsdiagramm}} betrachtet man die Projektion der \textit{Poincar\'e-Abbildung} auf \textbf{eine} Achse unter Veränderung eines Kontrollparameters, Parameter welche man aktiv im Experiment verändern kann, wobei man die durch die Projektion gewonnenen Werte gegen den jeweiligen Parameterwert aufträgt \citep{Lueck}.
    \newpage
    \item[\textbf{5.}]{\textbf{Phasendiagramm}}\\
    Wenn bei dem \textit{Bifurkationsdiagramm} mehrere unterschiedliche voneinander unabhängige Parameter existieren, verwendet man das \textit{\textbf{Phasendiagramm}}. Dazu trägt man die als Koordinatenachsen die jeweiligen Parameter, die das Systemverhalten beeinflussen, gegeneinander auf und erhält Landkarte des globalen Systemverhaltens in Abhängigkeit der gewählten Parameter \citep{Lueck}.  
\end{itemize}
% Manuel Lippert - Paul Schwanitz
% Physikalisches Praktikum

% Teilaufgabe 2

\section{Das invertierte Pendel}
\label{sec:invertPendel}
Das invertierte Pendel erzeugt eine \textit{nichtlinearer} Schwingung. Dabei besitzt das Pendel eine unten fest eingespannte Blattfeder mit Federkonstante $k$, welche über zwei horizontal in der Höhe $h$ und der Auslenkung $x_h$ angreifende Spiralfedern mit Federkonstante $k_s$ und der Auslenkung $\hat{x}$ angetrieben wird. Am oberen Ende der Blattfeder lässt sich ein Zusatzgewicht $M$ anbringen und der Winkel $\theta$ beschreibt den Winkel zwischen der Tangenten an der Pendelspitze und dem Lot. Weiterhin bezeichnet die Pendellänge $L$ die Länge zwischen dem Anfang und Ende der Blattfeder, welche vom Winkel $\theta$ abhängt (siehe Abb \ref{image:invertiertesPendel}).
\begin{center}
    \includegraphics[scale=0.25]{Pendel/invertiertesPendel.png}
    \captionof{figure}{Skizze invertiertes Pendel \citep{Lueck}}
    \label{image:invertiertesPendel}
\end{center}
\subsection{Herleitung der Bewegungsgleichung}
\label{sub:bewegungsgleichung}
Die Bewegungsgleichung lässt sich über die wirkenden Drehmomente der Bauteile bestimmen:
\begin{gather*}
        \text{Pendel}~+~\text{Dämpfung}~+~\text{Blattfeder}~-~\text{Spiralfedern}~-~\text{Gewicht} = 0
\end{gather*}
\begin{gather}
    \Rightarrow [M(L(\theta))^2\ddot{\theta}]+[2\delta\dot{\theta}]+[k\theta]-[hk_s(x_h+\hat{x}\cos(\omega t))]-[MgL(\theta)\sin(\theta)]=0
\end{gather}
Dabei bezeichnet $\delta$ die Dämpfungskonstante und $g$ die Erdbeschleunigung.\\
Im Folgenden wird die Pendellänge $L$ als konstant angenommen, obwohl diese nicht konstant ist und von der Art der verwendeten Masse $M$ abhängt. Weiterhin wird die Auslenkung $x_h$ als vernachlässigbar klein angesehen und der Angriffswinkel der Spiralfedern wird als $\frac{\pi}{2}$ genähert, was bei genügend kleiner Höhe $h$ gegeben ist.
Daraus folgt die genäherte Form der Bewegungsgleichung:
\begin{gather}
    ML^2\ddot{\theta}+2\delta\dot{\theta}+k\theta-MgL(\theta)\sin(\theta)=hk_s\hat{x}\cos(\omega t)=T_0\cos(\omega t)
\end{gather}
Wobei $T_0$ als die Amplitude des periodisch angreifenden Drehmoments interpretiert werden muss. Hierbei ist noch zu erwähnen, dass durch die Näherungen die Lösung dieser Differentialgleichung nicht der tatsächlichen Trajektorien des Systems entsprechen, da diese stark von der Anfangsbedingung abhängen, dennoch ist eine globale Aussage über das Verhalten mit der genähreten Differentialgleichung möglich.\\
Zu den letzten beiden Termen lässt sich dann ein Potenzial definieren und durch Entwicklung des Cosinus für kleine Winkel $\theta$ (Kleinwinkelnäherung KWN) bis zur 2.Ordnung ergibt sich das Potenzial des \textit{Duffing-Oszillator} \citep{Lueck}.
\begin{gather}
    V(\theta) = \frac{1}{2}k\theta^2 + MgL(\cos(\theta)-1)\overset{\text{KWN}}{\approx}\frac{1}{2}(k-MgL)\theta^2 + \frac{1}{24}MgL\theta^4~,V(0)=0
\end{gather}

\subsection{Symmetriebrechung}
\label{sub:symbrechung}
Bei einer {kritischen Masse} $M_k=\frac{k}{gL}$ erfährt das System des Pendels einen Übergang von einem monostabilen System ($M<M_k$) in ein bistabiles System ($M>M_k$), dabei ist die Bewegung des Pendels in beiden Fällen Unterschiedlich und muss deshalb getrennt betrachtet werden. Bei diesem Übergang verändert sich die Struktur des Potenzial (siehe Abb \ref{image:potPendel}), wodurch nun zwei Lösungen für das System möglich sind. Das Auftreten von mehreren Lösungen in einem System wird dann auch \textit{\textbf{Symmetriebrechung}} genannt \citep{Lueck}.
\begin{center}
    \includegraphics[scale = 0.2]{Pendel/PendelPotenzial.png}
    \captionof{figure}{Potenzialdarstellung für invertiertes Pendel \citep{Lueck}}
    \label{image:potPendel}
\end{center}

\subsection{Schwingungsdauer in Abhängigkeit der Masse}
\label{sub:schwingungsdauer}
Bei einem nichtlinearen Pendel hängt, im Gegensatz zu einem linearen Schwingungsvorgang, die Resonanzfrequenz $\omega_r$ von der Schwingungsamplitude $b$ ab $\Rightarrow \omega_r(b)$. Unverändert bleibt dennoch die Schwingungsamplitude $b(\omega)$ gegenüber der Resonanzkurve eines linearen Oszillators bei unterschiedlichen Anregungsfrequenzen $\omega$ \citep{Lueck}.
\begin{itemize}
    \item[1.] $M<M_k$ (Schwache Nichtlinearität)\\
    Pendel ist nach (\ref{sub:symbrechung}) monostabil. Der Vorgang lässt sich näherungsweise mit der Bewegungsgleichung eines Duffing-Oszillator nähern, wobei die Abhängigkeit der Resonanzfrequenz $\omega_r$ von der Amplitude $b$ berücksichtigt bleibt. Die Bewegungsgleichung lautet in diesem Fall:
    \begin{gather}
        \ddot{\theta} + 2\delta\dot{\theta} + \omega_0^2\theta + \gamma\theta^3 = f_a\cos(\omega t)
    \end{gather}
    Dabei bezeichnet $\gamma$ den Faktor der Nichtlinearität, $\omega_0$ die Resonanzfrequenz des Systems ohne Nichtlinearität ($\gamma=0$), $f_a$ die Anregungsamplitude mit Anregungsfrequenz $\omega$ und $\delta$ die Dämpfung.\\
    Durch Betrachtung des Verlaufs der Resonanzkurve in der Nähe von $\omega_0$ erhält man eine Gleichung dritter Ordnung für das Quadrat der Schwingungsamplitude $b$.
    \begin{gather}
        \left[\left((\omega^2-\omega_0^2) - \frac{3}{4}\gamma b^2\right)^2+(2\delta\omega)^2\right]b^2=f_a^2 %Angepasst an die Form in Wikipedia https://en.wikipedia.org/w/index.php?title=Duffing_equation&oldid=1031816809
        \label{eq:resonanzkurve}
    \end{gather}
    Diese Gleichung hat je nach Werten von $f_a, \gamma, \omega_0~\text{und}~\delta$ eine reelle oder zwei konjugierte komplexe Lösungen.
    Dabei ist es einfacher die Gleichung nach $\omega$ aufzulösen, wobei zur Vereinfachung $\omega_0=1$ angenommen wird. Damit wird (\ref{eq:resonanzkurve}) zu:
    \begin{gather}
        \left[\left((\omega^2-1) - \frac{3}{4}\gamma b^2\right)^2+(2\delta\omega)^2\right]b^2=f_a^2
    \end{gather}
    Mit der Lösung für $\omega$:
    \begin{gather}
        \omega^2_{1,2} = 1 - 2\delta^2 + \frac{3}{4}\gamma b^2 \pm \sqrt{\frac{f_a^2}{b^2}+ 4\delta^2\left[\delta^2 - \left(1 + \frac{3}{4}\gamma b^2\right)\right]}
    \end{gather}
    Bei hinreichender kleinen Dämpfung $\delta$ gibt es zwei verschiedene eingeschwungene Zustände, da die Lösung instabil wird (siehe Abb \ref{image:resonanzkurve}c)).
    \begin{center}
        \includegraphics[scale=0.3]{Pendel/Resonanzkurve.png}
        \captionof{figure}{Resonanzkurven für eines Duffing-Oszillator \citep{Lueck}}
        \label{image:resonanzkurve}
        \captionof*{figure}{a) linearer ($\gamma=0$) b) nichtlinear c) Hysterese}
    \end{center}
    % https://www.ila.uni-stuttgart.de/nlvib/downloads/HB_NLvib_presentation.pdf, https://www.hindawi.com/journals/mpe/2011/248328/
    Die Schwingungsdauer $T$ hängt hierbei logarithmisch von der Amplitude $b$ ab mit dem Zusammenhang:
    \begin{gather}
        T = T_0 + T_1\log(b)
    \end{gather}
    Dies geht aus experimentellen Daten von \citep{Lueck} hervor, wobei $T_0$ und $T_1$ Näherungsparameter sind.
    \item[2.] $M>M_k$ (Starke Nichtlinearität)\\
    Das Pendel ist in diesem Fall nach (\ref{sub:symbrechung}) bistabil und besitzt zwei stabile Ruhelagen. Durch die Verringerung von großen Antriebsfrequenzen $\omega$ entstehen nach Ende des Einschwingverhaltens nacheinander subharmonische Schwingungen mit einer Periodenverdopplungskaskade ($T_n=2^nT=2^n\frac{2\pi}{\omega}$), obwohl sich das Pendel unterhalb einer kritischen Frequenz $\omega_k$ chaotisch verhält. Dieses Verhalten ist auch ein Beispiel für ein Bifurkationsszenario \citep{Lueck}.
\end{itemize}
\newpage
\subsection{Aufbau Pendel}
\label{sub:aufbauPendel}
\begin{center}
    \includegraphics[scale=0.25]{Pendel/AufbauPendel.png}
    \captionof{figure}{Aufbau des invertierten Pendels \citep{Lueck}}
    \label{image:aufbauPendel}
\end{center}
Hierbei besteht das Pendel aus einer:
\begin{itemize}
    \item 5 mm starken Aluminium-Grundplatte
    \item 1 cm x 15 cm x 40 cm lange Blechstreifen aus einer Messing-Legierung mit hohem Kupferanteil, Dehnungsmessstreifen auf beiden Seiten (DMS, Widerstand abh. von der Dehnung) knapp oberhalb der Befestigung $\rightarrow$ Blattfeder
    \item Spiralfedern mit Federkonstante $k=027$ N/cm 
    \item Schrittmotor im Gehäuse mit 200 bzw. 400 Schritten (Halbschrittbereich)
    \item Multifunktionskarte Typs DAS 1602 der Firma Keithley (Taktimpulsgeber für Schrittmotoren)
\end{itemize}
Mit diesem Aufbau sind bis zu ca 5 Umdrehungen/s möglich, wobei die Antriebskraft durch das Verändern des Angriffspunktes der Spiralfeder am Übertragungshebel variierbar ist \citep{Lueck}.

\subsection*{Funktionsweise Differenzier-Schaltung}
Bei einer Differenzier-Schaltung wird nur die Änderung der Eingangsspannung zu einer Ausgangsspannung verarbeitet. Dabei wird ein Kondensator am Eingang in Reihe und ein Widerstand parallel zwischen Eingang und Ausgang des Operationsverstärker geschaltet. Durch den Kondensator fließt nur Strom, wenn sich die Eingangsspannung ändert, wobei die Ausgangsspannung proportional zur Änderungsgeschwindigkeit der Eingangsspannung ist. Durch den Operationsverstärker wird dann das Signal verstärkt, um dieses Signal besser über ein angeschlossenes Messgerät (z.B. Oszilloskop) betrachten zu können \citep{electronik}.

\begin{center}
    \includegraphics[scale=0.3]{Pendel/MessschaltungPendel.png}
    \captionof{figure}{Messschaltung des invertierten Pendels \citep{Lueck}}
    \label{image:schaltungPendel}
\end{center}
Für die Messschaltung (siehe Abb. \ref{image:schaltungPendel}) wird für eine höhere Genauigkeit zwei DMS in einer Brückenschaltung verschalten, um die Spannungsdifferenz von ihnen zu messen, wobei die Spannung dann proportional zur Pendelauslenkung $\theta$ ist. Für das Abgreifen der Geschwindigkeit als zweite Phasenraumvariable werden die Spannungen über eine Operationsverstärker-Schaltung differenziert. Die Spannungen werden dann einem im PC eingebauten Analog-Digital-Wandler-Karte gemessen, wobei die Messkarte über LABVIEW (Messprogramm) gesteuert wird \citep{Lueck}.
% Manuel Lippert - Paul Schwanitz
% Physikalisches Praktikum

% Teilaufgabe 3
\newpage
\section{Der Shinriki-Oszillator}
\label{sec:shinrikiOszi}


\subsection{Differentialgleichung und Aufbau des Shinriki-Oszillator}
\label{sub:dgl}

\begin{figure}[h]
    \centering
    %TODO #31
    \includegraphics[scale=0.15]{ShinrOsziSp.jpeg}
    \label{fig:shinrikiSp}
    \caption{Schaltplan des Shinriki-Oszilator \citep{Lueck}}
\end{figure}

Der Shinriki-Oszillator besteht aus einem negativen Impedanzkonverter (NIC) und einem LC-Parallelschwinkreis, die durch ein gegeneinader geschaltetes Zenerdiodenpaar und dem parallel geschalteten \(R_2\), gekoppelt sind. \\
Die Leitwertfunktion des Kopplungsglied ist \( f(V)\) und beschreibt den Strom, der über das Kopplungsglied fließt.
\(R_{NIC}\) ist der Widerstand des NIC innerhalb des relevanten Intervalls von -8,1 V bis 8,1 V \citep[]{Lueck}.\\
Damit und mit den Kirchhoffschen Regeln lassen sich nun die DGLs aufstellen:
\begin{align}
    C_1 \dot{V_1} &= V_1 (\frac{1}{R_{NIC}}-\frac{1}{R_1}) - f(V_1-V_2) \\
    C_2 \dot{V_2} &= f(V_1-V_2) - I_3 \\
    L \dot{I_3} &= -I_3R_3 + V_2
\end{align}

\subsection{NIC und Schwingung des Shinriki-Schaltkreis}
\label{sub:nic}
Ein NIC benutzt einen Operationsverstärker, um einen negativen ohmschen Wiederstand zu simulieren. Hierbei wird der gewünschte Widerstand einfach zwischen dem (-) Eingang des OpAmp und GND geschaltet. Durch den OpAmp wird ein Widerstand mit negativem Wert des eben eingesetzten simuliert. \\
Daher mus das System nicht mehr von außen zur Schwingung angeregt werden.
% TODO #29

\subsection{Geräusche einer Bifurkation}
\label{sub:tonBifurkation}
Eine Bifurkation ist eine verdopplung der Periodendauer, d.h. die Frequenz wird halbiert. Dies verursacht einen tieferen Ton.
% TODO #30

% etc.

    % 3.Kapitel Protokoll
    % Manuel Lippert - Paul Schwanitz
% Physikalisches Praktikum

% 3.Kapitel  Protokoll

\chapter{Messprotokoll}
\label{chap:protokoll}


\section{Versuchsdurchführung invertiertes Pendel}
\label{sec:versuchPendel}
\subsection*{Bifurkationsdiagramm}
\label{sub:bifu}
Vermessung der Gleichgewichtslage $\theta_g$ in Abhängigkeit der Masse $M$. Dabei wird die Gleichgewichtslage über eine Spannung $U_a$, welche durch den in (\ref{sub:aufbauPendel}) beschriebenen Schaltkreis erzeugt wird. Dabei wird angenommen (siehe (\ref{sub:aufbauPendel})), dass $U_a$ proportional zu dem Auslenkwinkel im Gleichgewicht $\theta_g$ ist und damit der generelle Verlauf der Graphen identisch ist.
\begin{itemize}
\item Messfehler Waage: $s_a = 0,005\, g = s_r$
\item Messfehler Multimeter: $s_a = 0,00005$ V$= s_r$\\
      Durch starke Schwankungen am Multimeter verändert sich der Wert des Fehlers des Multimeters mit der Zunahme der Masse $M$
\item Messfehler Stoppuhr: $s_a = 0,01$ s $=s_r$
\item Länge des Pendels (mit Stahlmaßstab): $l = 37$ cm
\item Gewicht Feststellschraube: $m_s = 3,14$ g
\end{itemize}
Datei: BifurkationPendel.csv

\textbf{Verifikation Ergebnis:}\\
Freie Schwingung des Pendels bei einer Auslenkung bis ca. 1V. Messung der Schwingunsdauer $T$ (10fach) mit Smartphone (Google Pixel 5). Über die Schwingungsdauer wird dann die Federkonstante $k$ des Pendels berechnet.

Datei: Schwingunsdauer\_woMass.csv; Messung ohne Masse.\\
Datei: Schwingunsdauer\_wMass.csv; Messung mit 12,58 g.

Grobe Auswertung: Kritische Masse $M_k\approx19,3$ g wurde durch Überschlagsrechnung bestätigt.

\subsection*{Schwache Nichtlinearität}
\label{sub:weakLin}
Montage des Dämpfungssegels, wobei dabei zu beachten ist, dass die kritsche Masse $M_k\approx19,3$ g nicht überschritten wird, damit man im monostabilen Zustand des Pendels bleibt.
\begin{itemize}
    \item Masse Dämfungssegel: $m=4,4$ g
    \item Zusätzlich montierte Masse: $m = 10,42$ g
\end{itemize}
Die Masse $M_{total}= 14,41$ g bleibt hierbei über den ganzen Versuchsteil unverändert und das Dämpfungssegel befestigt über den kompletten restlichen Versuchsverlauf.
\paragraph{a)}
Für die Amplitudenabhängigkeit des Pendels wird dieses einmal ausgelenkt und dessen Schwingung über das Messprogramm aufgezeichnet.

Datei: 06\_09\_2021\_14\_41\_30\_G11\_pendel\_0.dat

\paragraph{b)}
Messung der Hystereseschleife der Schwingung mit Messprogramm.

Einstellungen: 2000 Steps; Start: 0Hz; End: 1,1Hz\\
Datei: 06\_09\_2021\_14\_30\_52\_06\_09\_2021\_14\_30\_52\_G11\_pendel\_resonanz\_\_0.dat\\

\subsection*{Starke Nichtlinearität}
\label{sub:strongLin}
Veränderung der Masse über die kritische Masse für einen bistabilen Zustand, beachte das Dämpfungsblech aus (2a) ist immer noch mit befestigt.
\begin{itemize}
    \item Zusätzlich montierte Masse: $m = 19,44$ g
\end{itemize}
Diese Masse $M_{total}=23,84$ g bleibt auch über diesen Versuchsteil unverändert.

\paragraph{a)}
Variation der Antriebsfrequenz $\omega$ in kleinen Schritten zur Lokalisierung der Schwingungszustände. Aufnahme der Attraktoren und Leistungsspektren mit dem Messprogramm.

Labview Absturz bei 0,517 Hz 

Datei: Datum\_Uhrzeit\_G11\_pendel\_0.dat (Mehrere Files mit selben Namen in zugehörige Ordner gespeichert)

Neustart bei 0,52 Hz
quasichaotisch bei $\omega\approx0,411$ Hz

Datei: Datum\_Uhrzeit\_Richter\_pendel\_0.dat (Wegen Neustart Dateienbennung verändert gewesen)

Frequenzen den einzelnen Schwingungszuständen werden den Daten entnommen.

\paragraph{b)}
Verdeutlichung der Empfindlichkeit der Anfangsbedingung mit Ruhelage auf der linken Seite (Pendel hängt auf die linke Seite). Aufnahme von drei Zeitserien Trajektorien (eine mehr als benötigt) mittels des Messprogramms.
\begin{itemize}
    \item Anregungsfrequenz $\omega$: 0,411Hz
\end{itemize}
Schrittmotor in Anfangspos. (Armstellung ganz unten)

Datei: Datei: Datum\_Uhrzeit\_G11\_pendel\_0.dat (Mehrere Files mit selben Namen in zugehörige Ordner gespeichert)

\section{Versuchsdurchführung Shinriki}
\label{sec:versuchShin}

\textbf{Einstellfehler der Widerstände}
\begin{align}
    s_{R_1} = 1\, \text{k}\Omega \\
    s_{R_2} = 2\, \text{k}\Omega
\end{align}

\textbf{a) Phasendiagramm}\\
Wir stellen einzelne Paramterwerte für $R_2$ und $R_1$ ein und varieren je nach eingestelltem Parameter mit $R_2$ oder $R_1$ bis das Phasendiagramm vollständig abgefahren ist ab. Widerstände schwer ablesbar, weswegen Angaben fehlerhaft sein können, $R_1$ konnte hierbei genauer bestimmt werden. Alle Werte werden in $\text{k}\Omega$ angeben.\\
\begin{tabular}{c  r| c  r  r  r  r  r  r  r}
    Par &  & Var & Per1 & Per2 & Per4 & Chaos1 & Per3 & Chaos2 & Double\\
    \hline
    $R_2$ & 16,50 & $R_1$ & 13,40 & 20,10 & 21,10 & 21,46 & 22,20 & 22,30 & 24,62\\
    $R_2$ & 13,00 & $R_1$ & 16,16 & 23,68 & 25,19 & 25,86 & 26,72 & 26,82 & 28,98\\
    $R_1$ & 55,00 & $R_2$ & 7,88  & 8,52 & 8,72 & 8,76 & 8,92 & 8,96 & 9,16\\
\end{tabular}\\

\textbf{b) Schnitt im Phasendiagramm}\\
Wir schneiden bei fixierten $R_2=8,4~\text{k}\Omega$ durch das Phasendiagramm.
Daten werden elektronisch erstellt.

\textbf{c) Bifurkationsdiagramm}\\
Nutzen oben verwendeten Schnitt für diese Aufgabe zum Erstellen eines Bifurkationsdiagramms. Dies geschieht wieder elektronisch.

\textbf{d) Großmann-Feigenbaum-Konstante}\\
Wir vermessen nun gesondert die einzelnen Bifurkationen durch Variation von $R_1$ mit gleichen $R_2$ aus (b). Alle Werte werden $\text{k}\Omega$ angegeben. Hierbei steht der Index $i$ in $r_i$ für die einzeln Bifurkationen.\\
\begin{tabular}{c c c}
    $r_1$ & $r_2$ & $r_3$\\
    \hline
    59 & 65,6 & 67,6
\end{tabular}

\begin{itemize}
    \item $r_1$ Periode 1 auf 2
    \item $r_2$ Periode 2 auf 4
    \item $r_3$ Periode 4 auf 8
\end{itemize}

\textbf{e) Einbettungstheorem}\\
Aufnahme des Originalattraktors bei $R_2$ wie in (b) und $R_1=51~\text{k}\Omega$ und $\delta t=60$ n (?); Rekonstruktion mithilfe des entsprechenden Programmteils mit qualitativen Übereinstimmung der Form des rekonstruierten Attraktors mit dem Originalattraktor.


% Einbindung des Protokolls als pdf (mit Seitenzahl etc.)
% Erste Seite mit Überschrift
%\includepdf[pages = 1, landscape = false, nup = 1x1, scale = \skalierung , pagecommand={\thispagestyle{empty}\chapter{Protokoll}}]
%            {03-Protokoll/Protokoll.pdf}
% Restliche Seiten richtig skaliert
%\includepdf[pages = -, landscape = false, nup = 1x1, scale = \skalierung , pagecommand={}]
%            {03-Protokoll/Protokoll.pdf}

    % 4.Kapitel Versuchsauswertung
    % Matteo Kumar - Leonard Schatt
% Fortgeschrittenes Physikalisches Praktikum
% 4.Kapitel Versuchsauswertung

\chapter{Auswertung und Diskussion}
\label{chap:versuchsauswertung}

% Text

% Input der Teilauswertung je nach Produktion der Nebendateien ohne Ordner


%Teilauswertung Brewster

\section{Bestimmung des Verstärkungsfaktors}
\subsection{Brewster- und Grenzwinkel}

Bei der Aufnahme des Intensitätsplots übersteuerte die Diode teilweise, 
weshalb Abb.\ref{plot:Talpha}, in dem die transmittierte Intensität in relativen Einheiten gegen den Einfallswinkel aufgetragen ist, nicht den vollständigen Verlauf zeigt, 
da der Graph bei der relativen Einheit 1 abgeschnitten ist. 
%Der Verlauf des Transmissionskoeffizienten $T = \frac{I_T}{I_0}$ aufgetragen gegen den Winkel des Glasplättchens ist in Abb.\ref{plot:Talpha} gezeigt.

\begin{figure}[h]
    \centering
    \includegraphics[scale = 1.5]{Bilder/Auswertung/brewsterpfeil.png}
    \caption{Der Winkel des in den Resonator eingebrachten Glasplättchens $\alpha$ wird von 60°-90° in 0,01°-Schritten variiert. Aufgetragen ist die transmittierte Intensität 
    des Lasers in relativen Einheiten gegen $\alpha$ in Grad. Dabei übersteuerte die verwendete Photodiode regelmäßig. Auffällig ist der oszillierende Verlauf. 
    Zudem sind die Grenzwinkel $\alpha_G$ und der Brewsterwinkel $\alpha_B$ eingezeichnet.}
    \label{plot:Talpha}
\end{figure}

Im Graphen ist zunächst einmal der oszillierende Verlauf des Transmissionskoeffizienten auffällig. Dies lässt sich durch die Airy-Formel für die transmittierte Intensität 
erklären. Sie lautet 
\begin{equation}
    I_T = I_0 \frac{1}{1+F \sin^2(\frac{\Delta \phi}{2})}  
    \label{eq:airy}
\end{equation}
mit den Abkürzungen 
\begin{equation}
    F = \frac{4R}{(1-R)^2} \quad \mathrm{ und } \quad \Delta \phi = \frac{4\pi}{\lambda}d\sqrt{n^2-\sin^2\alpha},
    \label{eq:airyPar}
\end{equation} 
wobei $R$ der Reflektionskoeffizient ist.\\
In Gl.\ref{eq:airy} ist zu sehen, dass das $I_T = I_0$ für $\sin^2(\frac{\Delta \phi}{2}) = 0$ gilt. Da $\Delta \phi$ wiederum von $\sin^2\alpha$ abhängt, oszilliert auch die 
transmittierte Intensität.\\
Der Brewsterwinkel $\alpha_B$ ist derjenige Winkel, bei dem zur Einfallsebene parallel polarisiertes Licht vollständig transmittiert wird. Folglich erwarten wir $\alpha_B$ 
im absoluten Maximum in Abb.\ref{plot:Talpha}. Da die Photodiode aber regelmäßig übersteuerte, ist diese Identifizierung in diesem Fall nicht möglich. Stattdessen können wir 
aber einfach das maximale Minimum der transmittierten Intensität auslesen und für den Ablesefehler den Bereich der benachbarten Maxima einschließen, wodurch dieser sich zu 0,2° ergibt. 
Dazu muss noch der Fehler aus der Justage des Plättchens berücksichtigt werden. Da diese nicht ganz einfach war, soll dieser 0,5° betragen. 
Somit ergibt sich ein Brewsterwinkel von 
\begin{equation*}
    \textcolor{red}{\alpha_B = (55,7 \pm 0,6)^\circ}.
\end{equation*}
Die Grenzwinkel $\alpha_G$ sind definiert als die Winkel, bei denen der Laser erstmals anspringt bzw. erlischt, die transmittierte Intensität also auf 0 abfällt. Diese lesen sich 
ab zu 
\begin{equation*}
    \textcolor{red}{\alpha_{G1} = (46,1 \pm 0,5)^\circ \quad} \mathrm{ und } \textcolor{red}{ \quad \alpha_{G2} = (63,6 \pm 0,5)^\circ},
\end{equation*}
wobei wieder der Fehler der Justage berücksichtigt wurde.
Der Brewsterwinkel ist verknüpft mit dem Brechungsindex über die Beziehung 
\begin{equation*}
    \tan\alpha_B = \frac{n_2}{n_1}.
\end{equation*}
Für $n_1 = 1$ (Luft) folgt der Brechungsindex des Glasplättchens mit Fehlerfortpflanzung zu 
\begin{equation*}
    \textcolor{red}{n_{Glas} = 1,466 \pm 0,033},
\end{equation*}
was innerhalb des Fehlers sehr gut mit dem Brechungsindex von Quarzglas übereinstimmt, der bei 1,45886 liegt. (\cite{Mende2016}, S.302) 
Demnach kann unser Ergebnis für $\alpha_B$ als sinnvoll angesehen werden.
$\alpha_B$ und die $\alpha_G$ sind auch in Abb.\ref{plot:Talpha} eingezeichnet. Dabei ist eine leichte Asymmetrie des Graphen bezüglich $\alpha_B$ zu erkennen. Gl.\ref{eq:airy} ist zwar auf den 
ersten Blick symmetrisch, doch betrachtet man das darin vorkommende $F$ in Gl.\ref{eq:airyPar} genauer, so stellt man eine Abhängigkeit vom Reflektiosnkoeffizienten $R$ fest. 
Dieser ist über die Fresnel'schen Formeln definiert als 
\begin{equation*}
    R =\biggl (\frac{E_{0r}}{E_{0e}} \biggl)^2 = \biggl(\frac{n_2\cos\alpha - \cos\beta}{n_2\cos\alpha + \cos\beta} \biggl)^2 = \biggl(\frac{n_2^3\cos\alpha - (n_2^2-\sin^2\alpha)}{n_2^3\cos\alpha - (n_2^2-\sin^2\alpha)} \biggl)^2,
\end{equation*}
wobei $n_1 = 1$ angenommen wurde. In dieser Gleichung ist zu sehen, dass der Reflektionskoeffizient $R$ wiederum von $\alpha$ abhängt. Trägt man diese nun gegeneinander auf, so ist zu erkennen, dass keine 
Symmetrie bezüglich $\alpha_B$ existiert (\cite{Demtroeder2017}, S.224). Deshalb ist auch im Graphen der transmittierten Intensität eine solche Asymmetrie zu sehen.


\subsection{Verstärkungsfaktor}
Um den Verstärkungsfaktor des laseraktiven Mediums zu berechnen, ist es sinnvoll zunächst einmal die Gewinn-Verlust-Bilanz bei einem Durchlauf im Resonator aufzustellen. 
Dazu betrachte man Abb.\ref{pic:SchemaBrew}, die aus dem Versuchsskript entnommen wurde.
\begin{figure}[h]
    \centering
    \includegraphics[scale = 0.5]{Bilder/Auswertung/SchemaBrew.png}
    \caption{Schematischer Durchlauf durch den Resonator mit eingebautem Glasplättchen; aus dem Versuchsskript entnommen.}
    \label{pic:SchemaBrew}
\end{figure}
Im Resonator berechnet sich die Intensität $I_6$ nach einem Durchlauf (die in Abb.\ref{pic:SchemaBrew} äquivalent zu $I_0$ liegt) wie folgt:
\begin{align*}
    I_6 &= R_1 I_5 = \\
    &= R_1 v I_4 = \\
    &= R_1 v T I_3 = \\
    &= R_1 v T R_2 I_2 = \\
    &= R_1 v T^2 R_2 I_1 = \\
    &= R_1 v^2 T^2 R_2 I_0\\
\end{align*}
Hierbei sind $R_{1/2}$ die Reflektiosnkoeffizienten der Spiegel 1 bzw. 2 und $T$ der Transmissionskoeffizient des Glasplättchens. Im Bereich der Grenzwinkel ist die 
Gewinn-Verlsut-Billanz ausgeglichen, der Resonator ist im Leerlaufmodus ($I_0 \rightarrow 0$). Mit der Bedingung $I_0 = I_6$ folgt für den Verstärkungsfaktor 
\begin{equation*}
    v = \frac{1}{\sqrt{R_1R_2}T}.
\end{equation*}
Nimmt man die Strahlung durch das Glasplättchen als verlustfrei an, so gilt $T = 1-R$ und über die Fresnell'sche Formel
\begin{equation*}
    R = \frac{\tan(\alpha - \beta)}{\tan(\alpha + \beta)}
\end{equation*}
folgen mit dem Snellius'schen Brechungsgesetz
\begin{equation*}
    \beta = \arcsin(\frac{\sin\alpha}{n_{Glas}})
\end{equation*}
die Verstärkungsfaktoren für $\alpha_{G1}$ und $\alpha_{G2}$ zu 
\begin{equation*}
    v_1 = 1,0168 \pm 0,0004 \quad \mathrm{ und } \quad v_2 = 1,020 \pm 0,011,
\end{equation*}
wobei folgende Beziehungen für die Fehlerrechnung verwendet wurden:
\begin{align*}
    s_v &= \frac{s_R}{\sqrt{R_1R_2}(1-R)^2}\\
    s_R &= \sqrt{(\partial_\alpha R s_\alpha)^2 + (\partial_nRs_n)^2}\\
    \partial_\alpha R &= \frac{\frac{(1-\frac{\cos\alpha}{\sqrt{n^2-\sin^2\alpha}})\tan(\alpha+\arcsin(\frac{\sin\alpha}{n}))} {\cos^2(\alpha-\arcsin(\frac{\sin\alpha}{n}))} - \frac{(1+\frac{\cos\alpha}{\sqrt{n^2-\sin^2\alpha}})\tan(\alpha-\arcsin(\frac{\sin\alpha}{n}))} {\cos^2(\alpha+\arcsin(\frac{\sin\alpha}{n}))}} {\tan^2(\alpha - \arcsin(\frac{\sin\alpha}{n}))}\\
    \partial_nR &= \frac{\sin\alpha(\frac{\tan(\alpha+\arcsin(\frac{\sin\alpha}{n}))}{\cos^2(\alpha-\arcsin(\frac{\sin\alpha}{n}))} - \frac{\tan(\alpha-\arcsin(\frac{\sin\alpha}{n}))}{\cos^2(\alpha+\arcsin(\frac{\sin\alpha}{n}))})}{\tan^2(\alpha + \arcsin(\frac{\sin\alpha}{n}))}\\
\end{align*}
Mittelt man über beide Verstärkungsfaktoren, so erhält man einen Verstärkungsfaktor von 
\begin{equation*}
    \textcolor{red}{v = 1,019 \pm 0,005}.
\end{equation*}
Dies entspricht den Erwartungen, dass im Leerlauf der Verstärkungsfaktor ungefähr 1 ist.\\
Verwendet man nun nicht die Näherung, dass das Glas absorptionsfrei ist, so berechnet sich $T$ über die Airy-Formel zu 
\begin{equation*}
    T = \frac{1}{1+F\sin^2(\frac{2\pi d}{\lambda}\sqrt{n^2-\sin^2\alpha})}
\end{equation*}
mit den Abkürzungen aus Gl.\ref{eq:airyPar}.\\
Kennt man nun die Dicke $d$ und nimmt den Winkel $\alpha$ als gegeben an, kann man den Fehler von o.B.d.A. $v_1$ wie folgt berechnen ($v_2$ funktioniert analog und liefert keinen Mehrwert):
\begin{equation*}
    s_{v_1} = \sqrt{(\partial_dvs_d)^2 + (\partial_nvs_n)^2},
\end{equation*}
wobei sich die partiellen Ableitungen ergeben zu:
\begin{align*}
    \partial_dv &= \frac{F}{\sqrt{R_1R_2}}\sin(\frac{4\pi d}{\lambda}\sqrt{n^2-sin^2\alpha})\frac{2\pi}{\lambda}\sqrt{n^2-\sin^2\alpha}\\
    \partial_nv &= \frac{1}{\sqrt{R_1R_2}} \Bigg[ F \sin(\frac{4\pi d}{\lambda}\sqrt{n^2-sin^2\alpha})\frac{2nd\pi}{\lambda \sqrt{n^2-\sin^2\alpha}}\\
    &+ \sin^2\biggl(\frac{2\pi d}{\lambda}\sqrt{n^2-sin^2\alpha} \biggl ) \frac{4(1-R)^2+8(1-R)R}{(1-R)^4} \frac{\sin\alpha}{n\sqrt{n^2-\sin^2\alpha}\tan^2(\alpha+ \arcsin(\frac{\sin\alpha}{n}))}\\
    &\cdot \biggl (\frac{\tan(\alpha - \arcsin(\frac{\sin\alpha}{n}))}{\cos^2(\alpha + \arcsin(\frac{\sin\alpha}{n}))} - \frac{\tan(\alpha+ \arcsin(\frac{\sin\alpha}{n}))}{\cos^2(\alpha - \arcsin(\frac{\sin\alpha}{n}))} \biggl ) \Bigg]
\end{align*}
Ziel ist es, den Fehler für die Dicke des Plättchens $s_d$ so zu bestimmen, dass die selbe Ungenauigkeit wie bei der Drehspiegelmethode erzielt wird. Aus der Gleichsetzung der 
Fehler $s_{v_1}$ folgt dann 
\begin{equation*}
    s_d = \sqrt{s_{v_1}^2 - (\partial_nvs_n)^2}(\partial_dv)^{-1}.
\end{equation*}
Für $d = 150\,\mu$m (aus Versuchsskript), $\lambda = 632,8\,$nm und den bisher berechneten Werten 
kommt es zu keinem sinnvollen Ergebnis, da schon alleine der Fehleranteil des Brechungsindices bei der Bestimmung über die Plättchendicke größer ist als der 
Gesamtfehler der Drehspiegelmethode. Nimmt man nun auch $n$ als fehlerfrei an, so ergibt sich 
\begin{equation*}
    s_d = \frac{s_{v_1}}{\partial_dv} = 2,4 \mathrm{nm},
\end{equation*}
was ein sehr niedriger Wert ist, der mit einfachen Messmethoden nicht zu erreichen ist. Demnach ist entweder der Vergleich der Fehler selbst fehlerhaft oder die Drehspiegelmethode 
liefert wesentlich präzisere Ergebnisse.\\
Die Plättchendicke lässt sich über die Interferenzbedingung zusammen mit Gl.\ref{eq:airyPar} bestimmen:
\begin{equation}
    \Delta \phi = \frac{4\pi}{\lambda}d\sqrt{n^2-\sin^2\alpha} \overset{!}{=} 2\pi m 
    \label{eq:intbed}
\end{equation}
Hierbei ist $m$ eine ganze Zahl. Liest man nun die Maxima bei $m$ und $m+i$ aus, so ergibt sich $d$ aus Gl.\ref{eq:intbed} zu 
\begin{equation*}
    d = \frac{i\lambda}{2(\sqrt{n^2-\sin^2\alpha_{m+i}} - \sqrt{n^2-\sin^2\alpha_m})}.
\end{equation*}
Mit $\alpha_m = (39,0 \pm 0,6)^\circ$ und $\alpha_{m+79} = (68,6 \pm 0,5)^\circ$ folgt 
\begin{equation*}
    \textcolor{red}{d = (131 \pm 5)\,\mu\mathrm{m}}.
\end{equation*}
Für die Fehlerrechnung wurden folgende partielle Ableitungen verwendet:
\begin{align*}
    \partial_nd &= \frac{-in\lambda}{2}\big(\sqrt{n^2-\sin^2\alpha_{m+i}}- \sqrt{n^2-\sin^2\alpha_{m}}\big)^{-2}\big(\frac{1}{n^2-\sin^2\alpha_{m+i}} - \frac{1}{n^2-\sin^2\alpha_{m}}\big)\\
    \partial_{\alpha_{m/m+i}}d &= \frac{i\lambda}{2}\big(\sqrt{n^2-\sin^2\alpha_{m+i}}- \sqrt{n^2-\sin^2\alpha_{m}}\big)^{-2}\bigg(\frac{\mp \sin(2\alpha_{m/m+i})}{2\sqrt{n^2-\sin^2\alpha_{m/m+i}}}\bigg)\\
\end{align*}
Die Dicke stimmt in der Größenordnung mit der Angabe von 150\,$\mu$m im Versuchsskript überein, jedoch nicht innerhalb des berechneten Fehlers. Entweder wurde dieser zu klein 
abgeschätzt oder, was unserer Meinung nach wahrscheinlicher ist, in der Angabe im Skript ist im Wort 'circa' eine größere Abweichung enthalten.
\section{Axiale Lasermoden}

Den Abstand der Lasermoden bestimmen wir einfach durch Ablesen am Oszilloskop. Zuvor müssen wir aber herausfinden, wie das Zeitsignal auf der 
x-Achse des Oszilloskops mit der Frequenz zusammenhängt.\\

Dazu nehmen wir zweimal den selben Peak, aber in zwei nebeneinanderliegenden Darstellungen auf Channel 1 in Abbildung \ref{bild:FreierSpektralbereich}.
Dieser Abstand entspricht dem freien Spektralbereich des Interferometers. Dieser ist bei dem hier verwendeten Gerät 2\,GHz. Man könnte ebenfalls das Triggersignal verwenden, 
aber an den Peaks kann man das Maximum leichter ablesen. Man erhält also den Umrechnungsfaktor 

\begin{equation*}
    m = (152 \pm 7)\,\frac{\mathrm{MHz}}{\mathrm{ms}}
\end{equation*}

für die Umrechnung 

\begin{equation}
    \Delta \nu = m\cdot \Delta t
    \label{eq:Umrechnung}
\end{equation}

von der vom Oszilloskop ausgegebenen Zeitdifferenz in Frequenzen $\nu$.


\begin{figure}[ht]
    \centering
    \includegraphics[width = \linewidth]{Bilder/Auswertung/FabryPerotKalibr.png}
    \caption{Longitudinale Lasermoden mit dem Fabry-Pérot aufgenommen. Channel 1 ist das Messsignal und Channel 2 ist des Triggersignal des Interfermoters. Gekennzeichnet sind 
    mit den x-Marker zwei gleich Peaks.}
    \label{bild:FreierSpektralbereich}
\end{figure}


\subsection*{Abstand und Linienbreite Axialer Lasermoden}

Den Abstand der Moden bestimmt man auch grafisch aus Abbildung \ref{bild:AxialModenAbstand}. Aus diesem erhält man 
\begin{equation*}
    \Delta t = (1,660 \pm 0,086)\,\mathrm{ms}
\end{equation*}
und mit der Umrechnung aus Gleichung \ref{eq:Umrechnung} erhalt man 

\begin{equation}
    \textcolor{red}{\delta \nu = (252 \pm 17)\,\mathrm{MHz}}
    \label{eq:Modenabstand}
\end{equation}

den Abstand zweier longitudinaler Lasermoden. Die Unsicherheit wird hierbei aus geschätzter Ableseunsicherheit und
der Unsicherheit der Kalibrierung mittels Fehlerfortpflanzung berechnet. Auf selbe Weise wird 

\begin{equation}
    \textcolor{red}{\Delta \nu_{multi} = (13,3 \pm 1,5)\,\mathrm{MHz}}
\end{equation}

auch die Linienbreite (FWHM) aus Abbildung \ref{bild:Lininebreite} bestimmt. Gleiches tun wir auch für die einzelne Mode
aus Abbildung \ref{bild:LininebreiteSingle}. Damit erhalten wir die Werte, welche in Tabelle \ref{tab:Linienbreite} dargestellt sind.

\begin{table}[ht]
    \centering
 
    \begin{tabular}{lcr}
        \toprule
        Messgröße & Symbol & Wert in MHz\\
        \midrule
        Freier spektraler Bereich& $\Delta \nu_{FSB}$ & 2000\\
        Modenabstand& $\delta\nu$& $252 \pm 17$\\
        FWHM (multi-mode)& $\Delta\nu_{multi}$&$13,3 \pm 1,5$\\
        FWHM (single)& $\Delta\nu_{single}$&$16,7 \pm 2,6$\\
        \bottomrule        
    \end{tabular}
  
    \caption{Freier spektraler Bereich, Modenabstand, Halbwertsbreite des He-Ne-Laser und dem dazugehörendem 
    konfokalem Fabry-Pérot-Interferometer}
    \label{tab:Linienbreite}
\end{table}

Der Modenabstand kann auch theoretisch mit Gleichung \ref{eq:FSR} berechnet werden. Dabei erhalten wir mit dem einer Resonatorlänge $L = 540 \pm 5 \, \mathrm{mm}$ und 
einem angenommen Brechungsindex $n = 1$ einen Modenabstand von $\delta \nu _{theo} = 138 \pm 2\, \mathrm{MHz}$. Dabei ist das offensichtlich nicht wahr. Die konfokalen Spiegel erzeigen anscheinend eine ganz normale stehende Welle, welche nach zweimaligem Durchlaufen
des Resonators wieder ihren Anfangspunkt trifft. Damit erhält man 
\begin{equation}
    \textcolor{red}{\delta \nu_{theo} = 277 \pm 3\, \mathrm{MHz}}
    \label{eq:theFSR}
\end{equation}
als theoretischen Wert. Dieser stimmt im Rahmen des Fehlers nicht ganz mit unsere Werten überein, die Größenordnung ist jedoch richtig. Die Unterschiede lassen sich dadurch erklären, dass 
der Brechungsindex im Resonator nicht genau 1 ist. Dies verfälscht den theoretischen Wert.



\subsection*{Verstärkungsprofil}

Die verursachten Erschütterungen führten zu Längenänderungen im Resonator, sodass die Peaks der Moden gewackelt haben.
Somit kann man, wenn man mit den Linien wackelt, ein ungefähres Bild bekommen, wie das Verstärkungsprofil
des Verstärkers aussieht. Das Ergebnis ist in Abbildung \ref{bild:Verstaerkung} sichtbar.

\begin{figure}[h]
    \centering
    \includegraphics[width = \linewidth]{Bilder/Auswertung/FabryPerotVerst.png}
    \caption{Longitudinale Lasermoden mit dem Fabry-Pérot aufgenommen. Durch Erschütterung wurde das Verstärkungsprofil sichtbar gemacht.}
    \label{bild:Verstaerkung}
\end{figure}

Von diesem schätzen wir die Halbwertsbreite ab, soweit möglich. Die Halbwertsbreite 

\begin{equation}
    \textcolor{red}{\Delta\nu_{Verstaerkungsprofil} = 532 \pm 80 \, \mathrm{MHz}}
\end{equation}

hat einen relativ großen Fehler, da man die Spitze des Verstärkungsprofils nicht genau sehen kann und daher abschätzen muss.
Dies ist auch sinnvoll, da die Breite des Verstärkungsprofils mehrmals den Modenabstand beinhalten sollte.


\subsection*{Finesse und Auflösungsvermögen}

Die Finesse und das Auflösungsvermögen sind zwei Größen, die zur Charakterisierung des Resonators nützlich sind. Die Finesse ist dabei
definiert über das Verhältnis 
\begin{align}
    \mathcal{F} = \frac{\delta \nu_{FSR}}{\Delta\nu} \qquad s_{\mathcal{F}} = \sqrt{(\frac{s_{\Delta \nu}*\delta \nu_{FSR}}{\Delta\nu^2})^2+(\frac{s_{\delta\nu}}{\Delta \nu})^2}
\end{align}
 des freien Spektralbereichs zur Linienbreite. Die Auflösung 

 \begin{align}
     A = \frac{\nu}{\Delta\nu} = \frac{c}{\lambda\Delta\nu} \qquad s_A = A\sqrt{(\frac{s_{\Delta\nu}}{\Delta\nu})^2} = A\frac{s_{\Delta\nu}}{\Delta\nu}
 \end{align}
 
 ist hingegen das Verhältnis der Linienbreite zur Frequenz der Spektrallinie. Dabei wurde die Frequenz durch
 die Lichtgeschwindigkeit $c$ und die angegeben Wellenlänge $\lambda$ des Lasers (632,8\,nm) berechnet. 
 
 \begin{gather}
    \textcolor{red}{\mathcal{F} = 119\pm19}\\
    \textcolor{red}{A = (28,3\pm4,4)\cdot10^6}
 \end{gather}


 \subsection*{Mischfrequenzen}

 Die Bestimmug der Mischfrequenzen dient der Verifizierung des Frequenzabstands 
 Dabei können wir leider nicht das elektrische Feld anhand seiner Intensität direkt messen, da dieses zu hochfrequent ist. Was wir aber messen können, sind 
 die niederfrequenten Anteile, welche sich aus den Mischtermen ergeben. Dabei erwarten wir in etwa den Modenabstand zu erhalten. 
 Bei der Aufnahme ist es relativ schwierig einen Wert abzulesen, da wir zwei kleine Peaks nebeneinander sehen. Die Mischfrequenz 
 \begin{equation*}
     \textcolor{red}{\Delta \nu_{Misch1} = (281,93\pm0,40)\,\mathrm{MHz} }
 \end{equation*}

 hat trotzdem noch deutlich weniger Unsicherheit als der vorher bestimmte Wert. Auch dieser liegt nahe am theoretischen Wert aus Gleichung \ref{eq:theFSR} und dem Wert des 
 Modenabstandes aus Tabelle \ref{tab:Linienbreite}. Die Messunsicherheiten der drei Messungen überlappen leider nicht. Das ist ein Hinweis darauf, dass es noch Fehlerquellen im Hintergrund gibt, welche wir nicht
 berücksichtigt haben.

 \subsection*{Laser als Längenmessgerät}

 Durch das Einbringen des Glasplättchen verlängern wir den optischen Weg im Resonator. Dabei müssen wir 
 festhalten, dass es sich bei unserem Resonator um einen konfokalen Resonator handelt. Es sieht im Spektrum so aus, 
 als hätten wir den Resonator verlängert. Diese Verlängerung 

 \begin{equation*}
     \Delta L = d*n - d
 \end{equation*}

 berechnet sich aus der Dicke $d$ des Plättchens und dem Brechungsindex $n$ des Glasplättchens. Allgemein ist der 
 Modenabstand 
 
 \begin{equation*}
     \Delta\nu = \frac{c}{2L\cdot n}
 \end{equation*}

 abhängig von der Länge des Resonators. Bei einem konfokalem Laser ist er sogar abhängig von 

 \begin{equation*}
    \Delta\nu = \frac{c}{4L\cdot n},
\end{equation*}

da der Laserstrahl vier Mal L zurücklegen muss um mit sich selbst zu interagieren\footnote{\url{https://www.uni-muenster.de/Physik.AP/Denz/Studieren/Lehrveranstaltungen/photonik_ws1213_laser.html}, Eingesehen: 07.10.2021}
Damit ergibt sich bei für die Dicke 

\begin{equation}
    d = \frac{c}{4(n-1)}\cdot \biggl( \frac{1}{\Delta\nu_{Misch2}}-\frac{1}{\Delta\nu_{Misch2}} \biggl)
\end{equation}


des Plättchens, wobei Mischfrequenz 2 $\Delta \nu_{Misch2} = (281,90\pm0,40)\,\mathrm{MHz}$ die Frequenz des verlängerten Resonators ist. \\
Mit dem Brechungsindex von 1,46 aus vorherigem Versuchsteil erhält man 

\begin{equation}
    \textcolor{red}{d = (0,06 \pm 0,90)\,\mathrm{mm,}}
\end{equation}

was aber im Rahmen des Fehlers ein realistischer Wert ist. Der Fehler ist aber vergleichsweise groß. Das liegt unter anderem daran, dass die
Abstände der Mischfrequenzen viel kleiner als ihre Fehler sind. Man sieht, dass es sich die Methode mäßig zum bestimmen von Längen eignet, wenn es sich um so kleine Strecken handelt.  Bei größeren Längenunterschieden sollte die Methode jedoch funktionieren. 
% Gaussstrahlen

\section{Gaußstrahlen}
\label{sec:gauss}

\subsection{Strahlausbreitungsparameter $M^2$}

\subsubsection*{ohne Fernfeldnäherung}

Für die Breite eines idealen Gaußstrahls gilt laut Skript 
\begin{equation*}
    w_G(z) = w_{0,G} \sqrt{1 + \biggl(\frac{\lambda z}{\pi w_{0,G}^2}\biggl)^2}.
\end{equation*}

Damit ergibt sich für den Stahldurchmesser 
\begin{equation*}
    d_G(z) = 2 w_G(z) = 2 w_{0,G} \sqrt{1 + \biggl(\frac{\lambda z}{\pi w_{0,G}^2}\biggl)^2} = d_{0,G} \sqrt{1 + \biggl(\frac{4\lambda z}{\pi d_{0,G}^2}\biggl)^2}.
\end{equation*}

Weicht der Strahl nun von einem idealen Gaußstrahl ab, so wird sein Durchmesser um den Faktor $M$ größer. Er berechnet sich nach
\begin{equation*}
    d(z) = M d_{0,G} \sqrt{1 + \biggl(\frac{4\lambda z}{\pi d_{0,G}^2}\biggl)^2} = d_0 \sqrt{1 +\biggl (\frac{4M^2\lambda z}{\pi d_0^2}\biggl)^2},
\end{equation*}
wobei $d_0$ hier der Durchmesser der Strahltallie des nicht-idealen Gaußstrahls ist.
Für einen Abstand $z = -(L_2-L_1)/2$ ergibt sich also
\begin{align}
    d_1 &= d(-(L_2-L_1)/2) = d_0 \sqrt{1 +\biggl (\frac{-(L_2-L_1)2M^2\lambda}{\pi d_0^2}\biggl)^2} \nonumber \\
    \leftrightarrow M^2 &= \frac{\pi d_0^2}{2 \lambda (L_2-L_1) } \sqrt{\biggl(\frac{d_1}{d_0}\biggl)^2-1},
    \label{eq:M}
\end{align}
was die Formel zur Berechnung des Strahlausbreitungsparameters aus dem Skript ist.

Im Folgenden soll mithilfe dieser Formel der Strahlausbreitungsparameter für den Experimentier- und den Hilfslaser bestimmt werden. Dazu wurde die Intensität der Strahlen 
nach dem Durchgang durch eine Linse (f = 300\,$\frac{1}{m}$) an verschiedenen Positionen mit einer CCD-Kamera gemessen. Dabei wurden die Daten entlang einer horizontalen und 
einer vertikalen Achse durch den Bereich des Strahlquerschnitts mit der höchsten Intensität entnommen. Nachdem eine gaußförmige Verteilung der Intensität erwartet wird, wird 
eine Funktion der Form 
\begin{equation*}
    I(x) = I_0 \exp(\frac{-2(x-x_0)^2}{w^2}) + I_{off}
\end{equation*}

an die Datensätze gefittet. Der Parameter $I_{off}$ ist durch den Offset in den gemessenen Intensitäten der Kamera bedingt. Aus diesen Fits ergeben sich die Breiten und Fehler der 
Strahlen $w$, die in Tab.\ref{tab:M} bis \ref{tab:M3} zu sehen sind. Für Berechnungen, die die Strahldurchmesser $d$ benötigen, müssen die Werte von $w$ und $s_w$ einfach verdoppelt werden.
%TABELLE

\begin{table}
    \centering
    \begin{tabular}{rrrrrr}
        \toprule
        $L$ / mm &  $s_L$ / mm &    $w$ / mm &  $s_w$ / mm&     $M^2$ &        $s_M^2$ \\
        \midrule
        770 &  10 &  0,05928 &  0,00042 &        &        \\
        780 &  10 &  0,06022 &  0,00028 &  0,312 &  0,449 \\
        760 &  10 &  0,07365 &  0,00070 &  1,286 &  1,819 \\
        790 &  10 &  0,08248 &  0,00043 &  0,844 &  0,596 \\
        750 &  10 &  0,09498 &  0,00084 &  1,092 &  0,772 \\
        800 &  10 &  0,11303 &  0,00047 &  0,944 &  0,445 \\
        740 &  10 &  0,12093 &  0,00045 &  1,034 &  0,487 \\
        730 &  10 &  0,14284 &  0,00062 &  0,956 &  0,338 \\
        810 &  10 &  0,15814 &  0,00067 &  1,078 &  0,381 \\
        720 &  10 &  0,17334 &  0,00058 &  0,958 &  0,271 \\
        820 &  10 &  0,19323 &  0,00112 &  1,082 &  0,306 \\
        700 &  10 &  0,22388 &  0,00110 &  0,907 &  0,183 \\
        840 &  10 &  0,25699 &  0,00111 &  1,051 &  0,212 \\
        680 &  10 &  0,31518 &  0,00152 &  1,012 &  0,159 \\
        860 &  10 &  0,33974 &  0,00162 &  1,093 &  0,172\\
        660 &  10 &  0,36696 &  0,00205 &  0,968 &  0,124 \\
        880 &  10 &  0,38290 &  0,00232 &  1,012 &  0,130 \\
        600 &  10 &  0,55240 &  0,00250 &  0,950 &  0,079 \\
        960 &  10 &  0,61462 &  0,00377 &  0,947 &  0,071 \\
        500 &  10 &  0,91721 &  0,01036 &  0,997 &  0,053 \\
        \bottomrule
    \end{tabular}
    \captionof{table}{Werte der verschiedenen Strahlbreiten $w$, Strahlausbreitungsparameter $M^2$ und zugehörige Fehler für verschiedene Positionen $L$ 
    des horizontalen Schnitts durch den Strahl des Experimentierlasers. Die Werte sind aufsteigend nach $w$ sortiert. Der Wert für $L$ = 770\,mm wurde als Strahltaille benutzt; deshalb existieren 
    hier keine Werte für $M^2$ und dessen Fehler.}
    \label{tab:M}
\end{table}

\begin{table}
    \centering
    \begin{tabular}{rrrrrr}
        \toprule
        $L$ / mm&  $s_L$ / mm &      $w$ / mm&    $s_w$ / mm&         $M^2$ &        $s_M^2$ \\
        \midrule
        780 &  10 &  0,0609 &  0,0002 &       &       \\
        770 &  10 &  0,0614 &  0,0004 &  0,22 &  0,33 \\
        760 &  10 &  0,0775 &  0,0007 &  0,72 &  0,51 \\
        790 &  10 &  0,0800 &  0,0004 &  1,57 &  2,22 \\
        750 &  10 &  0,0992 &  0,0008 &  0,79 &  0,37 \\
        800 &  10 &  0,1076 &  0,0004 &  1,34 &  0,94 \\
        740 &  10 &  0,1288 &  0,0004 &  0,85 &  0,30 \\
        730 &  10 &  0,1524 &  0,0006 &  0,84 &  0,23 \\
        810 &  10 &  0,1548 &  0,0005 &  1,43 &  0,67 \\
        720 &  10 &  0,1825 &  0,0006 &  0,86 &  0,20 \\
        820 &  10 &  0,1885 &  0,0009 &  1,35 &  0,47 \\
        700 &  10 &  0,2301 &  0,0012 &  0,83 &  0,14 \\
        840 &  10 &  0,2503 &  0,0010 &  1,22 &  0,28 \\
        860 &  10 &  0,3015 &  0,0013 &  1,11 &  0,19 \\
        680 &  10 &  0,3178 &  0,0013 &  0,94 &  0,13 \\
        880 &  10 &  0,3737 &  0,0019 &  1,11 &  0,15 \\
        660 &  10 &  0,3869 &  0,0025 &  0,96 &  0,11 \\
        600 &  10 &  0,5885 &  0,0040 &  0,98 &  0,07 \\
        960 &  10 &  0,6066 &  0,0049 &  1,01 &  0,08 \\
        500 &  10 &  0,9058 &  0,0105 &  0,97 &  0,05 \\
        \bottomrule
    \end{tabular}
    \captionof{table}{Werte der verschiedenen Strahlbreiten $w$, Strahlausbreitungsparameter $M^2$ und zugehörige Fehler für verschiedene Positionen $L$ 
    des vertikalen Schnitts durch den Strahl des Experimentierlasers. Die Werte sind aufsteigend nach $w$ sortiert. Der Wert für $L$ = 780\,mm wurde als Strahltaille benutzt; deshalb existieren 
    hier keine Werte für $M^2$ und dessen Fehler.}
    \label{tab:M1}
\end{table}

\begin{table}
    \centering
    \begin{tabular}{rrrrrr}
        \toprule
        $L$ / mm &  $s_L$ / mm &    $w$ / mm &  $s_w$ / mm&     $M^2$ &        $s_M^2$ \\
        \midrule
        830 &  10 &  0,0583 &  0,0006 &       &      \\
        820 &  10 &  0,0619 &  0,0005 &  0,60 &  0,85 \\
        810 &  10 &  0,0710 &  0,0006 &  0,58 &  0,41 \\
        840 &  10 &  0,1022 &  0,0010 &  2,43 &  3,43 \\
        800 &  10 &  0,1047 &  0,0010 &  0,83 &  0,39 \\
        850 &  10 &  0,1320 &  0,0012 &  1,71 &  1,21 \\
        790 &  10 &  0,1428 &  0,0011 &  0,94 &  0,33 \\
        780 &  10 &  0,1747 &  0,0010 &  0,95 &  0,26 \\
        860 &  10 &  0,1780 &  0,0015 &  1,62 &  0,76 \\
        870 &  10 &  0,2169 &  0,0016 &  1,51 &  0,53 \\
        770 &  10 &  0,2235 &  0,0015 &  1,04 &  0,24 \\
        760 &  10 &  0,2646 &  0,0015 &  1,06 &  0,21 \\
        890 &  10 &  0,3102 &  0,0025 &  1,47 &  0,34 \\
        740 &  10 &  0,3410 &  0,0028 &  1,08 &  0,17 \\
        910 &  10 &  0,3875 &  0,0030 &  1,38 &  0,24 \\
        720 &  10 &  0,4217 &  0,0025 &  1,09 &  0,14 \\
        930 &  10 &  0,4578 &  0,0034 &  1,31 &  0,18 \\
        700 &  10 &  0,5104 &  0,0038 &  1,12 &  0,12 \\
        950 &  10 &  0,5229 &  0,0056 &  1,25 &  0,14 \\
        650 &  10 &  0,6915 &  0,0074 &  1,10 &  0,08 \\
        550 &  10 &  1,1528 &  0,0174 &  1,19 &  0,06 \\
        \bottomrule
    \end{tabular}
    \captionof{table}{Werte der verschiedenen Strahlbreiten $w$, Strahlausbreitungsparameter $M^2$ und zugehörige Fehler für verschiedene Positionen $L$ 
    des horizontalen Schnitts durch den Strahl des Hilfslasers. Die Werte sind aufsteigend nach $w$ sortiert. Der Wert für $L$ = 770\,mm wurde als Strahltaille benutzt; deshalb existieren 
    hier keine Werte für $M^2$ und dessen Fehler.}
    \label{tab:M2}
\end{table}

\begin{table}
    \centering
    \begin{tabular}{rrrrrr}
        \toprule
        $L$ / mm &  $s_L$ / mm &    $w$ / mm &  $s_w$ / mm&     $M^2$ &        $s_M^2$ \\
        \midrule
        820 &  10 &  0,0622 &  0,00047 &       &       \\
        810 &  10 &  0,0663 &  0,0003 &  0,70 &  1,00 \\
        830 &  10 &  0,0686 &  0,0006 &  0,89 &  1,26 \\
        800 &  10 &  0,0816 &  0,0008 &  0,81 &  0,57 \\
        840 &  10 &  0,0982 &  0,0008 &  1,17 &  0,83 \\
        790 &  10 &  0,1106 &  0,0008 &  0,94 &  0,44 \\
        850 &  10 &  0,1301 &  0,0009 &  1,17 &  0,55 \\
        780 &  10 &  0,1494 &  0,0008 &  1,04 &  0,37 \\
        860 &  10 &  0,1699 &  0,0008 &  1,22 &  0,43 \\
        770 &  10 &  0,1771 &  0,0011 &  1,02 &  0,29 \\
        870 &  10 &  0,2021 &  0,0009 &  1,18 &  0,33 \\
        760 &  10 &  0,2078 &  0,0010 &  1,02 &  0,24 \\
        740 &  10 &  0,2639 &  0,0009 &  0,99 &  0,17 \\
        890 &  10 &  0,2657 &  0,0010 &  1,14 &  0,23 \\
        720 &  10 &  0,3357 &  0,0016 &  1,01 &  0,14 \\
        910 &  10 &  0,3361 &  0,0015 &  1,13 &  0,17 \\
        700 &  10 &  0,4155 &  0,0027 &  1,05 &  0,12 \\
        930 &  10 &  0,4250 &  0,0024 &  1,18 &  0,15 \\
        950 &  10 &  0,4556 &  0,0022 &  1,07 &  0,11 \\
        650 &  10 &  0,5865 &  0,0037 &  1,06 &  0,08 \\
        550 &  10 &  0,9381 &  0,0128 &  1,07 &  0,05 \\
        \bottomrule
    \end{tabular}
    \captionof{table}{Werte der verschiedenen Strahlbreiten $w$, Strahlausbreitungsparameter $M^2$ und zugehörige Fehler für verschiedene Positionen $L$ 
    des vertikalen Schnitts durch den Strahl des Hilfslasers. Die Werte sind aufsteigend nach $w$ sortiert. Der Wert für $L$ = 770\,mm wurde als Strahltaille benutzt; deshalb existieren 
    hier keine Werte für $M^2$ und dessen Fehler.}
    \label{tab:M3}
\end{table}

Zur Berechnung des Strahlausbreitungsparameters wird Gl.\ref{eq:M} umgeschrieben zu
\begin{equation*}
    M^2 = \frac{\pi d_0^2}{2 \lambda 2|L_0-L_1| } \sqrt{\biggl(\frac{d_1}{d_0}\biggl)^2-1},
\end{equation*}
wobei verwendet wurde, dass $d_1$ und $d_2$ symmetrisch um $d_0$ liegen.\\
Für $d_0$ (und $L_0$) werden jeweils die kleinsten Werte der Strahldurchmesser der jeweiligen Messreihen gewählt und $M^2$ für alle anderen Werte von $d$ (und $L$) als 
$d_1$ ($L_1$) berechnet. Der Fehler für $M^2$ ergibt sich aus der klassischen Fehlerfortpflanzung unter der Verwendung folgender partieller Ableitungen:
\begin{align*}
    \partial_{d_0}M^2 &= \frac{M^2}{d_0}(2-\frac{1}{1-(\frac{d_0}{d_1})^2}) \\
    \partial_{d_1}M^2 &= \frac{M^2}{d_1 - \frac{d_0^2}{d_1}} \\
    \partial_{L_{0/1}}M^2 &= \pm \frac{M^2}{L_0-L_1} \\
\end{align*}
Dabei wurde für die Längen ein Fehler von 10\,mm angenommen, da die Position der Kamera auf ihrem Schlitten nicht eindeutig festzustellen war. 
Die berechneten $M^2$ mit zugehörigen Fehlern $s_{M^2}$ finden sich ebenfalls in Tab.\ref{tab:M} bis \ref{tab:M3}.
Es ist zu sehen, dass nicht alle Werte von $M^2$ sinnvoll sind. Generell sollten keine Werte kleiner als 1 zu finden sein; davon gibt es unseren Berechnungen nach aber einige. 
Gerade an den Punkten nah an der angenommenen Stahltaille, vor allem beim Experimentierlaser, ist die Abweichung enorm. Dies könnte auch daran liegen, dass die Taille nicht genau getroffen wurde. Um eine Übersicht 
über den Strahlausbreitungsparameter zu erhalten, wird für jede Messreihe der Mittelwert und der dazugehörige Fehler gebildet, wobei die grob abweichenden Werte nah der Taille für den 
Experimentierlaser verworfen werden. Die Ergebnisse sind 
\begin{equation*}
    \textcolor{red}{M^2_{v,exp} = (1,05 \pm 0,15)\,\mathrm{mm}} , \,\, \textcolor{red}{ M^2_{h,exp} = (1,01 \pm 0,14)\,\mathrm{mm}},\\
\end{equation*}
\begin{equation*}
    \textcolor{red}{M^2_{v,hilf} = (1,05 \pm 0,14)\,\mathrm{mm}}\,\,\mathrm{und}\,\,\textcolor{red}{ M^2_{h,hilf} = (1,22 \pm 0,16)\,\mathrm{mm}}.\\    
\end{equation*}
Alle Mittelwerte sind immerhin größer als 1, wenn auch wohl betragsmäßig deutlich zu nah. 
Die Fehler würden zwar geringfügig höhere Werte zulassen; dennoch sind die Werte unseres Erachtens aufgrund methodischer Mängel mit Vorischt zu genießen. Zumindest die 
Größenordnung ist allerdings sinnvoll.

\subsubsection*{Fernfeldmethode}
Eine weitere Möglichkeit zur Bestimmung von $M^2$ ist die Fernfeldmethode. Dabei wird ausgenutzt, dass für $z>5z_R$ (mit Rayleighbereich $z_R$) der Strahlradius sich asymptotisch 
einem Grenzwinkel $\theta_0$ annähert; der Graph des Strahlradius aufgetragen gegen die Postion wird also eine Gerade. 
Die Bereiche, ab denen die Fernfeldnäherung zulässig ist, $5z_R$ ergeben sich nach 
\begin{equation*}
    5z_R = 5 \frac{\pi w_0^2}{\lambda}.
\end{equation*}
Für $w_0$ wurden die selben Taillen wie in der Berechnung ohne Fernfeldnäherung benutzt (vgl. Tab.\ref{tab:M} bis \ref{tab:M3}). Die berechneten Bereiche sind  
\begin{equation*}
    5z_{R,v} = 93,59\,\mathrm{mm} \quad \mathrm{ und } \quad 5z_{R,h} = 87,24\,\mathrm{mm}.
\end{equation*}
In den entsprechenden Bereichen werden nun Geraden gefittet, wie in den Abb.\ref{pic:gerfitve} bis \ref{pic:gerfithh} zu sehen. Aus der Steigung der Geraden ergibt sich $\theta_0$ 
aus 
\begin{equation*}
    \theta_0 = |\arctan(m)|,
\end{equation*}
mit zugehörigem Fehler
\begin{equation*}
    s_{\theta_0} = \frac{1}{1+m^2}s_m,
\end{equation*}
wobei $s_m$ der Fehler aus der linearen Regression ist.
Der Strahlausbreitungsparameter ergibt sich dann aus
\begin{equation*}
    M^2 = \frac{w_0\theta_0 \pi }{\lambda}.
\end{equation*}
Der zughörige Fehler ergibt sich aus der Fehlerfortpflanzung. Die berechneten Werte für $M^2$ und seine Fehler finden sich in Tab.\ref{tab:Mfern}, wobei für jede Messreihe 
zwei Werte berechnet wurden, da die Öffnungswinkel links und rechts der Strahltaille gesondert betrachtet wurden. %tabelle
\begin{table}
    \centering
    \begin{tabular}{rll}
        \toprule
        Wertebereich & Laser & $M^2$ / mm \\
        \midrule
        vertikaler Schnitt, links von Strahltaille & Exp & 0,99 $\pm$ 0,02 \\
        vertikaler Schnitt, rechts von Strahltaille & Exp & 0,887 $\pm$ 0,006 \\
        horizontaler Schnitt, links von Strahltaille & Exp & 0,99 $\pm$ 0,04 \\
        horizontaler Schnitt, rechts von Strahltaille & Exp & 0,82 $\pm$ 0,04 \\
        vertikaler Schnitt, links von Strahltaille & Hilf & 1,088 $\pm$ 0,019 \\
        vertikaler Schnitt, rechts von Strahltaille & Hilf & 0,473 $\pm$ 0,003 \\
        horizontaler Schnitt, links von Strahltaille & Hilf & 1,31 $\pm$ 0,06 \\
        horizontaler Schnitt, rechts von Strahltaille & Hilf & 1,04 $\pm$ 0,03 \\
        \bottomrule
    \end{tabular}
    \captionof{table}{Berechnete Werte für $M^2$ nach der Fernfeldmethode. 'Exp' steht für den Experimentierlaser, 'Hilf' für den Hilfslaser. Dabei sind für den Experimentierlaser sämtliche Werte kleiner als 1, was nicht sein sollte. Der Fehler für den Wert des 
    vertikalen Schnitts rechts von der Strahltaille ist bei beiden Lasern sehr klein; dies liegt daran, dass die Regression selber keinen Fehler hat, da nur zwei Punkte verwendet wurden. Das Ergebnis für den Hilfslaser weicht zudem stark ab, 
    ist deshalb nicht ganz aussagekräftig und wird im weiteren Verlauf auch ausgesondert werden.}
    \label{tab:Mfern}
\end{table}
Dabei ist auch hier festzustellen, dass die Werte zwar in der richtigen Größenordnung liegen. Jedoch sind die Werte, vor allem die des Experimentierlasers deutlich zu klein. 
Die Werte für den Hilfslaser sind etwas besser, abgesehen von dem Wert für den vertikalen Schnitt, rechts von der Taille; dieser wurde allerdings nur über zwei Punkte berechnet, 
die auch anhand der Abb.\ref{pic:gerfitve} bis \ref{pic:gerfithh} als nicht repräsentativ eingeschätzt werden und der Wert für $M^2$ somit übergangen werden kann. Die übrigen Werte scheinen auch 
etwas niedrig, da sie aber größer als 1 sind, sind sie zumindest nicht eindeuig als falsch zu identifizeren. 

\begin{figure}[h]
    \centering
    \includegraphics[scale = 0.75]{Bilder/Auswertung/gerfitv.png}
    \caption{Strahldurchmesser $w$ aufgetragen in blauen Punkten gegen die Position $L$ für den vertikalen Schnitt durch den Strahl. Dabei ist der lineare Verlauf für Werte weit entgernt von der Strahltaille 
    gut zu erkennen; die zugehörige lineare Regression ist in Rot eingetragen. Die Regression zur Rechten basiert nur auf zwei Punkten, passt aber dem Augenschein nach trotzdem in etwa zum Verlauf und wird deshalb nicht verworfen.}
    \label{pic:gerfitve}
\end{figure}

\begin{figure}[h]
    \centering
    \includegraphics[scale = 0.75]{Bilder/Auswertung/gerfith.png}
    \caption{Strahldurchmesser $w$ aufgetragen in blauen Punkten gegen die Position $L$ für den vertikalen Schnitt durch den Strahl. Dabei ist der lineare Verlauf für Werte weit entgernt von der Strahltaille 
    gut zu erkennen; die zugehörige lineare Regression ist in Rot eingetragen.}
    \label{pic:gerfithe}
\end{figure}

\begin{figure}[h]
    \centering
    \includegraphics[scale = 0.75]{Bilder/Auswertung/gerfitvh.png}
    \caption{Strahldurchmesser $w$ aufgetragen in blauen Punkten gegen die Position $L$ für den vertikalen Schnitt durch den Strahl. Dabei ist der lineare Verlauf für Werte weit entgernt von der Strahltaille 
    gut zu erkennen; die zugehörige lineare Regression ist in Rot eingetragen. Die Regression zur Rechten basiert nur auf zwei Punkten und ist deshalb von großer Unsicherheit. Da sie zudem augenscheinlich auch 
    nicht die Weiterführung des Graphen ist, kann der aus dieser Geraden resultierende Strahlausbreitungsparameter verworfen werden.}
    \label{pic:gerfitvh}
\end{figure}

\begin{figure}[h]
    \centering
    \includegraphics[scale = 0.75]{Bilder/Auswertung/gerfithh.png}
    \caption{Strahldurchmesser $w$ aufgetragen in blauen Punkten gegen die Position $L$ für den vertikalen Schnitt durch den Strahl. Dabei ist der lineare Verlauf für Werte weit entgernt von der Strahltaille 
    gut zu erkennen; die zugehörige lineare Regression ist in Rot eingetragen.}
    \label{pic:gerfithh}
\end{figure}
\clearpage
\subsection{Effektive Brennweite der Linse}
\label{subs:f}
Für Gaußsche Strahlen gelten zwar die Gesetze der geometrischen Optik nicht, es kann aber trotzdem eine effektive Brennweite $f_{eff}$ nach 
\begin{equation*}
    \frac{1}{s} + \frac{1}{s'} = \frac{1}{f_{eff}}
\end{equation*}
bestimmt werden. Nimmt man nun an, dass die Strahltaille des Experimentierlasers in der Mitte des Resonators liegt, so gilt für die Strecke zwischen Strahltaille und 
Linse $s$
\begin{align*}
    s &= d_{Linse-Spiegel} + d_{Spiegel-Spiegel} + d_{Spiegel-Resonator} + 0,5 \cdot L_{Resonator} = \\
    &= ((370 - 40) + 201 + (420-50) + 0,5 \cdot 540)\mathrm{mm}.
\end{align*}
Die verwendeten Längen wurden dem Protokoll entnommen. Bei einem Fehler von 5\,mm pro Länge ergibt dies nach Fehlerfortpflanzung einen Fehler von $s_m = 28$\,mm. 
Die Länge $s'$ ist abhängig von der Position der Strahltaille nach der Linse und variiert zwischen senkrechtem und waagerechtem Schnitt durch den Strahl. Für die 
Positionen und Fehler wurden die Werte aus Tab.\ref{tab:M} bis \ref{tab:M3} genommen. Damit ergeben sich
\begin{equation*}
    s'_v = (410 \pm 14)\mathrm{mm} \qquad s'_h = (400 \pm 14)\mathrm{mm}.
\end{equation*}
Aus diesen Werten lassen sich die Werte für $f_{eff}$ berechnen; die dazugehörigen Fehler folgen aus der Fehlerfortpflanzung, wobei folgende Beziehung verwendet wurde:
\begin{equation*}
    \partial_{s^{(|)}}f_{eff} = \frac{f_{eff}^2}{s^{(|)2}}
\end{equation*}
Die effektiven Brennweiten sind
\begin{equation*}
    f_{eff,v} = (303 \pm 13)\,\mathrm{mm}\qquad\mathrm{und }f_{eff,h}=(298 \pm 13)\,\mathrm{mm}
\end{equation*}
Bildet man den Mittelwert, so hat die Linse unseren Berechnungen nach eine effektive Brennweite von
\begin{equation*}
    \textcolor{red}{f_{eff} = (301 \pm 9)\, \mathrm{mm}}
\end{equation*}
Dieses Ergebnis stimmt auch gut mit der Angabe auf der Linse selbst überein; deshalb kann es als sinnvoll angesehen werden.

\subsection{Strahltaillen}
\subsubsection*{Im Resonator}
Die Stahltaille im Resonator ergibt sich nach 
\begin{equation*}
    w_{00} = \sqrt{\frac{\lambda L}{\pi}} \cdot \biggl(\frac{(1-g^2)}{4(1-g)^2}\biggl)^{0,25},
\end{equation*}
wobei $L$ die Resonatorlänge und $g = 1 - \frac{L}{R}$ der Spiegelparameter mit Spiegelkrümmungsradius $R$ ist. Ferner wurde benutzt, dass $R$ und somit $g$ für beide Spiegel 
gleich sind(\cite{TUB2018}, S.8). 
Für die Fehlerfortpflanzung wurden dabei die folgenden Ableitungen verwendet:
\begin{align*}
    \partial_Lw_{00}  &= \frac{w_{00}}{2L}\\
    \partial_gw_{00} &= w_{00} \frac{1 - g(1-g)}{2(1-g)(1-g^2)}
\end{align*}
Die Fehler für $L$ und damit für $g$ sind die aus der Messung der Resonatorlänge. Insgesamt ergibt sich die Strahltaille zu
\begin{equation*}
    \textcolor{red}{w_{00} = (224,0 \pm 1,5)\,\mu\mathrm{m}}
\end{equation*}

\subsubsection*{Hinter der Linse, berechnet}
Ausgehend von der berechneten Strahltaille im Resonator ergibt sich die Strahltaille hinter der Linse nach 
\begin{equation*}
    w_0 = w_{00}\,\sqrt{\frac{s'-f}{|s|-f}}. \mathrm{\footnotemark}
\end{equation*}
\footnotetext{\url{https://www.edmundoptics.de/knowledge-center/application-notes/lasers/gaussian-beam-propagation/}, Stand:08.10.21}
Dabei entsprechen die Werte für $s$, $s'$ und $f$ denen aus Abschnitt \ref{subs:f}. Für die Fehlerfortpflanzung wurden folgende partielle Ableitungen verwendet: 
\begin{align*}
    \partial_{w_{00}}w_0 &= \frac{w_0}{w_{00}}\\
    \partial_{s^{(|)}}w_{0} &= \mp \frac{w_0}{2(s^{(|)}-f)}
\end{align*}
Damit ergeben sich für die berechneten Strahltaillen für den vertikalen bzw. horizontalen Schnitt des Strahls die Werte 
\begin{equation*}
    \textcolor{red}{w_{0,v} = (78 \pm 7)\,\mu\mathrm{m} \qquad w_{0,h} = (77 \pm 7)\,\mu\mathrm{m}}.
\end{equation*}
Vergleicht man diese mit dem gemessenen Werten für $w_0$ in Tab.\ref{tab:M} und \ref{tab:M1} ($w_{0,v}$ = (60,9  $\pm$ 2)\,$\mu$m, $w_{0,h}$ = (59,3  $\pm$ 4)\,$\mu$m), so ist festzustellen, 
dass die Werte in der selben Größenordnung liegen. Allerdings sind die berechneten Werte um etwa den Faktor 1,3 zu groß und die Abweichung lässt sich auch nicht alleine durch 
die berechneten Fehler erklären. Möglicherweise liegt der Fehler in dem nicht idealen Versuchsaufbau oder es wurden in der Berechnung weitere Faktoren vernachlässigt (z.B. 
könnte die Strahltaille nicht exakt in der Mitte des Resonators liegen oder noch wahrscheinlicher die Strahltaille wurde nicht an der exakten Position hinter der Linse 
bestimmt).

\subsection{Intensität im Fokus}
Die Intensität eines Gaußstrahls lässt sich schreiben als 
\begin{equation*}
    %I = I_0 \exp(\frac{-2r^2}{w^2}), \mathrm{\footnotemark}
    I = \frac{2P}{\pi w^2} \exp(\frac{-2r^2}{w^2}), \mathrm{\footnotemark}
\end{equation*}
\footnotetext{\url{https://www.edmundoptics.de/knowledge-center/application-notes/lasers/gaussian-beam-propagation/}, Stand:08.10.21}
wobei $r$ der radiale Abstand zur Stahlmitte und $P$ die Leistung des Lasers ist. Die Leistung innerhalb einer Kreisscheibe mit Radius $R$ berechnet sich nach 
\begin{align*}
    P(R) &= \iint I(r) dA = \frac{2P}{\pi w^2} \int_0^{2\pi} \int_0^R \exp(\frac{-2r^2}{w^2})r\, d\phi dr = \\
    &= \frac{4P}{w^2} [-\frac{w^2}{4}\exp(\frac{-2r^2}{w^2})]_0^R \\
    &= P (1-\exp(\frac{-2R^2}{w^2}))\\
    &= P(1-e^{-2}),
\end{align*}
wobei im letzten Schritt für $R$ der Strahldurchmesser $w$ eingesetzt wurde. Im Fokus gilt $w = w_0$ damit folgt für die Leistung pro Fläche (=Intensität)
\begin{equation*}
    \frac{P}{A} = I_{fok} = \frac{P(1-e^{-2})}{w^2\pi} = \frac{P(1-e^{-2})}{w_0^2\pi}
\end{equation*}
Und mit $P =$1\,mW, der Ableitung für die Fehlerfortpflanzung 
\begin{equation*}
    \partial_{w_0}I = \frac{2I}{w_0}s_{w_0}
\end{equation*}
und dem Wert für die Strahltaille aus Tab.\ref{tab:M} ergibt sich ein Wert von 
\begin{equation*}
    \textcolor{red}{I_{fok} = (73,0 \pm 1,0)\,\frac{\mathrm{kW}}{\mathrm{m^2}}}
\end{equation*}
(für den horizontalen und vertikalen Schnitt des Strahls sind die Taillen fast identisch; zur Abschätzung der gesamten 
Strahltaille wurde der größere von beiden gewählt).

\section{Transversalmoden eines Lasers}
\label{section:transvM}

Ein Laser hat mehrere Moden. Normalerweise ist die $TEM_{00}$-Mode dominant, das heißt man sieht einen zusammenhängenden 
Punkt mit näherungsweise gaußförmigem Profil. Es gibt jedoch auch noch andere Moden, wie in dem Grundlagenkapitel \ref{subs:moden}
beschrieben. Diese haben wir versucht durch die eingebrachte Drahtblende zu erzeugen. Dabei haben wir die in Abbildung \ref{bild:Moden} gezeigten Moden beobachtet.
\begin{figure}[ht]
    \centering
    \subfloat[$TEM_{00}$]{\label{TEM00}%
      \includegraphics[width=0.235\textwidth]
      {Bilder/Auswertung/TEM00.png}}\quad
    \subfloat[$TEM_{10}$]{\label{TEM10}
      \includegraphics[width=0.25\textwidth]
      {Bilder/Auswertung/TEM10.png}}\quad
    \subfloat[$TEM_{01}$]{\label{TEM01}%
      \includegraphics[width=0.25\textwidth]
      {Bilder/Auswertung/TEM01.png}}\quad
    \subfloat[$TEM_{20}$]{\label{TEM20}%
      \includegraphics[width=0.25\textwidth]
      {Bilder/Auswertung/TEM20.png}}\quad
      \subfloat[$TEM_{30}$]{\label{TEM30}%
      \includegraphics[width=0.25\textwidth]
      {Bilder/Auswertung/TEM30.png}}\quad
      \subfloat[$TEM_{un.}$]{\label{TEMunz}%
      \includegraphics[width=0.25\textwidth]
      {Bilder/Auswertung/TEMunsugeordnet.png}}
      \subfloat[$TEM_{01*}$]{\label{TEM01*}\quad
      \includegraphics[width=0.245\textwidth]
      {Bilder/Auswertung/TEM11.png}}
      \caption{Transversalmoden der HeNe-Laser}
      \label{bild:Moden}
  \end{figure}
  Normalerweise sollten die Moden radialsymmetrisch sein. Diese kann man hier nur vereinzelt beobachten, da die Brewsterfenster die Radialsymmetrie aufheben.
  Spannenderweise kann in Abbildung \ref{TEM01*} auch eine radialsymmetrischen Mode beobachten werden. Außerdem
  gibt es eine Mode in \ref{TEMunz}, welche wir nicht identifizieren konnten. Diese könnte eine Mischmode sein. Abgesehen von den fotografierten Moden konnten wir auch eine 
  $TEM_{11}$-Mode beobachten, diese aber leider nicht aufnehmen.
  \clearpage
\section{Hologramm}

Ein Hologramm ist in gewissem Sinne die Weiterentwicklung des Photos. Während bei einem Schwarzweißfilm nur die Intensität 
des Wellenfeldes gemessen wird, misst ein Farbphoto schon die Wellenlänge des verwendeten Lichtes mit. Das Hologramm 
beinhaltet nun außerdem Informationen über die Phase des Lichtes. Damit lässt sich ein dreidimensionales Abbild des 
Objektes rekonstruieren. Dabei wird das Hologramm mit einer bestimmten Quelle, der Referenzquelle, aufgenommen. Will man diese nun wiedergeben, benötigt man 
die Referenzquelle oder eine gleichartige Quelle. Wir haben hier ein Hologramm untersucht. Dieses zeigt einen Schlumpf 
beim Fußballspielen, was man in Abbildung \ref{bild:Holo} bewundern kann. 

\begin{figure}[ht]
    \centering
    \includegraphics[width = 12cm]{Bilder/Auswertung/Holo.png}
    \caption{Hologramm mit einem HeNe-Hilfslaser aufgenommen}
    \label{bild:Holo}
\end{figure}

Dabei ist das Hologramm leider nicht optimal zu sehen, da wir nur den Hilfslaser verwenden konnten. Trotzdem sieht
man auch in Abbildung \ref{bild:Holo} die dreidimensionale Darstellung des Bildes.


% etc.

    % 5.Kapitel Fazit
    %Matteo Kumar - Leonard Schatt
% Fortgeschrittenes Physikalisches Praktikum

% 5. Kapitel Einleitung

\chapter{Fazit}
\label{chap:fazit}
Der Versuch hat auf jeden Fall das Verständnis von Alpha- und Gammaspektroskopie vertieft. Dabei hat man den Umgang mit radioaktiven 
Präparaten und deren Messmethoden gelernt. Insbesondere konnten wir bestimmte Isotope in unserer Umwelt nachweisen. Außerdem haben wir die Kurzreichweitigkeit von 
Alphastrahlung und die Abschirmbarkeit von Gammastrahlung verifiziert. Damit können wir eine bessere Einschätzung von Gefahrenpotentialen 
bei zukünftigen Versuchen vornehmen. 
% Platz für Text



    % Anhang
    %% Matteo Kumar - Leonard Schatt
% Physikalisches Praktikum

% Anhang

\appendix

% Text

% Matteo Kumar - Leonard Schatt
% Physikalisches Praktikum

% Anhang A

\chapter{Anhang}
\label{chap:anhangA}
\section{Methodik}
\subsection{Versuchsaufbauten}
\subsubsection{Aufbau spektrale Empfindlichkeit}

\begin{figure}[h]
    \centering
    \includegraphics[width = \linewidth]{Bilder/Aufbau6.jpg}
    \caption{Xenon-Lampe mit vorgelagertem Filter}
\end{figure}

\begin{figure}[h]
    \centering
    \includegraphics[width = 8cm]{Bilder/Aufbau3.jpg}
    \caption{Verwendete Messgeräte (in schwarz der Lock-in Verstärker)}
\end{figure}

\begin{figure}[h]
    \centering
    \includegraphics[width = 8cm]{Bilder/Aufbau4.jpg}
    \caption{Innenansicht des Gitterspektrometers}
\end{figure}
\clearpage


\subsubsection{Versuchsteil U-I-Kennlinien}

\begin{figure}[h]
    \centering
    \includegraphics[width = \linewidth]{Bilder/Aufbau1.jpg}
    \caption{Baustrahler mit vorgelagerten Glasplatten}
\end{figure}


\clearpage

\section{Fitten der Shockley-Gleichung}
\label{section:AnhangShock}

\begin{figure}[ht]
    \centering
    \includegraphics[width = \linewidth]{Bilder/CIS180Plot.pdf}
    \caption{Gefittete Shockley-Gleichung an das CIS-Modul bei 180V Trafospannung}    
\end{figure}

\begin{figure}[ht]
    \centering
    \includegraphics[width = \linewidth]{Bilder/CIS230Plot.pdf}
    \caption{Gefittete Shockley-Gleichung an das CIS-Modul bei 230V Trafospannung}
\end{figure}

\begin{figure}[ht]
    \centering
    \includegraphics[width = \linewidth]{Bilder/CISDunkelPlot.pdf}
    \caption{Gefittete Schockley-Gleichung an Dunkelmessung der CIS-Zelle}
\end{figure}


\begin{figure}[ht]
    \centering
    \includegraphics[width = \linewidth]{Bilder/SiMonoDunkelPlot.pdf}
    \caption{Gefittete Schockley-Gleichung an das Mono-Si-Modul bei 130V}
\end{figure}
\begin{figure}[ht]
    \centering
    \includegraphics[width = \linewidth]{Bilder/SiMulti130Plot.pdf}
    \caption{Gefittete Schockley-Gleichung an das Mono-Si-Modul bei 130V}
\end{figure}
\begin{figure}[ht]
    \centering
    \includegraphics[width = \linewidth]{Bilder/SiMulti180Plot.pdf}
    \caption{Gefittete Schockley-Gleichung an das Mono-Si-Modul bei 130V}
\end{figure}
\begin{figure}[ht]
    \centering
    \includegraphics[width = \linewidth]{Bilder/SiMulti230Plot.pdf}
    \caption{Gefittete Schockley-Gleichung an das Mono-Si-Modul bei 130V}
\end{figure}
\begin{figure}[ht]
    \centering
    \includegraphics[width = \linewidth]{Bilder/SiMultiDunkelPlot.pdf}
    \caption{Gefittete Schockley-Gleichung an das Mono-Si-Modul bei 130V}
\end{figure}
\clearpage


\section{Wirkungsgrad}
\label{section:AnhangWirkungsgrad}

\begin{figure}[ht]
    \centering
    \includegraphics[width = \linewidth]{Bilder/UOCInt.png}
    \caption{Leerlaufspannung in Abhängigkeit der Lichtintensität bei drei unterschiedlichen Solarmodule. Die Fehler 
    sind aus Schwankungen geschätzt}
\end{figure}

\begin{figure}[ht]
    \centering
    \includegraphics[width = \linewidth]{Bilder/IPhInt.png}
    \caption{Photostrom in Abhängigkeit der Lichtintensität bei drei unterschiedlichen Solarmodule. Die Fehler 
    sind aus Schwankungen geschätzt}

\end{figure}

\begin{figure}[ht]
    \centering
    \includegraphics[width = \linewidth]{Bilder/MPPInt.png}
    \caption{Maximale Leistung $P_{Max}$ in Abhängigkeit der Lichtintensität bei drei unterschiedlichen Solarmodule. Die Fehler 
    sind aus Schwankungen geschätzt}  
\end{figure}

\begin{figure}[ht]
    \centering
    \includegraphics[width = \linewidth]{Bilder/FFInt.png}
    \caption{Füllfaktor $FF$ in Abhängigkeit der Lichtintensität bei drei unterschiedlichen Solarmodule. Die Fehler 
    sind aus Schwankungen geschätzt}  
\end{figure}

\clearpage
\subsection{Messdaten}
\begin{figure}[h]
    \captionsetup{justification=centering,margin=2cm}
    \centering
    \includegraphics[angle = 90, width = 12cm]{Bilder/Daten/MessunngMonoSI.png}
    \caption{Messdaten zur Messung der spektralen Empfindlichkeit der MonoSi-Zelle}
\end{figure}


\begin{figure}[h]
    \captionsetup{justification=centering,margin=2cm}
    \centering
    \includegraphics[angle = 90, width = 12cm]{Bilder/Daten/MessunngMonoSiDunkel.png}
    \caption{Dunkelmessung der MonoSi-Zelle}
\end{figure}

\begin{figure}[h]
    \captionsetup{justification=centering,margin=2cm}
    \centering
    \includegraphics[angle = 90, width = 12cm]{Bilder/Daten/MessunngMonoSi130.png}
    \caption{Beleuchtete Messung bei 130V am Stelltransformator}
\end{figure}

\begin{figure}[h]
    \captionsetup{justification=centering,margin=2cm}
    \centering
    \includegraphics[angle = 90, width = 12cm]{Bilder/Daten/MessunngMonoSi180.png}
    \caption{Beleuchtete Messung bei 180V am Stelltransformator}
\end{figure}

\begin{figure}[h]
    \captionsetup{justification=centering,margin=2cm}
    \centering
    \includegraphics[angle = 90, width = 12cm]{Bilder/Daten/MessungMonoSi230.png}
    \caption{Beleuchtete Messung bei 230V am Stelltransformator}
\end{figure}






\begin{figure}[h]
    \captionsetup{justification=centering,margin=2cm}
    \centering
    \includegraphics[angle = 90, width = 12cm]{Bilder/Daten/MessunngMultiSi.png}
    \caption{Messdaten zur Messung der spektralen Empfindlichkeit der MultiSi-Zelle}
\end{figure}


\begin{figure}[h]
    \captionsetup{justification=centering,margin=2cm}
    \centering
    \includegraphics[angle = 90, width = 12cm]{Bilder/Daten/MessunngMonoSiDunkel.png}
    \caption{Dunkelmessung der MultiSi-Zelle}
\end{figure}

\begin{figure}[h]
    \captionsetup{justification=centering,margin=2cm}
    \centering
    \includegraphics[angle = 90, width = 12cm]{Bilder/Daten/MessunngMulriSi130.png}
    \caption{Beleuchtete Messung bei 130V am Stelltransformator an der Multi-Si-Zelle}
\end{figure}

\begin{figure}[h]
    \captionsetup{justification=centering,margin=2cm}
    \centering
    \includegraphics[angle = 90, width = 12cm]{Bilder/Daten/MessungMultiSi180.png}
    \caption{Beleuchtete Messung bei 180V am Stelltransformator an der Multi-Si-Zelle}
\end{figure}

\begin{figure}[h]
    \captionsetup{justification=centering,margin=2cm}
    \centering
    \includegraphics[angle = 90, width = 12cm]{Bilder/Daten/MessungMultiSi230.png}
    \caption{Beleuchtete Messung bei 230V am Stelltransformator an der Multi-Si-Zelle}
\end{figure}






\begin{figure}[h]
    \captionsetup{justification=centering,margin=2cm}
    \centering
    \includegraphics[angle = 90, width = 12cm]{Bilder/Daten/MessunngCIS.png}
    \caption{Messdaten zur Messung der spektralen Empfindlichkeit an der CIS-Zelle}
\end{figure}


\begin{figure}[h]
    \captionsetup{justification=centering,margin=2cm}
    \centering
    \includegraphics[angle = 90, width = 12cm]{Bilder/Daten/MessunngCISDunkel.png}
    \caption{Dunkelmessung an der CIS-Zelle}
\end{figure}

\begin{figure}[h]
    \captionsetup{justification=centering,margin=2cm}
    \centering
    \includegraphics[angle = 90, width = 12cm]{Bilder/Daten/MessunngCIS130.png}
    \caption{Beleuchtete Messung bei 130V am Stelltransformator an der CIS-Zelle}
\end{figure}


\begin{figure}[h]
    \captionsetup{justification=centering,margin=2cm}
    \centering
    \includegraphics[angle = 90, width = 12cm]{Bilder/Daten/MessunngMonoSi180.png}
    \caption{Beleuchtete Messung bei 180V am Stelltransformator an der CIS-Zelle}
\end{figure}


\begin{figure}[h]
    \captionsetup{justification=centering,margin=2cm}
    \centering
    \includegraphics[angle = 90, width = 12cm]{Bilder/Daten/MessunngCIS230.png}
    \caption{Beleuchtete Messung bei 230V am Stelltransformator an der CIS-Zelle}
\end{figure}



    % Literatur

    \bibliographystyle{Auswertung.bst}
    \bibliography{Auswertung.bib}

    %Abbildungsverzeichnis
    \listoffigures

\end{document}