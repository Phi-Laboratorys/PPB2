% Autor: Manuel Lippert
% Physikalisches Praktikum

% Main-Datei für die Auswertung in TeX

% Struktur:
% Jedes Kapitel hat einen Input-File. Um Merge-Konflikte zu verhindern wird angeraten für jede 
% Datei eine eigene Tex Datei zu machen und sie im jeweiligen Kapitel zu importieren. Die in
% Input-Struktur dient zur besseren Übersicht und für mögliche Ordner, welche hier vorhanden sind. Die Zahlen vor den 
% Ordner dient zur Ordnung der einzelnen tex-Files nach Kapiteln


% Packages
\documentclass[paper=a4,bibliography=totoc,BCOR=10mm,twoside,numbers=noenddot,fontsize=11pt]{scrreprt}
\usepackage[english]{babel}
\usepackage[T1]{fontenc}
\usepackage[latin1, utf8]{inputenc} %ä, ö, ü inbegriffen
\usepackage[babel,german=quotes]{csquotes} %For Quotes
\usepackage{lmodern}
\usepackage{graphicx}
\usepackage{nicefrac}
\usepackage{fancyvrb}
\usepackage{amsmath,amssymb,amstext}
\usepackage{siunitx}
\usepackage{url}
\usepackage{natbib}
\usepackage{microtype}
\usepackage[format=plain]{caption}
\usepackage{physics}
\usepackage{titleref} 

% Zusätzliche Packages
\usepackage{geometry} % Verändert Seitengeometrie
\usepackage{anyfontsize} % Alle Schriftgrößen möglich machen
\usepackage[table]{xcolor} % Farbliche Gestaltung Tabellen
\usepackage{ifthen} % Für kompliziertere tex-Files
\usepackage[absolute,overlay]{textpos} %Textboxen
\usepackage{amsfonts} % Schriftarten
\usepackage{xstring} % Stringoperationen
\usepackage{tikz} % Zeichnungen
\usepackage{pdfpages} % Import von pdfs (Protokolle)
\usepackage{hyperref} % Verlinkungen im Dokument
\usepackage{makecell} % Zeilenumbruch in Zelle

% Abschnittseinrückung und -abstand
% Die folgenden Zeilen sollen möglichst nicht verändert werden
\parindent 0.0cm
\parskip 0.8ex plus 0.5ex minus 0.5ex

% Anzahl und Größe von Gleitobjekten
% maximal 2 Objekte oben und unten
% erlaubt auch größere Bilder, welche die ganze Seite benötigen
% Die folgenden Zeilen sollen möglichst nicht verändert werden
\setcounter{bottomnumber}{2}
\setcounter{topnumber}{2}
\renewcommand{\bottomfraction}{1.}
\renewcommand{\topfraction}{1.}
\renewcommand{\textfraction}{0.}

%\sc und \bc veraltet. Daher: (20.09.2018)
\DeclareOldFontCommand{\sc}{\normalfont\scshape}{\@nomath\sc}
\DeclareOldFontCommand{\bf}{\normalfont\scshape}{\textbf}

% Verschiedenes
\pagestyle{headings}          % Der Seitenstil sollte möglichst nicht verändert werden
\graphicspath{{./Bilder/}}    % Der Pfad für die Abbildungen Abbildungen wird gesetzt
\VerbatimFootnotes            % \verb etc. auch in \footnotes mφglich

% Funktionen
\newcommand\tab[1][1cm]{\hspace*{#1}}
\newcommand{\vect}[1]{\boldsymbol{\mathbf{#1}}}
\newcolumntype{g}{>{\columncolor[rgb]{ .741,  .843,  .933}}l}
% Tiefgestellte Zahlen nicht kursiv
\catcode`_=\active
\newcommand_[1]{\ensuremath{\sb{\mathrm{#1}}}}

\begin{document}

    \nonfrenchspacing

    % 0.Chapter Cover
    % 0. Cover

% Hier sind nur die Variablen und der Abschnitt Informationen (unten) zu bearbeiten der REst läuft automatisch ab (z.b Farbenänderung)

% Noch abänderbar nur ein Vorschlag
\newgeometry{top=30mm, bottom=20mm, inner=20mm, outer=20mm}
\thispagestyle{empty}

% Colors (Notability Colors)
\definecolor{Notablue}{HTML}{3498DB}		
\definecolor{Notared}{HTML}{CF366C}			
\definecolor{Notagreen}{HTML}{19B092}		
\definecolor{Notaorange}{HTML}{FA9D00}		
\definecolor{Notagrey}{HTML}{969696}		
\definecolor{Notalavendel}{HTML}{9DBBD8}	

% Boolean by default false. Für Absatz in der Überschrift
\newboolean{twoRows}
\newboolean{symbol}

% Funktions
\makeatletter
   \def\vhrulefill#1{\leavevmode\leaders\hrule\@height#1\hfill \kern\z@}
\makeatother
\newcommand*\ruleline[1]{\par\noindent\raisebox{.8ex}{\makebox[\linewidth]{\vhrulefill{\linethickness}\hspace{1ex}\raisebox{-.8ex}{#1}\hspace{1ex}\vhrulefill{\linethickness}}}}

% Variables
\def\schriftgrosse{50}
\def\linethickness{1,5pt}

\def\farbe{black}
\def\fach{PPBphys2}
\def\name{Manuel Lippert - Paul Schwanitz}
\def\titel{Rasterelektronen- \\[0,5cm] mikroskop} % Absatz mit \\[0,5cm]; u = Übung, k = Klausur; s = Skript, e = Ergebnis
\def\bottom{WS2021/22}
\def\datum{13.09.2021}
\def\platz{NWII | 2.1.00.267}
\def\betreuer{Inga Elvers}

\def\teilnehmerm{Manuel Lippert}
\def\emailm{Manuel.Lippert@uni-bayreuth.de}
\def\teilnehmerp{Paul Schwanitz}
\def\emailp{Paul.Schwanitz@uni-bayreuth.de}

%\def\auswertp{}
%\def\messp{}
%\def\protop{}

\def\groupnr{11}

\begin{titlepage}
			
	\centering
	{\LARGE \sffamily {\textbf{\bottom}\par}}
	\vspace{2,5cm}
    {\fontsize{30}{0}\sffamily\ruleline{\textcolor{\farbe}{\textbf{\fach}}}\par}
    \vspace{6cm}
	{\Large\sffamily \ruleline{\name}\par}
		
	\IfSubStr {\titel} {\\[0,5cm]} {\setboolean{twoRows}{true}} {\setboolean{twoRows}{false}}
	
	\ifthenelse{\boolean{twoRows}}
		{
			\begin{textblock*}{21cm}(0cm,8,5cm) % {block width} (coords), centered		
				{\fontsize{\schriftgrosse}{0}\sffamily\textcolor{\farbe}{\textbf{\titel}}\par}
			\end{textblock*}
		}
		{
			\begin{textblock*}{21cm}(0cm,9cm) % {block width} (coords), centered		
				{\fontsize{\schriftgrosse}{0}\sffamily\textcolor{\farbe}{\textbf{\titel}}\par}
			\end{textblock*} 
		}
	
	% Choose Logo
	\ifthenelse {\equal{\farbe}{Notared}} {\def\logo{Bilder/Logo/UniBTNotared}}
		{\ifthenelse {\equal{\farbe}{Notagreen}} {\def\logo{Bilder/Logo/UniBTNotagreen}}
			{\ifthenelse {\equal{\farbe}{Notablue}} {\def\logo{Bilder/Logo/UniBTNotablue}}
				{\ifthenelse {\equal{\farbe}{Notaorange}} {\def\logo{Bilder/Logo/UniBTNotaorange}}
					{\ifthenelse {\equal{\farbe}{Notagrey}} {\def\logo{Bilder/Logo/UniBTNotagrey}}
						{\ifthenelse {\equal{\farbe}{Notalavendel}} {\def\logo{Bilder/Logo/UniBTNotalavendel}}	
							{\ifthenelse {\equal{\farbe}{black}} {\def\logo{Bilder/Logo/UniBT}}	
								{\def\logo{noLogo}}
							}
						}
					}
				}
			}
		}	

	\IfSubStr{\logo}{noLogo}{\setboolean{symbol}{false}}{\setboolean{symbol}{true}}
	
	% Gruppe
	\vspace{10cm}
	{\large\sffamily{Gruppe \groupnr}}
	
	%Logo
	\vfill

	\ifthenelse{\boolean{symbol}}
		{
			\begin{figure}[h]
			\begin{center}
				
				\includegraphics[width=2cm]{\logo}
				
			\end{center}
			\end{figure}
		}
	
\end{titlepage}

\restoregeometry

% Information
\chapter*{Informationen}
\label{chap:info}

\begin{tabular}{l l}

	{\textbf{Versuchstag}} \hspace{1cm} & \hspace{1cm} {\datum}\\[0,2cm]
	{\textbf{Versuchsplatz}} \hspace{1cm} & \hspace{1cm} {\platz}\\[0,2cm]
	{\textbf{Betreuer}} \hspace{1cm} & \hspace{1cm} {\betreuer}\\[1,2cm]
	{\textbf{Gruppen Nr.}} \hspace{1cm} & \hspace{1cm} {\groupnr}\\[0.2cm]
	% Für Fortgeschittenenen Praktikum
	{\textbf{Teilnehmer}} \hspace{1cm} & \hspace{1cm} {\teilnehmerm~(\emailm)}\\[0.2cm]
						  \hspace{1cm} & \hspace{1cm} {\teilnehmerp~(\emailp)}\\[0.2cm]
	% Für Grundpraktikum
	%{\textbf{Auswertperson}} \hspace{1cm} & \hspace{1cm} {\auswertp}\\[0.2cm]
	%{\textbf{Messperson}} \hspace{1cm} & \hspace{1cm} {\messp}\\[0.2cm]
	%{\textbf{Protokollperson}} \hspace{1cm} & \hspace{1cm} {\protop}\\[0.2cm]

\end{tabular}

    \thispagestyle{empty}
    \cleardoublepage
    \tableofcontents
    \cleardoublepage

    % 1.Chapter Instructions
    % 1. Introduction

\chapter{Introduction}
\label{chap:intro}

Spectroscopy is a very important tool to discover the structure from atoms and molecules. With the help of that tool, the physicists discover the spectral lines of the H-atom. This is where spectroscopy began. \\
The research in this are also helps to discover things like fine and hyperfine structure. What finaly leads to Quantum mechanics.\\
Furthermore, spectroscopy is also very useful tool for other areas of science like chemistry or biology. In this areas spectroscopy helps to analyze probes and provides information about which materials the probe is made of.\\
So the aim of this Experiment is to get in contact with experimental atom and nucleus physics by means of Doppler-free  Saturation Spectroscopy of Rubidium.

    % 2.Chapter Theorie
    % 2. Fragen zur Vorbereitung

\chapter{Theoretischer Hintergrund}
\label{chap:theo}

% Text

% Input der Teilaufgaben je nach Produktion der Nebendateien ohne Ordner

\section{Rauschen}

Dieser Abschnitt soll einen groben Überblick über die verschiedenen Arten von Rauschen und ihre Ursachen liefern.

\subsection{Thermisches Rauschen}
Das thermische Rauschen wird von den statistischen Bewegungen der freien Ladungsträger, meist Elektronen, verursacht. Das thermische Rauschen ist weiterhin von der Frequenz unabhängig, weshalb es oft als weißes Rauschen bezeichnet wird.
Die Rauschspannung $u_R$, an einem Widerstand $R$, kann mit folgender Formel berechnet werden: 
\begin{align}
    u_R = \sqrt{4kTRB} \qquad \text{mit} \quad B = f_{max} - f_{min}
    \label{eq:thR}
\end{align}
Wobei T die absolute Temperatur ist und k die Boltzmannkonstante. B bezeichnet die Bandbreite des Messgerätes. Da es sich um statistische Schwankungen handelt, ergibt eine Mittelung von $u_R$ über die Zeit Null.\\
Aus obiger Formel \ref{eq:thR} kann durch die Division durch R eine Formel für die Rauschleistung hergeleitet werden.
\begin{align}
    P_R = 4kTB
\end{align}
Es geht klar hervor, dass die Rauschleistung nur von Temperatur und Frequenz abhängt. Somit wäre es also theoretisch möglich das thermische Rauschen, durch Kühlung des Versuchsaufbaus auf den absoluten Nullpunkt, abzustellen. Dies ist jedoch nicht praktikabel, da es mit enormen Kosten und Aufwand verbunden wäre \citep{VA}.

\subsection{Schrotrauschen}
Wie das thermische Rauschen ist auch das Schrotrauschen ein statisches und von der Frequenz unabhängiges Rauschen, weshalb es ebenfalls ein weißes Rauschen ist. Die Ursache ist jedoch die Quantelung der elektrischen Ladung, welche sich ebenfalls stochastisch bewegen und somit das Schrotrauschen verursachen. Der Effektivwert des Rauschstroms $i_R$ kann durch folgende Formel ausgedrückt werden:
\begin{align}
    i_R^2 = 2eIB
\end{align}
Wobei B wieder die Bandbreite des Messgerätes ist, e die Elektronenladung und I der fließende Gleichstrom.
Für die Rauschleistung $P_R$, über einen Übergang mit Widerstand $R$, gilt: 
\begin{align}
    P_R = i_R^2 R = 2eIRB
\end{align}
Aus dieser Formel ist ableitbar, dass das Schrotrauschen durch die Erniedrigung des Gleichstroms, der durch den Übergang fließt, erreicht werden kann \citep{VA}.

\subsection{Funkelrauschen}
Durch Störstellen im Material kann es zu Funkelrauschen kommen, welches mit zunehmender Frequenz abnimmt. Deshalb wird es auch als $\frac{1}{f}$ Rauschen bezeichnet. Abhilfe schafft hier, die Messungen bei hohen Frequenzen durchzuführen \citep{VA}.

\newpage
\section{Umwelteinflüsse}
\label{sec:umwelt}
Häufig ist bei Messungen jedoch nicht nur Rauschen ein Problem, sondern störende Umwelteinflüsse. Unsere Umwelt ist voll von Störquellen wie elektromagnetischen Wellen, beispielsweise von Radiosendern, welche die Messungen verfälschen, da Kabel im Versuchsaufbau für diese als Antenne fungieren können. Eines der stärksten Störeinflüsse ist wohl das Netzbrummen, was durch die öffentliche Stromversorgung bei 50 Hz verursacht wird. Ebenso können diverse elektrische Geräte wie Elektromotoren oder alte Röhrenmonitore Störungen verursachen \citep{VA}.\\

Um die Einstrahlung von störenden elektromagnetischen Wellen zu vermeiden ist eine gute Abschirmung von diesen vonnöten.
Darum werden die Messgeräte abgeschirmt und Koaxialkabel verwendet. Ein Koaxialkabel besteht aus einem Draht, welche von einer Isolierschicht umgeben ist, welche wiederum durch ein Drahtgeflecht umgeben ist. Als letztes umgibt das Kabel noch eine weitere Isolierung. Der Aufbau wird in Abbildung \ref{fig:Koaxialkabel} veranschaulicht. Das Drahtgeflecht dient hierbei als Schirm und verhindert somit die Einstrahlung von Störeinflüssen. Des Weiteren dient der Schirm auch dazu, um ein gemeinsames Erdpotential bereitzustellen, was eine Störung durch Erdschleifen verhindert. Erdschleifen werden im nächsten Kapitel genauer erklärt.

\begin{figure}[h]
    \centering
    \includegraphics[width=\textwidth]{KoaxialKabel.jpeg}
    \caption{Aufbau eines Koaxialkabels und Schaltsymbol (unten rechts) \citep{VA}}
    \label{fig:Koaxialkabel}
\end{figure}

\newpage
\section{Erdschleifen}
Im vorherigen Abschnitt wurde bereits der Begriff Erdschleifen genannt, welcher nun genauer erklärt werden soll. Wenn die Abschirmungen verschiedener Apparaturen nicht miteinander verbunden sind, sondern jede einzeln geerdet ist, dann existieren dennoch kleine Potenzialunterschiede, die elektrostatisch in das System eingekoppelt werden und somit Störungen verursachen. Um Erdschleifen zu vermeiden, sollten alle Abschirmungen verbunden sein und an einem einzigen Erdungspunkt geerdet werden \citep{VA}. In Abbildung \ref{fig:Erdschleife} wird dies schematisch dargestellt.
\begin{figure}[h]
    \centering
    \begin{subfigure}{0.45\textwidth}
        \centering
        \includegraphics[width=\textwidth]{Erdschleife.jpeg}
        \caption{Falsche Erdung \citep{VA}}
    \end{subfigure}
    \hfill 
    \begin{subfigure}{0.45\textwidth}
        \centering
        \includegraphics[width=\textwidth]{keineErdschleife.jpeg}
        \caption{Richtige Erdung \citep{VA}}
    \end{subfigure}
    \caption{Beispiele für falsche und richtige Erdung}
    \label{fig:Erdschleife}
\end{figure}
% Teilaufgabe 11
\newpage
\section{Möglichkeiten der Signal-Rausch Verbesserung}
\label{sec:verbesserung}
Dieser Abschnitt behandelt die Methoden zur Signal/Rausch-Verbesserung und dessen Umsetzung.
\subsection{Filter}
\label{sub:filter}
Eine Möglichkeit zur Signal/Rausch-Verbesserung ist der Einsatz eines Filters. Die Aufgabe des Filters ist hierbei die Unterdrückung bestimmter Frequenzen. Dabei ist zu beachten, dass das Signal bei einer anderen Frequenz auftritt als das Rauschen, sonst würden man nämlich das Signal mit filtern. Bei der Filterung wird die Bandbreite das Rauschen stark reduziert (Zu sehen an Gleichung (2.1) und (2.3)), aber auch Störstrahlen aus der Umgebung werden minimiert \citep{VA}.
\subsection*{Filtertypen}
\begin{itemize}
    \item[1)]\textbf{Tiefpass}\\
    Ein Tiefpassfilter filtert Frequenzen \textbf{oberhalb} der Grenzfrequenz heraus und lassen Frequenzen unterhalb nahezu ungedämpft durch. Die Grenzfrequenz ist dadurch charakterisiert, dass das Ausgangssignal zu dieser Frequenz um 3dB kleiner ist (ab dort beginnt der Durchlassbereich) \citep{electronik}.
    \item[2)]\textbf{Hochpass}\\
    Ein Hochpassfilter filtert Frequenzen \textbf{unterhalb} der Grenzfrequenz heraus und lassen Frequenzen oberhalb nahezu ungedämpft durch. Ein Hochpassfilter ist das Gegenstück zum Tiefpassfilter \citep{electronik}. Ein Hochpassfilter kann dazu verwendet werden, die 50Hz Brummspannung aus dem Messsignal zu filtern \citep{VA}.
    \item[3)]\textbf{Bandpass}\\
    Ein Bandpass sperrt Frequenzen \textbf{unter- und oberhalb} eines definierten Frequenzbandes. Das Frequenzband ist durch die 3dB-Bandbreite um die Mittenfrequenz charakterisiert. Diese Art des Filters ist eine Reihenschaltung aus Tiefpass- und Hochpassfilter und die Bandbreite wird durch die Grenzfrequenzen der jeweiligen Filter festgelegt \citep{electronik}. Der Bandpass findet heirbei Einsatz bei der Filterung von breitbandigen Rauschen \citep{VA}.
    \item[4)]\textbf{Bandsperre/Notch-Filter}\\
    Ein Bandsperre sperrt einen schmalen Frequenzbereich innerhalb eines breiten Frequenzbandes und kann wie das Gegenstück zu einem Bandpass angesehen werden \citep{electronik}.
\end{itemize}
In Abbildung \ref{image:verlauf} ist der generelle Verlauf der jeweiligen Filtertypen im Frequenzbereich dargestellt.
\newpage
\begin{center}
    \includegraphics[scale=0.3]{VerlaufFilter.png}
    \captionof{figure}{Verlauf der jeweiligen Filtertypen im Frequenzbereich \citep{grafikQuelle}}
    \label{image:verlauf}
\end{center}
\subsection*{Ordnung eines Filters}
Die Ordnung eines Filters gibt an wie oft ein Filter hintereinander in Reihe geschaltet wurde. Somit wären zwei Tiefpassfilter in Reihe geschaltet ein Tiefpassfilter 2.Ordnung. Je höher die Ordnung eines Filters ist, desto steiler ist die sogenannte Flankensteilheit (Steigung der Flanken in Abbildung \ref{image:verlauf}). Es ist aber zu beachten, dass sich durch die Ordnung die Phase in der Nähe der Grenzfrequenz verändern kann \citep{electronik}.
\subsection*{Arten der Filterimplementierung}
\begin{itemize}
    \item[1)]\textbf{Butterworth-Tiefpassfilter}\\
    Der Butterworth-Tiefpassfilter besitzt einen lagen horizontalen Frequenzgang, welcher erst kurz vor der Grenzfrequenz scharf abknickt. In der Sprungantwort lässt sich ein kräftiges Überschwingen (abhängig von der Ordnung) registrieren \citep{VA}.
    \item[2)]\textbf{Tschebyscheff-Tiefpassfilter}\\
    Ein Tschebyscheff-Tiefpassfilter besitzt oberhalb der Grenzfrequenz einen noch steileren Abfall als der Butterworthfilter, womit das Überschwingen der Sprungantwort im Vergleich noch stärker ist. Im Durchlassbereich verläuft die Verstärkung aber nicht monoton, sondern wellig mit konstanter Amplitude \citep{VA}.
    \item[3)]\textbf{Bessel-Tiefpassfilter}\\
    Der Bessel-Tiefpassfilter besitzt unter der Voraussetzung, dass die Phasenverschiebung in einem bestimmten Frequenzbereich proportional zur Frequenz ist, ein optimales Rechteck-Übertragungsverhalten. Der Frequenzgang knickt aber nicht so stark ein, wie bei den zwei vorher genannten Filter \citep{VA}.
    \item[4)]\textbf{Tschebyscheff-Tiefpass}\\
    Der Tschebyscheff-Tiefpass besitzt nach der Grenzfrequenz einen steilen Knick im Frequenzgang, wodurch auch dieser Filter eine Überschwingung der Sprungantwort aufweist. Dies hat zur Folge, dass der Frequenzgang im Durchlassbereich eine Welligkeit besitzt. Durch Verminderung dieser Welligkeit geht der Tschebyscheff kontinuierlich in den Butterworth über \citep{VA}. 
\end{itemize}

\subsection{Signalmittelung}
\label{sub:mittelung}
Bei stark verrauschten Signalen wird die Signalmittelung verwendet. Zu beachten ist aber, dass die Bandbreiten des Signals und des Rauschens in derselben Größenordnung liegen, was die Anwendung eines Filters ausschließen würde. Eine weitere Voraussetzung ist, dass das Signal wiederholbar und dessen Phase bekannt ist. Bei der Signalmittelung wird das Signal mit Rauschen in $n$ Segmente unterteilt und in $n$ Kanälen gespeichert. Dieser Vorgang wird mehrmals wiederholt und jeder neue Durchgang zum vorhandenen Speicherinhalt addiert, was ein Anwachsen des Signals proportional zu den Wiederholungen verursacht. Beim Rauschen aufgrund seiner statischen Natur wird nur der quadratischen Mittelwert auf das vorhandene Messsignal addiert. Allgemein erhält man für $N$ Wiederholungen eine Signal/Rausch-Verbesserung von $\sqrt{N}$ \citep{VA}. In Abbildung \ref{image:mittelung} wird die erzielte Signal/Rausch-Verbesserung durch ein Vorher-Nachher-Bild gezeigt.
\begin{center}
    \begin{tabular}{c c}
        \includegraphics[scale = 0.3]{Mittelung1.png} &
        \includegraphics[scale = 0.3]{Mittelung2.png}  
    \end{tabular}
    \captionof{figure}{Anwendung Signalmittelung: Links Signalmessung und Rechts Mehrfachmessung \citep{VA}}
    \label{image:mittelung}
\end{center}
\newpage
\subsection{Lock-In Verstärker}
\label{sub:lockin}
Ein Lock-In Verstärker ist ein Detektor zum Auflösen von kleinen Wechselspannungssignalen bis zu ein paar nV. Dabei ist eine akkurate Messung des Signals, welches 1000\,fach stärkeres Rauschen besitzt, möglich. Es kommt dafür eine sogenannte phasensensitive Detektion zum Einsatz, um eine bestimmte Komponente des Signals bei einer bestimmten Referenzfrequenz $f_{r}$ und Phase $\Theta_{r}$ zu bestimmen. Dabei wird Rauschen, welches nicht auf der Referenzfrequenz liegt, herausgefiltert und trägt nicht mehr zum Signal bei.
\subsection*{Phasensensitive Detektion}
Beim Lock-In wird wie oben erwähnt eine Referenzfrequenz benötigt. Diese Frequenz wird meistens durch das Experiment vorgegeben (z.B. Funktionsgenerator) und der Lock-In detektiert die Antwort des Experiments bei dieser Referenzfrequenz. In Abbildung \ref{image:signalRef} sind schematisch die jeweiligen Signale gezeigt. Als Referenzsignal wird eine Rechteckschwingung mit $f_{r}$ verwendet und das Experiment selbst wird von einer Sinusschwingung (Input-Signal) mit Funktion $s(t)=U_{s}\sin(\omega_{r}t + \Theta_{s})$ mit $\omega_{r}=2\pi f_{r}$, Amplitude $U_s$ und Phasenverschiebung $\Theta_{s}$ angeregt. Der Lock-In Verstärker erzeugt dann selbst eine Sinusschwingung mit Frequenz $f_l$ und der Funktion $l(t) = \sin(\omega_{l}t + \Theta_{r})$ mit $\omega_{l}=2\pi f_{l}$ als Signal, das Lock-In-Signal.
\begin{center}
    \includegraphics[scale = 0.25]{SignalLockIn.png}
    \captionof{figure}{Signale bei einem Lock-In Verstärker \citep{lockin}}
    \label{image:signalRef}
\end{center}
Nun multipliziert der Lock-In Verstärker das Input-Signal und das Lock-In-Signal mit einem phasensensitiven Detektor (PSD) oder einem Multiplikator. Daraus folgt mit Anwendung des Additionstheorems des Sinus:
\begin{gather}
    \begin{aligned}
        s_{x} &= s(t)\cdot l(t) = U_{s}\sin(\omega_{r}t + \Theta_{s}) \cdot \sin(\omega_{l}t + \Theta_{r})\\
                &= \frac{U_{s}}{2}\left[\cos((\omega_{r}-\omega_{l}) + (\Theta_s - \Theta_r)) + \cos((\omega_{r}+\omega_{l}) + (\Theta_s + \Theta_r))\right]
    \end{aligned}
\end{gather}
Damit ist das Output-Signal $s_x$ zwei Wechselspannungssignale mit unterschiedlichen Frequenzen. Dieses Signal wird durch einen Tiefpassfilter geschickt, welcher alle Wechselspannungssignale ab der Grenzfrequenz $f_g$ eliminiert. Die Grenzfrequenz $f_g$ wird dabei am Lock-In Verstärker über die Zeitkonstante $\tau$ eingestellt mit der Beziehung: $f_g\sim\frac{1}{\tau}$\\ Wenn $\omega_r = \omega_l$ ist, ist das Signal mit der Differenz der Frequenz ein Gleichspannungssignal und wird somit nicht gefiltert. Der neue PSD Output ist dann:
\begin{gather}
    s_{x} = \frac{U_s}{2} \cos(\Theta_s-\Theta_r) = \frac{U_s}{2} \cos(\Delta\Theta)  
\end{gather}
Bei einem Phasenunterschied von $\Delta\Theta = 0$ lässt sich hierbei das Maximum der Amplitude messen ($\frac{U_s}{2}$) und bei $\Delta\Theta = \frac{\pi}{2}$ misst man überhaupt kein Output-Signal mehr. Solche Lock-In Verstärker mit nur einem PSD werden auch Einphasen-Lock-In genannt. Man kann aber auch ein zweites Lock-In-Signal mit Cosinusschwinung verwenden und dieses mit einem zweiten PSD mit dem Input-Signal multiplizieren. Mit demselben Prozess wie zuvor erhält man die Beziehung:
\begin{gather}
    s_{y} = \frac{U_s}{2} \sin(\Delta\Theta) 
\end{gather} 
Das Output $s_x$ heißt \enquote{in Phase}-Komponente und $s_y$ die \enquote{Quadratur}-Komponente, weil bei $\Delta\Theta = 0$ ist $s_y = 0$ und $s_x$ misst das Input-Signal.\\

Weiterhin kann man mit $s_x$ und $s_y$ wie folgt die Amplitude des Output-Signals bestimmen:
\begin{gather}
    A = (s_x^2 + s_y^2)^{\frac{1}{2}} = \frac{U_s}{2}
\end{gather}
\subsection*{Was wird mit einem Lock-In gemessen?}
Ein Lock-In Verstärker misst aufgrund der Multiplikation mit einer Sinusschwingung (Cosinusschwingung) den ersten Term der Fourier-Reihe des Input-Signals bei der Referenzfrequenz $f_r$. Dabei ist aber zu beachten, dass auch Rauschen, welches bei der Referenzfrequenz auftritt, mit gemessen wird und gegebenenfalls abgezogen werden muss. Dies erkennt man daran, dass das Messsignal am Lock-In Verstärker Schwankungen aufweist.\\ In Abbildung \ref{image:Block} ist zusätzlich der Aufbau des Lock-In Verstärkers dargestellt \citep{lockin}.
\newpage
\begin{center}
    \includegraphics[scale = 0.6]{BLockdiagrammLockIn.png}
    \captionof{figure}{Blockdiagramm Lock-In Verstärker \citep{lockin}}
    \label{image:Block}
\end{center}

\newpage
\section{Leistungspegel}
\label{sec:pegel}
Der Leistungspegel $L_p$ gibt das 10fache logarithmische Verhältnis zwischen Nutzleistung $P$ und Bezugsleistung $P_0$ in \si{\deci\bel} an und ist definiert als:
\begin{gather}
    L_P = 10 \log_{10}\left(\frac{P}{P_0}\right)
    \label{eq:pegel}
\end{gather}
In diesem Versuch ist vom Interesse den Leistungspegel der Spannungen $L_U$ anzugeben. Dafür wird die Formel $P = U \cdot I$ mit dem ohmischen Gesetz $R = \frac{U}{I}$ umgestellt zu $P = \frac{U^2}{R}$ und in Gleichung (\ref{eq:pegel}) eingesetzt. Somit erhält man:
\begin{gather}
    L_U = 10 \log_{10}\left(\frac{U^2}{U_0^2}\right) = 20 \log_{10}\left(\frac{U}{U_0}\right)
    \label{eq:spannungpegel}
\end{gather}
Damit entspricht der Leistungspegel der Spannungen das doppelte des herkömmlich definierten Leistungspegel \citep{electronik}.\\
Um das Verständnis zu vertiefen werden noch einige bestimmte Verhältnisse zu charakteristischen \si{\deci\bel} angegeben:
\begin{center}
    \begin{tabular}{c | c c}
        $L$/dB & $\frac{P}{P_0}$ & $\frac{U}{U_0}$\\
        \hline
        $-n\cdot10$ & $1\cdot10^{-n}$ &  $1\cdot10^{-n/2}$ \\
        -20 & 1/100 & 1/10 \\
        -10 & 1/10 & 1/$\sqrt{10}$ \\
        -6 & 1/4 & 1/2\\
        -3 & 1/2 & 1/$\sqrt{2}$\\
        0 & 1 & 1\\
        3 & 2 & $\sqrt{2}$\\
        6 & 4 & 2\\
        10 & 10 & $\sqrt{10}$ \\
        20 & 100 & 10 \\
        $n\cdot10$ & $1\cdot10^{n}$ &  $1\cdot10^{n/2}$ \\
    \end{tabular}
    \captionof{table}{Leistungspegel und zugehörige Verhältnisse}
    \label{tab:pegel}
\end{center}
In Tabelle \ref{tab:pegel} lässt sich sehr gut die Verdopplungsregel des Leistungspegels erkennen. Dabei steigt der Pegel um 3\,dB an (bei $L_U$ um 6\,dB), wenn man das Verhältnis verdoppelt. Dies hat damit zu tun, dass der 10er-Logarithmus von 2 ungefähr 0.3 ist und dieser mit der Multiplikation von 10 nach der Gleichung (\ref{eq:pegel}) dann 3\,dB ergibt.
\newpage
\section{Theorem von Nyquist}
Das Theorem von Nyquist oder auch Abtasttheorem besagt, dass aus den Abtastwerten das ursprüngliche Signal (kontinuierlich) fehlerfrei rekonstruiert werden kann, wenn die Abtastfrequenz mindestens doppelt so groß ist wie die höchste Signalfrequenz $f_{max}$. 
\begin{gather}
    f \geq 2\cdot f_{max}
\end{gather} 
Die Frequenz $2\cdot f_{max}$ wird als \textit{Nyquist-Frequenz} bezeichnet. Aus dem Abtasttheorem folgt auch, dass das Spektrum des Signals \textit{bandbegrenzt} ist, d.h. das Signal im Spektrum muss ab der maximalen Frequenz $f_{max}$ gleich 0 sein \citep{praktikum}.
\section{Fouriertransformation und Schnelle Fouriertransformation}
Bei einer Fouriertransformation wird ein gegebenes Signal (ggf. eine Funktion) komplett in den Frequenzraum transformiert, dabei wird ein Integral von $-\infty$ bis $\infty$ ausgewertet. Der Rechenaufwand einer Fouriertransformation ist in der Regel in der Praxis sehr hoch, weswegen meist eine Schnelle Fouriertransformation (engl. Fast Fourier Transformation (FFT)) durchgeführt wird. Unter einer FFT versteht man eine effiziente Realisierung der Diskrete Fouriertransformation (DFT), mit der redundante Rechenschritte vermieden werden. Bei einer DFT wird nur ein abgetastetes Signal in den Frequenzraum überführt. Der Rechenaufwand von $N^2$ bei einer DFT verringert sich dann  bei einer FFT zu etwa $N\log_2\left(N\right)$ \citep{praktikum}.
\section{Fourier-Reihe und Effektivspannung}
\label{sec:fourierseries}
\subsection*{Allgemeines zur Fourier-Reihe und Effektivspannungen}
\label{sub:fourierseriesAllgemein}
Eine Fourier-Reihe zerlegt eine gegeben periodische Funktion in ihre jeweiligen Sinus und Cosinusanteile. Die reelle Fourier-Reihe einer bestimmten $T$-periodischen Funktion lässt sich mit den folgenden Formeln berechnen:
\begin{gather}
    f(t) = \frac{a_0}{2} + \sum^{\infty}_{k=1} \left[a_k \sin(k\frac{2\pi}{T} t) +b_k \cos(k\frac{2\pi}{T} t)\right]\\
    a_k = \frac{2}{T} \int^{\frac{T}{2}}_{-\frac{T}{2}} f(t)\cos(k \frac{2\pi}{T} t)dt \tab
    b_k = \frac{2}{T} \int^{\frac{T}{2}}_{-\frac{T}{2}} f(t)\sin(k \frac{2\pi}{T} t)dt
\end{gather}
Dabei sind die Grenzen der Integrale von $-\frac{T}{2}$ bis $\frac{T}{2}$ nicht fest, sie können verschoben werden. Es ist aber wichtig, dass über eine Periode integriert wird in diesem Fall über eine komplette Periodendauer $T$ \citep{praktikum}.\\

Um die effektiven Spannungswerte der jeweiligen Schwingungsform zu bestimmen, bildet man das sogenannte \enquote{Quadratische Mittel}. Dieses ist wie folgt definiert:
\begin{gather}
    U_{eff} = \sqrt{\frac{1}{T}\int^T_0 f(t)^2 dt}
\end{gather}
Hierbei ist es wieder zu erwähnen, dass die Grenzen der Integration nicht fest sind, aber die Integration über eine Periodenlänge erfolgen muss \citep{messtechnik}.\\

Die detaillierteren Berechnungen zu jeder Schwingungsform lassen sich in Anhang \ref{app:Berechnung} nachlesen.

\subsection*{Sinusschwingung}
\label{sub:sinus}
Der Fall der Sinusschwingung ist besonders einfach, da wie vorangegangen erwähnt, die Fourier-Reihe eine periodische Funktion in ihre Sinus und Cosinusanteile zerlegt. Daraus folgt die Fourier-Reihe der Sinusschwingung ist die Sinusschwingung selbst und kann somit trivial angegeben werden als:
\begin{gather}
    \boxed{f(t) = U_0\sin(\frac{2\pi}{T} t)}
\end{gather}
Die Effektivspannungen der Sinusschwingung:
\begin{gather}
    \boxed{U_{eff}=\frac{U_0}{\sqrt{2}}\approx 0,70711 \cdot U_0}
\end{gather}

\subsection*{Rechteckschwingung}
\label{sub:square}
Als Nächstes wird die Fourier-Reihe der Rechteckschwingung bestimmt. Diese hat die Form:
\begin{gather}
    f(t) = 
    \begin{cases}
        +U_0, & 0 \leq t \leq \frac{T}{2} \\
        -U_0, & \frac{T}{2} \leq t \leq T \\
    \end{cases}
\end{gather}
Da die Funktion der Rechteckschwingung punktsymmetrisch zum Ursprung ist, fallen alle Cosinusanteile weg, da $a_k = 0$. Somit muss nur $b_k$ wie folgt berechnet werden:
\begin{gather}
    b_k =
    \begin{cases}
        0, & k~\text{gerade}\\
        \frac{4U_0}{\pi}\frac{1}{k}, & k~\text{ungerade}\\
    \end{cases}
\end{gather} 
Es werden nur noch Terme mit ungeraden $k$ betrachtet und man erhält:
\begin{gather}
    \boxed{f(t) = \frac{4U_0}{\pi} \sum^{\infty}_{k=1} \frac{1}{2k-1} \sin((2k-1)\frac{2\pi}{T}t)}
\end{gather}
Die Effektivspannungen der Rechteckschwingung:
\begin{gather}
    \boxed{U_{eff} = U_0}
\end{gather}
\subsection*{Dreiecksspannung}
\label{sub:triangle}
Als Letztes wollen wir die Dreieckschwingung betrachtet. Die Form dieser ist definiert wie folgt:
\begin{gather}
    f(t) = 
    \begin{cases}
        at, & -\frac{T}{4} \leq t \leq \frac{T}{4} \\
        a\left(\frac{T}{2}-t\right), & \frac{T}{4} \leq t \leq \frac{3T}{4} \\
    \end{cases}
    ~\text{mit}~U_0 = \frac{aT}{4}
\end{gather} 
Die Funktion ist erneut punktsymmetrisch zum Ursprung, wodurch wieder alle $a_k$-Koeffizienten 0 sind. $b_k$ ergibt sich dann durch wie folgt:
\begin{gather}
    b_k =
    \begin{cases}
        0, & k~\text{gerade}\\
        \frac{8U_0}{\pi^2}\frac{(-1)^{k-1}}{k^2}, & k~\text{ungerade}\\
    \end{cases}
\end{gather}
Es werden wieder nur die Terme mit ungeraden $k$ betrachtet. Somit erhält man:
\begin{gather}
    \boxed{f(t) = \frac{8U_0}{\pi^2} \sum^{\infty}_{k=1} \frac{(-1)^{k-1}}{(2k-1)^2} \sin((2k-1)\frac{2\pi}{T}t)}
\end{gather} 
Die Effektivspannungen der Dreieckschwingung:
\begin{gather}
     \boxed{U_{eff} = \frac{U_0}{\sqrt{3}}\approx 0,57735 \cdot U_0}
\end{gather}
\section*{Gemeinsamkeiten von Rechteck und Dreiecksschwingung}
Vergleicht man die Fourierreihe der Rechteck und Dreiecksschwingung erkennt man, dass beide Reihen nur aus Sinusschwingungen bestehen mit den selben Argumenten. Weiterhin ist zu erwähnen, dass die Reihen nur ungerade $k$ besitzen. Die einzigen Unterschiede treten auf bei den Vorfaktoren und den Summenkoeffizienten. Bei den Summenkoeffizienten besitzt die Rechteckschwingung eine $\frac{1}{k}$-Faktor, während die Dreiecksschwingung einen $\frac{1}{k^2}$-Faktor, welcher alterniert, aufweist. Dieser Vergleich zeigt deutlich, dass zwischen den beiden Schwingungsformen in ihrem Aufbau kein großer Unterschied in der Fourierreihe besteht, obwohl die Signal verschiedene Eigenschaften und Formen besitzen.

% etc.

    % 3.Kapitel Protocol
    % 3. Protocol

\chapter{Protocol}
\label{chap:protocol}

\section*{Adjustment for measurement}
We adjust the test setup after instruction of the supervisor with the adjusting tip. For that we make sure to take two points for two mirrors that are as far as possible away from the specific mirror.

\section*{Chanels}

\begin{tabular}[]{c|c|l}
    Input & Output & Description\\
    \hline
    ai0 & RF Output & balance of reference beam and measurement beam\\
    ai1 & Monitor + & measurement beam\\
    ai3 & Monitor - & reference beam\\
    ai4 & PD2 Output & signal of fabry-pérot-interferometer\\
\end{tabular}\\

For our files we have taken the name: \textit{MesN\_TempY\_ZPeaks} there $N$ is number of the measurement/id and $Y$ is the temperature \SI{24}{\celsius}, \SI{38}{\celsius}, \SI{56,2}{\celsius} without the unit and $Z$ is the number to identify the peak with the values all, 1, 2, 3, 4 which is number from left to right.
To every measurement we save a .dat-file and a .bmp-file on our usb-stick.
We also used always the act-value and not the set-value for measurement.

\section*{Measurement}
\begin{itemize}
    \item \textbf{Temperature: \SI{24}{\celsius}} \\ 
    Filename: \textit{date+time\_group11\_MesN\_Temp24\_ZPeak.dat and .bmp}\\ $N$ from 1 to 5, act-value: \SI{24}{\celsius}
    \item \textbf{Temperature: \SI{38}{\celsius}} \\ 
    Filename: \textit{date+time\_group11\_MesN\_Temp38\_ZPeak.dat and .bmp}\\ $N$ from 6 to 10, act-value: \SI{40}{\celsius}
    \item \textbf{Temperature: \SI{56,2}{\celsius}} \\ 
    Filename: \textit{date+time\_group11\_MesN\_Temp38\_ZPeak.dat and .bmp}\\ $N$ from 11 to 15, act-value: \SI{60}{\celsius}
    \item \textbf{Distance fabry-pérot-interferometer:} \SI{72,5}{\centi\metre}+\SI{36}{\centi\metre}+\SI{43}{\centi\metre}=\SI{151,5}{\centi\metre}\\ measured with tape measure
\end{itemize}

\begin{center}
    \includegraphics[scale = 0.4]{currentTuning.jpg}
    \captionof{figure}{Currunt Tuning at \SI{22}{\celsius} from laser instructions}
    \label{image:currentTuning}
\end{center}

    % 4.Kapitel Evaluation
    % 4. Evaluation

\chapter{Evaluation}
\label{chap:eval}
We have chosen to take the measured data for \SI{24}{\celsius}, because the reference beam and the sample beam have had the best alignment. It seems also in this data the observation of the height of the lamb dips and the width of the lines would be much easier as in the other temperatures.

% Text

% Part 1

\section{Freeing Absorption Spectrum from Trend and identify Lines}
\label{sec:freeing}
Now we want to free the spectrum from any trend for that the fit a linear function on the spectrum of the reference beam and subtract the fit from the sample beam and reference beam. It is worth mentioning that we have moved the reference beam spectrum down to the niveau of the sample beam spectrum and after that we fitted the linear function. In fig. \ref{image:trends} we can see the absorption spectrum with trends and the linear fit and in fig. \ref{image:trendless} the spectrum without trends. The fit is managed with the function polyfit of the numpy-module in python. We did also cut the data so that we can clearly see the four peaks without any jumps in mode.
\begin{center}
    \includegraphics[scale=0.47]{Aufg-1/trend24.pdf}
    \captionof{figure}{absorption spectrum with trends}
    \label{image:trends}
\end{center}
\begin{center}
    \includegraphics[scale=0.47]{Aufg-1/trendless24.pdf}
    \captionof{figure}{absorption Spectrum without trends}
    \label{image:trendless}
\end{center}
We applied the same procedure on the data of the fabry-pérot and see clear in fig. \ref{image:allTrendless} that the signal is not very stable, but this is not from interest for the upcoming evaluation because we need only the distance between to peaks.
\begin{center}
    \includegraphics[scale=0.47]{Aufg-1/alltrendless24.pdf}
    \captionof{figure}{absorption spectrum without trends all channels}
    \label{image:allTrendless}
\end{center}
\subsection*{Identfication considering intensity and order}
To identify the lines of the spectrum we used the current wavelength curve (look fig. \ref{image:currentTuning}) to convert the laser current to the wavelength. To do this we convert the fig. \ref{image:currentTuning} into a csv-file using python. For that we measured roughly the points of the peaks from 105 mA to 150 mA (because that is the area of interest) and calculated a basic linear function of the form $y=mx +t$ between them (look fig. \ref{image:currentTuningCut}). Then we obtain three separate linear functions that can transform our data in the corresponding wavelength in the specific area of the current wavelength curve.
\begin{center}
    \begin{tabular}{c | c c c c}
        {} & 1 & 2 & 3 & 4 \\
        \hline
        laser current/mA & 107.0 & 128.5 & 129.5 & 149.5\\
        wavelength/nm & 780.23125 & 780.25125 & 780.234 & 780.2535\\
    \end{tabular}
    \captionof{table}{point used for recreating current wavelength curve}
    \begin{tabular}{c | r r }
        area & m/$\frac{\text{nm}}{\text{mA}}$ & t/nm\\
        \hline
        $1 \rightarrow 2$ & 0.00093  & 780.132 \\
        $2 \rightarrow 3$ & -0.01725 & 782.468 \\
        $3 \rightarrow 4$ & 0.00097  & 780.108 \\
    \end{tabular}
    \captionof{table}{linear functions of the current wavelength curve}
\end{center}
\begin{center}
    \includegraphics[scale=0.45]{Aufg-1/currentTuning.pdf}
    \captionof{figure}{cut current wavelength curve from 105 mA to 150 mA using python}
    \label{image:currentTuningCut}
\end{center}
From the csv-file we can easily find our value for the peaks of the reference beam and convert them into the corresponding wavelength. Then we compared the measured wavelength with the literature \citep{RDL85,RDL87} there $F$ are the quantum number of the $5^2S_{1/2}$-state.
\begin{center}
    \begin{tabular}{c | c | c c c | c c }
        \makecell{peak\\order} & \makecell{laser\\current/mA} & \makecell{wavelength\\measured/nm} & \makecell{wavelength\\literature/nm} & deviation/nm & isotope & $F$ \\
        \hline
        1 & 119.4429 & 780.243 & 780.233 & 0.010 & $^{87}Rb$ & 1 \\
        2 & 125.0267 & 780.248 & 780.238 & 0.010 & $^{85}Rb$ & 2 \\
        3 & 131.8581 & 780.236 & 780.244 & 0.008 & $^{85}Rb$ & 3 \\
        4 & 134.7829 & 780.239 & 780.246 & 0.007 & $^{87}Rb$ & 2 \\       
    \end{tabular}
    \captionof{table}{identification by intensity and order}
    \label{tab:identify}
\end{center}
Because $^{85}Rb$ have to be isotope with the highest occurrence we can clearly identify the highest peaks of the absorption spectrum as the one of $^{85}Rb$. In addition to that we can conclude that the measured data have some kind of bias, roughly 0.010 nm, in each area of the current wavelength curve of the laser.
\section{Distance between Energy Levels}
\label{sec:distance}
\subsection*{Current Wavelength Curve}
As mentioned in chapter \ref{sec:freeing} we already have transformed all data into the wavelength using the current wavelength curve. Because of that the order of the identified peaks from table \ref{tab:identify} has changed to 3 (${85}^Rb$, $F=3$), 4 (${87}^Rb$, $F=2$), 1 (${87}^Rb$, $F=1$), 2 (${85}^Rb$, $F=2$). The next step is to transform the wavelength into the frequency using the formula $\nu=\frac{c}{\lambda}$, where $c$ is the speed of light in vacuum. The result can be seen in fig. \ref{image:fequency} where the new order of the peaks is 3, 4, 1, 2.
\begin{center}
    \includegraphics[scale=0.47]{Aufg-2/frequencyallPeakTemp24.pdf}
    \captionof{figure}{absorption spectrum after transformation to \textbf{frequency}}
    \label{image:fequency}
\end{center}
One can also see the occurrence of overlapping in fig. \ref{image:fequency} that is caused by the transformation from the laser current to the wavelength. It is possible to erase the overlapping lines from the data but for this evaluation it is necessary.
By using the same method as in chapter \ref{sec:freeing} to identify the peaks we obtain distance as followed:
\begin{center}
    \begin{tabular}{c | c}
        peak area & $d_{current}$/THz\\
        \hline
        $2 \rightarrow 1$ & 0.00256\\
        $1 \rightarrow 4$ & 0.00181\\
        $4 \rightarrow 3$ & 0.00140\\
    \end{tabular}
    \captionof{table}{distances between the energy levels using current wavelength curve}
    \label{tab:currentMethode}
\end{center}
\subsection*{Fabry-Pérot Interferometer}
To calculate the distance of the energy level we start using the function of the Fabry-Pérot interferometer and insert the measured length $d$:
\begin{gather}
    \Delta\omega_{FSR} = \frac{c}{2nd} \overset{n=1}{=} \frac{c}{2d} \overset{d=\SI{1.515}{\metre}}{=} \SI{98.9}{\mega\hertz} 
\end{gather}
In the trendfree line of the interferometer we can count from 117.0455 mA to 137.6816 mA an amount of 97 maxima peaks. For each peak we get a current value that we convert into relative frequency $\Delta\omega_{FSR}$ starting with the value 0 for the first peak. This gives us the figure \ref{image:relfequency}.
\begin{center}
    \includegraphics[scale = 0.45]{Aufg-2/relfrequencyallPeakTemp24.pdf}
    \captionof{figure}{absorption spectrum after transformation to \textbf{relative frequency}}
    \label{image:relfequency}
\end{center}
In the upcoming figures we decided to abstain from including the relative frequency transformed data for clearness. 
\newpage
In following we count the amount of fabry-pérot interferometer peaks between each peak of the absorption spectrum and multiply that with $\Delta\omega_{FSR}$. That gives us in comparsion with the calculated data from table \ref{tab:currentMethode}:
\begin{center}
    \begin{tabular}{c | c c c}
        peak area & difference/mA & $d_{interferometer}$/THz & $d_{current}$/THz\\
        \hline
        $1 \rightarrow 2$ & 5.5838 & $26\cdot\Delta\omega_{FSR} = 0.00257$ & 0.00256\\
        $2 \rightarrow 3$ & 6.8314 & $32\cdot\Delta\omega_{FSR} = 0.00316$ &   ---  \\
        $3 \rightarrow 4$ & 2.9248 & $12\cdot\Delta\omega_{FSR} = 0.00119$ & 0.00140\\
    \end{tabular}
    \captionof{table}{distance between the energy levels using Fabry-Pérot interferometer and comparison to usage of current wavelength curve}
    \label{tab:interferometerMethode}
\end{center} 
In table \ref{tab:interferometerMethode} we can clearly see that both methods are equal in evaluation.
\newpage

%\begin{gather}
%    \frac{97\cdot \Delta\omega_{FSR}}{\SI{137.6816}{\milli\ampere}-\SI{117.0455}{\milli\ampere}} = 0.465\frac{\si{\giga\hertz}}{\si{\milli\ampere}}
%\end{gather}
%This means that 1 mA correspond to \SI{0.465}{\giga\hertz}.

%In following we use the table \ref{tab:identify} from chapter \ref{sec:freeing} for the information of the current for each peak and take the difference of them. That gives us in comparsion with the calculated data from table \ref{tab:currentMethode}:
%\begin{center}
%    \begin{tabular}{c | c c c}
%        peak area & difference/mA & $d_{interferometer}$/THz & $d_{current}$/THz\\
%        \hline
%        $1 \rightarrow 2$ & 5.5838 & 0.00260 & 0.00256\\
%        $2 \rightarrow 3$ & 6.8314 & 0.00318 &   ---  \\
%        $3 \rightarrow 4$ & 2.9248 & 0.00136 & 0.00140\\
%    \end{tabular}
%    \captionof{table}{distance between the energy levels using Fabry-Pérot interferometer and comparison to usage of current wavelength curve}
%    \label{tab:interferometerMethode}
%\end{center} 
\section{The real ratio of the rubidium isotopes}
\label{sec:ratio}
Now we calculate the ratio of each isotope by identifying the area under every peak. For that we are fitting the gaussian distribution for each peak of our data for the reference beam absorption spectrum. We take the current axis because for the ratio it does not matter which axis we use. Furthermore, we use the data that is freed from any trends (look fig. \ref{image:trendless}).Then the fitting function has the form of:
\begin{gather}
    y = a\cdot\exp(-\left(\frac{(x-b)}{\sqrt{2}c}\right)^2)
    \label{eq:gaussFit}
\end{gather}
$b$ is here the x-value of each peak that we have already obtained in table \ref{tab:identify}. We get with the curve_fit of scipy.optimize package from python:
\begin{center}
    \begin{tabular}{c | c c c}
        peak & a/V & b/mA & c/mA\\
        \hline
        1 &  -0.00539 & 119.4429 & 0.57669\\
        2 &  -0.01742 & 125.0267 & 0.52976\\
        3 &  -0.03085 & 131.8581 & 0.62443\\
        4 &  -0.01462 & 134.7829 & 0.74425\\
    \end{tabular}
    \captionof{table}{fitting data for each peak}
\end{center}
In figure \ref{image:gaussFit} is shown how each gaussian fit looks for each peak.\\
With the parameter we calculated above we are now able to determine the area under each curve of each peak with the following relation:
\begin{gather}
    \int^{\infty}_{-\infty}\exp(-k(x-\mu)^2)\,dx = \sqrt{\frac{\pi}{k}} \xrightarrow{\mu = b,~k = \frac{1}{2c^2}}\int^{\infty}_{-\infty}a\cdot\exp(-\left(\frac{(x-b)}{\sqrt{2}c}\right)^2) = a \sqrt{2\pi c^2}
\end{gather}
This expression gives us then the area as follows:
\begin{center}
    \begin{tabular}{c | c | c}
        peak & isotope & area/$1\cdot 10^{-6}$ W\\
        \hline
        1 & $^{87}Rb$ &  -7.79 \\
        2 & $^{85}Rb$ & -23.13 \\
        3 & $^{85}Rb$ & -48.29 \\
        4 & $^{87}Rb$ & -27.27 \\
    \end{tabular}
    \captionof{table}{area under the curve of each peak}
\end{center}
The area under the peaks is proportional to the amount of atoms of each isotope in the gas what gives us:
\begin{gather}
    ^{85}Rb = \frac{23.13+48.29}{7.79+23.13+48.29+27.27} = 0.671 \Rightarrow  {^{87}Rb} = 0.329
\end{gather}
Meaning that in our probe there is \SI{67.1}{\percent} $^{85}Rb$ and \SI{32.9}{\percent} $^{87}Rb$ which compared to the literature \citep{RDL85,RDL87} (\SI{72.2}{\percent} $^{85}Rb$ and \SI{27.8}{\percent} $^{87}Rb$) shows that the assumption from chapter \ref{sec:freeing} to take the intensity into account was correct.
\begin{center}
    \includegraphics[scale=0.72, angle = 90]{Aufg-3/gaussFit.pdf}
    \captionof{figure}{gaussian fit for each Peak of the reference beam spectrum}
    \label{image:gaussFit}
\end{center}
\section{Hyperfine Dips}
\label{sec:hyperfine}
Firstly we want to show all absorption dips separately starting with peak 1 and ending with peak 4.
\begin{center}
    \includegraphics[scale=0.45]{Aufg-4/hyperfine1.pdf}
    \captionof{figure}{absorption spectrum of peak number 1}
    \label{image:peak1}
\end{center}
\begin{center}
    \includegraphics[scale=0.45]{Aufg-4/hyperfine2.pdf}
    \captionof{figure}{absorption spectrum of peak number 2}
    \label{image:peak2}
\end{center}
\begin{center}
    \includegraphics[scale=0.45]{Aufg-4/hyperfine3.pdf}
    \captionof{figure}{absorption spectrum of peak number 3}
    \label{image:peak3}
\end{center}
\begin{center}
    \includegraphics[scale=0.45]{Aufg-4/hyperfine4.pdf}
    \captionof{figure}{absorption spectrum of peak number 4}
    \label{image:peak4}
\end{center}
\newpage
It was chosen to use the peak number 3 because there are the hyperfine dips clearly visible. Its the isotope $^{85}Rb$. For the upcuming calculation we will use the measured data of peak 3 only because of that we have had an offset in our data from $z_{off}=\SI{0.072}{\volt}$ and have to transform the laser current again like in chapter \ref{sec:freeing} into wavelength and frequency.
The fit of the Lorentz curve (lorentzian) was achieved with the function:
\begin{gather}
    y = \frac{a c^2}{(x-b)^2+c^2} - d
\end{gather}
By observing the formula of the fit it gets clear that $a$ is nothing else as the laser current value and $b$ the amplitude of each hyperfine dip peak. From that conclusion we get for the hyperfine peaks numbered from left to right:
\begin{center}
    \begin{tabular}{c | c c c c c c}
        \makecell{hyperfine\\peak} &\makecell{$a$/mA} & $a$/nm &  $a$/THz &  $b$/V & $c$/mA & $d$/V \\
        \hline
        1   &   131.8489  &  780.236290  &  384.232907   &  -0.0269 & -0.0266 & -0.032\\
        2   &   132.0006  &  780.236438  &  384.232834   &  -0.0169 & -0.0285 & -0.032\\
        3   &   132.0778  &  780.236513  &  384.232797   &  -0.0177 & -0.0325 & -0.032\\
        4   &   132.1607  &  780.236594  &  384.232757   &  -0.0254 & -0.0680 & -0.032\\
        5   &   132.2349  &  780.236667  &  384.232722   &  -0.0243 & -0.0613 & -0.032\\
        \end{tabular}
        \captionof{table}{fitting data for each hyperfine dip}
        \label{tab:lorFit}
\end{center}
Furthermore the lorentzian fit for each dip can be seen in fig. \ref{image:lorFit}.\\
It is clear that there are more dips than possible hyperfine transition. In case of $^{85}Rb$ each dip represents a transition from energy levels $F=3$ of the state $5^2S_{1/2}$ to a different energy level of the $5^2P_{3/2}$ state with quantum number $F'$, in short: $F=3\rightarrow F'$. With the selection rule of the hyperfine transition ($\Delta F = 0,\pm1$) we obtain that there should be only three dips visible ($F'=2,3,4$) the remaining dips come from cross over resonances.
For Comparison with the literature \citep{RDL85} we calculate the distance between the dips using table \ref{tab:lorFit} and obtain:
\begin{center}
    \begin{tabular}{c | c c | c c}
        \makecell{dip area} & \makecell{distance\\measured/MHz} & \makecell{distance\\literature/MHz} & \makecell{transition\\ $F'_1\leftrightarrow F'_2$} \\
        \hline
        $1\rightarrow 2$ & 73.00 & 63.38 & $2\leftrightarrow 3$ \\
        $2\rightarrow 5$ & 114.00 & 120.99 & $3\leftrightarrow 4$\\
    \end{tabular}
    \captionof{table}{distance between the selected dips}
    \label{tab:disDip}
\end{center}
The Comparison with the literature shows us that indeed the peek 3 should be $^{85}Rb$ as we assumed in chapter \ref{sec:freeing} and \ref{sec:ratio}. The difference between literature and measurement could have the same cause as the difference in chapter \ref{sec:freeing}.
\newpage
\begin{center}
    \includegraphics[scale=0.72, angle = 90]{Aufg-4/hyperfinePeak3.pdf}
    \captionof{figure}{lorentzian fit for each dip of peak 3}
    \label{image:lorFit}
\end{center}

%$1\rightarrow 2$ & 73.0 & & \\
%        $2\rightarrow 3$ & 37.0 & & \\
%        $3\rightarrow 4$ & 40.0 & & \\
%        $4\rightarrow 5$ & 37.0 & & \\
\section{Hyperfine constant}
In the following we use the equation \ref{eq:HFS} to calculate the hyperfine constants.
\begin{align}
    \label{eq:HFS}
    \Delta E_{HFS} = \frac{a}{2} [F(F+1) - J(J+1) - I(I+1)]
\end{align}
We know the Energy difference and the Quantum numbers of the transitions from table \ref{tab:disDip}. So we can rewrite the equation: 
\begin{align}
    a = \frac{2(\Delta E)}{F_2(F_2+1)-F_1(F_1+1)} \qquad \text{mit} \quad \Delta E = h \cdot \Delta \nu
\end{align}
After we put in the values we get the following results: 
\begin{table}[h]
    \centering
\begin{tabular}{c|c|c}
    transition & a in $10^{-26}$ J & literature: a in $10^{-26}$ J \\
    \hline
    $2\leftrightarrow 3$ & 1.612 & 1.400\\
    $3\leftrightarrow 4$ & 1.888 & 1.840
\end{tabular}
\caption{hyperfine constants}
\end{table} \\
The literature value of $a$ is computed by the literature value of $\Delta \nu$ from table \ref{tab:disDip}. The difference of one value cloud have the same problem as mentioned in chapter \ref{sec:hyperfine}. 
\newpage
\section{Gas Temperatures}
\label{sec:temp}
To calculate the gas temperature we can use the formula
\begin{gather}
    \Delta \nu_D = \frac{2\nu_0}{c}\sqrt{\ln(2)\frac{2k_BT}{m}} = \frac{2\nu_0\hat{v}}{c}\sqrt{\ln(2)}~\text{with}~\hat{v}= \sqrt{\frac{2k_BT}{m}},
\end{gather}
where $\Delta\nu_D$ is the doppler width, $\nu_0$ the frequency of the peak, $\hat{v}$ the most probable velocity, $k_B$ the Boltzmann constant, $T$ the temperature and $c$ the speed of light in vacuum.
First we have to calculate the doppler width or most probable velocity. For that we are fitting a gaussian on the spectrum of the reference beam as in chapter \ref{image:gaussFit} with the form:
\begin{gather}
    y = y(\nu_0) \exp(-\left(\frac{\nu-\nu_0}{\sigma}\right)^2)
\end{gather}
With the value of $\sigma$ one can calculate $\Delta\nu_D$ or $\hat{v}$ as following:
\begin{gather}
    \Delta\nu_D = 2\sqrt{\ln(2)}\sigma \Rightarrow \sigma = \frac{\nu_0\hat{v}}{c} \Leftrightarrow \hat{v} = \frac{\sigma c}{\nu_0}
\end{gather}
After that one can obtain $T$ and the mean velocity $\overline{v}$:
\begin{gather}
    \hat{v} =  \sqrt{\frac{2k_BT}{m}} \Leftrightarrow T = \frac{m}{2 k_B} \hat{v}^2~\text{and}~\overline{v} = \sqrt{\frac{8k_BT}{\pi m}} = \sqrt{\frac{4}{\pi}}\cdot\hat{v}
\end{gather} 
We selected peak 2 for the upcoming calculation for each temperature. Because of the previous chapters \ref{sec:freeing} and \ref{sec:ratio} we know that peak 2 have to be $^{87}Rb$ and so the mass is $m\approx1.44322\cdot 10^{-25}$kg. We have transformed the laser current into frequency as in chapter \ref{sec:freeing} and with the knowledge of $\sigma$ we get following table:
\begin{center}
    \begin{tabular}{r | c c c| c c | c c}
        Temp & $\nu_0$/THz & $y(\nu_0)$/V & $\sigma$/MHz & $\hat{v}$/$\frac{m}{s}$ & $\overline{v}$/$\frac{m}{s}$& $T$/K & $T_{act}$/K\\
        \hline 
        \SI{24}{\celsius}   &384.22713 & -0.0173 & 345.26881 & 269.40 & 303.98 & 379.31 & 297,15\\ 
        \SI{38}{\celsius}   &384.22737 & -0.0313 & 361.10835 & 281.75 & 317.93 & 414.91 & 313,15\\ 
        \SI{56.2}{\celsius} &384.22724 & -0.0482 & 398.62826 & 311.03 & 350.96 & 505.62 & 329,35\\ 
    \end{tabular}
    \captionof{table}{fit parameter, velocities and temperature of the gas for each acted temperature}
\end{center}
It appears that the calculated temperatures for the gas differs from the acting temperature. But it is worth noticing that the value for $\sigma$ and so also the doppler width $\Delta\nu_D$ increases with higher temperature. In conclusion to that it gets clear that the scale of the measured values is right and that the transformation from current to frequency once again has cause problems in determine the gas temperature accurately.\\

Lastly we show in fig. \ref{image:gaussFit24}, \ref{image:gaussFit38} and \ref{image:gaussFit56} gaussian fit for each temperature.
\begin{center}
    \includegraphics[scale = 0.3]{Aufg-5/gaussFitTemp24.pdf}
    \captionof{figure}{gaussian fit for temperature \SI{24}{\celsius}}
    \label{image:gaussFit24}
\end{center}
\begin{center}
    \includegraphics[scale = 0.3]{Aufg-5/gaussFitTemp38.pdf}
    \captionof{figure}{gaussian fit for temperature \SI{38}{\celsius}}
    \label{image:gaussFit38}
\end{center}
\begin{center}
    \includegraphics[scale = 0.3]{Aufg-5/gaussFitTemp56.pdf}
    \captionof{figure}{gaussian fit for temperature \SI{56.2}{\celsius}}
    \label{image:gaussFit56}
\end{center}

% etc.

    % 5.Chapter Closure
    % 5. Closure

\chapter{Closure}
\label{chap:close}

In the experiment and the following evaluation of the measured data it becomes clear that the measurement have to be very precisely to detect the small transitions for the hyperfine structure. In the evaluation itself we learned how to fit functions on our measured data, and we get a good overview in the handling of precise measured data.\\
Overall this experiment gave us a good insight into the topic of saturation spectroscopy and it was a pleasure to see how theory becomes reality.

    % Appendix
    %% Appendix

\appendix

% Text

% Charlotte Geiger - Manuel Lippert - Leonard Schatt
% Physikalisches Praktikum

% Anhang A

\chapter{Berechnungen Fourier-Reihenkoeffizient und Effektivspannung}
\label{app:Berechnung}

\section*{Sinusschwingung}
Effektivspannung:
\begin{gather}
    U_{eff} = \sqrt{\frac{1}{T}\int^T_0 U_0^2 \sin^2\left(\frac{2\pi}{T}\right) dt} = \sqrt{\frac{U_0^2}{T} \left[\frac{t}{2} - \frac{T\sin(\frac{4\pi}{T}t)}{8\pi}\bigg \vert^T_0 \right]} = \sqrt{\frac{U_0^2}{T}\frac{T}{2}} = \frac{U_0}{\sqrt{2}}
\end{gather}

\section*{Rechteckschwingung}
$b_k$ der Fourier-Reihe:
\begin{gather}
    \begin{aligned}
        b_k &= \frac{2}{T} \int^{T}_{0} f(t)\sin(k \frac{2\pi}{T} t)dt\\
            &= \frac{2}{T} \left[ \int^{\frac{T}{2}}_{0} U_0\sin(k \frac{2\pi}{T} t)dt - \int^{T}_{\frac{T}{2}} U_0\sin(k \frac{2\pi}{T} t)dt\right]\\
            &= \frac{2}{T}\frac{U_0T}{k2\pi} \left[-\cos(k \frac{2\pi}{T} t) \bigg \vert^{\frac{T}{2}}_{0} + \cos(k \frac{2\pi}{T} t) \bigg \vert^{T}_{\frac{T}{2}} \right]\\
            &= \frac{U_0}{k\pi} \left[-\cos(k\pi)+1 + 1 - \cos(k\pi)\right]\\
            &= \frac{2U_0}{k\pi}\left[1-\cos(k\pi)\right]
    \end{aligned}\\[0,5cm]
    \Rightarrow b_k =
    \begin{cases}
        0, & k~\text{gerade}\\
        \frac{4U_0}{\pi}\frac{1}{k}, & k~\text{ungerade}\\
    \end{cases}
\end{gather}

Effektivspannung:
\begin{gather}
    U_{eff} = \sqrt{\frac{1}{T}\left[\int^{\frac{T}{2}}_0 U_0^2dt + \int^T_{\frac{T}{2}} U_0^2 dt\right]} = \sqrt{\frac{U_0^2}{T}\left[\frac{T}{2}+T-\frac{T}{2}\right]} = U_0
\end{gather}

\section*{Dreiecksschwingung}
$b_k$ der Fourier-Reihe:
\begin{gather}
    \begin{aligned}
        b_k &= \frac{2}{T} \int^{\frac{3T}{4}}_{-\frac{T}{4}} f(t)\sin(k \frac{2\pi}{T} t)dt\\
            &= \frac{2}{T} \left[\int^{\frac{T}{4}}_{-\frac{T}{4}} at\sin(k \frac{2\pi}{T} t)dt+ \int^{\frac{3T}{4}}_{\frac{T}{4}} a\left(\frac{T}{2}-t\right)\sin(k \frac{2\pi}{T} t)dt\right]\\
            &= \frac{2a}{T} \left[\int^{\frac{T}{4}}_{-\frac{T}{4}} t\sin(k \frac{2\pi}{T} t)dt - \int^{\frac{3T}{4}}_{\frac{T}{4}}t\sin(k \frac{2\pi}{T} t)dt\right] + \int^{\frac{3T}{4}}_{\frac{T}{4}}a\sin(k \frac{2\pi}{T} t)dt\\
            &= (\text{I}) + (\text{II})\\[0,5cm]
        (\text{I}) &= \frac{2a}{T} \left[\int^{\frac{T}{4}}_{-\frac{T}{4}} t\sin(k \frac{2\pi}{T} t)
            dt - \int^{\frac{3T}{4}}_{\frac{T}{4}}t\sin(k \frac{2\pi}{T} t)dt\right] \Rightarrow~\text{Partielle Integration}\\
            &= \frac{2a}{T}\frac{T}{k2\pi} \left[-t\cos(k \frac{2\pi}{T} t) \bigg \vert^{\frac{T}{4}}_{-\frac{T}{4}} + t\cos(k \frac{2\pi}{T} t) \bigg \vert^{\frac{3T}{4}}_{\frac{T}{4}}\right]\\
            &\tab+\frac{2a}{T}\frac{T}{k2\pi}\left[\int^{\frac{T}{4}}_{-\frac{T}{4}} \cos(k \frac{2\pi}{T} t)dt - \int^{\frac{3T}{4}}_{\frac{T}{4}} \cos(k \frac{2\pi}{T} t)dt\right]\\
            &= \frac{2a}{T}\frac{T}{k2\pi} \left[-\frac{T}{4}\cos(k \frac{\pi}{2}) -\frac{T}{4}\cos(- k \frac{\pi}{2}) + \frac{3T}{4} \cos(k \frac{3\pi}{2}) - \frac{T}{4} \cos(k \frac{\pi}{2}) \right]\\
            &\tab+\frac{2a}{T}\frac{T}{k2\pi}\left[\int^{\frac{T}{4}}_{-\frac{T}{4}} \cos(k \frac{2\pi}{T} t)dt - \int^{\frac{3T}{4}}_{\frac{T}{4}} \cos(k \frac{2\pi}{T} t)dt\right]\\
            &= \frac{2a}{T}\frac{T}{k2\pi}\left[\int^{\frac{T}{4}}_{-\frac{T}{4}} \cos(k \frac{2\pi}{T} t)dt - \int^{\frac{3T}{4}}_{\frac{T}{4}} \cos(k \frac{2\pi}{T} t)dt\right]\\
            &= \frac{2a}{T}\left(\frac{T}{k2\pi}\right)^2 \left[\sin(k \frac{2\pi}{T} t) \bigg \vert^{\frac{T}{4}}_{-\frac{T}{4}} - \sin(k \frac{2\pi}{T} t) \bigg \vert^{\frac{3T}{4}}_{\frac{T}{2}}\right]\\
            &= \frac{2a}{T}\left(\frac{T}{k2\pi}\right)^2 \left[\sin(k\frac{\pi}{2}) - \sin(-k\frac{\pi}{2}) - \sin(k\frac{3\pi}{2}) + \sin(k\frac{\pi}{2})\right]\\[0,5cm]
        (\text{II}) &= \int^{\frac{3T}{4}}_{\frac{T}{4}}a\sin(k \frac{2\pi}{T} t)dt = \frac{aT}{2\pi} \left[\cos(k\frac{2\pi}{T}t)\bigg \vert^{\frac{3T}{4}}_{\frac{T}{4}}\right]\\
                    &= \frac{aT}{2\pi} \left[\cos(k\frac{3\pi}{2}) - \cos(k\frac{\pi}{2})\right] = 0
    \end{aligned}\\[0,5cm]
    \Rightarrow b_k =
    \begin{cases}
        0, & k~\text{gerade}\\
        \frac{8aT}{4\pi^2}\frac{(-1)^{k-1}}{k^2} = \frac{8U_0}{\pi^2}\frac{(-1)^{k-1}}{k^2}, & k~\text{ungerade}\\
    \end{cases}
\end{gather}
Effektivspannung:
\begin{gather}
    \begin{aligned}
        U_{eff} &= \sqrt{\frac{1}{T}\left[\int^{\frac{T}{4}}_{-\frac{T}{4}} (at)^2dt + \int^{\frac{3T}{4}}_{\frac{T}{4}} \left(a\left(\frac{T}{2}-t\right)\right)^2 dt\right]}\\
                &= \sqrt{\frac{a^2}{T}\left[\int^{\frac{T}{4}}_{-\frac{T}{4}} t^2dt + \int^{\frac{3T}{4}}_{\frac{T}{4}} \left(\frac{T}{2}-t\right)^2 dt\right]}
                = \sqrt{\frac{a^2}{3T}\left[t^3 \bigg \vert^{\frac{T}{4}}_{-\frac{T}{4}} - \left(\frac{T}{2}-t\right)^3 \bigg \vert^{\frac{3T}{4}}_{\frac{T}{4}}\right]}\\
                &= \sqrt{\frac{a^2}{3T}\left[\left(\frac{T}{4}\right)^3 + \left(\frac{T}{4}\right)^3 - \left(\frac{T}{2}- \frac{3T}{4} \right)^3 +  \left(\frac{T}{2}- \frac{T}{4} \right)^3\right]}\\
                &= \sqrt{\frac{a^2}{3T}\left(\frac{T^3}{16}\right)} = \sqrt{\frac{1}{3}\left(\frac{aT}{4}\right)^2} = \sqrt{\frac{U_0^2}{3}} = \frac{U_0}{\sqrt{3}}
     \end{aligned}
\end{gather}


    % Literatur
    \bibliographystyle{Auswertung.bst}
    \bibliography{Auswertung.bib}
    
\end{document}