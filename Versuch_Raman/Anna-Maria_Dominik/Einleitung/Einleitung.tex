\chapter{Einleitung}
Der Raman-Effekt wurde bereits 1923 vorhergesagt und schließlich 1928 von  
C. V. Raman nachgewiesen. \\
Als Raman-Streuung wird die unelastische Streuung von Licht an Molekülen oder 
Festkörpern bezeichnet.
Bei einem Zusammenstoß von Licht (Photon) und Molekül kommt es zu einer Energieübertragung, diese 
kann sowohl vom Photon auf das Molekül stattfinden als auch vom Molekül 
auf das Photon. Den ersten Vorgang bezeichnet man auch als \textit{Stokes-Streuung},
während der zweite \textit{Anti-Stokes-Streuung} genannt wird. 
Je nachdem welcher Prozess stattfindet, besitzt das Photon eine höhere oder 
niedrigere Frequenz als das einfallende Licht. 
Diese Frequenzdifferenz ist spezifisch für die einzelnen zu untersuchenden Moleküle oder Atome.
Mithilfe der Raman-Spektroskopie können Raman-Spektren aufgenommen werden 
und durch diese damit sichtbaren charakteristische Frequenzunterschiede können Rückschlüsse auf das 
bestrahlte Molekül gezogen werden. 
Im aufgenommenen Spektrum sind jedoch nicht nur die Frequenzunterschiede des Raman-Effekts
sichtbar, sondern zum größten Teil (bis zu 1000-mal wahrscheinlicher) 
auch die Frequenz des eingestrahlten Lichtes, dies ist die sogenannte 
\textit{Rayleigh-Streuung}. \\

Ziel unseres Versuches ist es, dass man am Beispiel von einfachen fünfatomigen 
Molekülen Raman Spektren aufnimmt und anschließend die Moleküle auf 
ihre Symmetrie, atomaren Aufbau und deren Strukturen untersucht. 
Es werden die frequenzverschobenen Streuspektren der Moleküle 
Tetrachlormethan ($CCl_4$), Chloroform ($CHCl_3$), deuteriertem Chloroform ($CDCl_3$) 
und Bromoform ($CHBr_3$) betrachtet und deren beobachteten Raman-Linien und zum Teil 
deren Depolarisationsgrade bestimmt. 
