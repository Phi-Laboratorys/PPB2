\section{Wie groß ist der Offset des Monochromators}
Der Monochromator besitzt einen Offset der gemessenen Wellenlänge zu der realen Wellenlänge.
Dieser wird korrigiert, indem die Rayleigh-Wellenlänge auf die Wellenlänge des Lasers korrigiert wird.\\
Beispielweise wurde bei $CCl_4$ (Stokes, $0^\circ$ Polarisation) die Rayleigh Wellenlänge bei $632,4\,\text{nm}$ gemessen.
Der Offset zur Laserwellenlänge $\left(\lambda_L=632,8\,\text{nm}\right)$ beträgt: $\lambda_{off}=0,4\,\text{nm}$.\\

Folgende Offsets ergeben sich für alle Messungen:
\begin{table}[h]\begin{center}
    \subfigure[$0^\circ$ Polarisation]{\begin{tabular}{c|cc}Molekül&Stokes&Anti-Stokes\\\hline $CCl_4$&0,4&0,3\\$CHCl_3$&0,4&0,4\\$CDCl_3$&0,4&0,4\\$CHBr_3$&0,4&0,4\end{tabular}}
    \hspace{1cm}
    \subfigure[$90^\circ$ Polarisation]{\begin{tabular}{c|cc}Molekül&Stokes&Anti-Stokes\\\hline $CCl_4$&0,5&0,6\\$CHCl_3$&0,6&0,6\\$CDCl_3$&0,6&0,6\\$CHBr_3$&0,6&0,5\end{tabular}}
    \caption{Monochromator-Offset in (nm)}\end{center}
\end{table}\\
Diese Werte, werden nun zu den entsprechenden Messungen hinzuaddiert.