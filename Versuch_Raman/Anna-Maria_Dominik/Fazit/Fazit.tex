\chapter{Fazit}
Der Versuch Raman-Spektroskopie vermittelte einen tieferen Einblick in die 
Strukturen und Schwingungen von Molekülen. 
Während dem Versuch wurden Spektren der Moleküle 
Tetrachlorkohlenstoff ($CCl_4$), Chloroform ($CHCl_3$), deuteriertem Chloroform ($CDCl_3$) 
und Bromoform ($CHBr_3$) aufgenommen.
Anschließend wurden die aufgenommenen Raman-Spektren mit Literaturwerten verglichen.
Unsere gemessenen Werte stimmen gut mit den theoretischen überein.
Die Lage der Raman-Linien war, unter Berücksichtigung der Fehler, gut den 
Literaturwerten zu zuordnen. Natürlich fehlten uns einige Raman-Linien, dies ist jedoch nicht weiter 
verwunderlich, da unsere Möglichkeit die Raman-Linien zu identifizieren 
beschränkt waren. Auch die bestimmten 
Depolarisationsgrade weichen nicht stark von den theoretisch erwarteten ab.
Ähnlich war dies auch bei der Zuordnung der Molekülschwingungen, auch diese 
konnten zugeordnet werden. \\

Zusammenfassend lässt sich sagen, dass der Versuch unser Verständnis über
Molekülschwingungen sowie Molekülstrukturen erheblich verbessert hat.
 
